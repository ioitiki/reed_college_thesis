\documentclass[12pt,twoside]{reedthesis}
\usepackage{graphicx,latexsym} 
\usepackage{amssymb,amsthm,amsmath}
\usepackage{longtable,booktabs,setspace} 
\usepackage{url}
\usepackage{natbib}
\usepackage{epic}
\usepackage[justification=centering]{caption}
\usepackage{float}
\usepackage{multicol}
\usepackage[graphics]{preview}
\usepackage[svgnames]{xcolor}
\usepackage{tikz}
\usepackage{wasysym}


\usetikzlibrary{decorations.markings}
\usetikzlibrary{shapes.geometric}

\pgfdeclarelayer{edgelayer}
\pgfdeclarelayer{nodelayer}
\pgfsetlayers{edgelayer,nodelayer,main}

\definecolor{black}{rgb}{0.000,0.000,0.000}
\definecolor{red}{rgb}{1.000,0.000,0.000}
\definecolor{green}{rgb}{0.000,1.000,0.000}
\definecolor{yellow}{rgb}{1.000,1.000,0.000}
\definecolor{blue}{rgb}{0.000,0.000,1.000}

\tikzstyle{node}=[circle,fill=black,inner sep=0pt,draw=black,minimum size=3pt]
\tikzstyle{simple}=[-,draw=black,line width=1.000]
\tikzstyle{dot}=[-,draw=red,line width=1.000]
\tikzstyle{dash}=[-,draw=blue,line width=1.000]
\tikzstyle{poss}=[dashed,draw=blue,line width=1.000]
\tikzstyle{apm}=[dotted,draw=blue,line width=1.000]
\tikzstyle{bpm}=[dotted,draw=red,line width=1.000]
\tikzstyle{yo}=[dashed,draw=black,line width=1.000]

\newlength{\imagewidth}
\newlength{\imagescale}
\newtheorem{mydef}{Definition}
\newtheorem{theorem}{Theorem}
\newtheorem{lemma}{Lemma}


\PreviewEnvironment{tikzpicture}
\pagestyle{empty}
\setlength{\parskip}{0pt}

\title{A Combinatorial Game Theoretic Analysis of the Impartial Game of Dots and Boxes}
\author{Andrew P. Malkin}
\division{Mathematics and Natural Sciences}
\advisor{Joe Roberts}
\department{Mathematics}


\begin{document}


\maketitle


\tableofcontents


\chapter*{Abstract}
In this thesis we study the strategies and tactics of the game Dots and Boxes.  Drawing on the work done by \emph{Elwyn R. Berkamp}, \emph{John H. Conway}, \emph{Richard K. Guy}, and \emph{Ilan Vardi} -- as well as several others -- I have tried to compile a thorough exposition and analysis of the strategies utilized by players of all skill levels.  The goal is to show how combinatorial game theory enables one to mathematically determine if given strategies are valid, under what conditions they are viable, and how to employ them to elevate one's game. Given the complexity of Dots and Boxes, many board positions must be individually analyzed using the \emph{Nimdots Method} in order to determine the correct course of play.  This concept is is introduced in the last chapter and provides the necessary tools for further in-depth study.


\chapter*{Preface}
\begin{itemize}
\item Introduce the definition of a mathematical game.
\item Describe what combinatorial game theory is.
\item Explain why this field of math works to study Dots and Boxes.
\item Introduce the key players who developed current Dots and Boxes Strategies and where most of the material for this thesis comes from.
\item Emphasize why a compilation of the material is fruitful and interesting.
\end{itemize}


\chapter*{Conventions for Games}
Throughout this thesis there are several conventions of play that will be consistently used.  First, Player A will always be first to move and Player B will always be second to move.  This means that no matter how many turns have been played or number of boxes captured, Player A will always play the odd turns and Player B the even turns.

Secondly, when referring to the size of a Dots and Boxes board one can either give the dimensions in terms of the number of boxes or of the number of dots.  That is, a $4$x$4$ boxes board will have 16 total boxes, and $5$x$5$, or 25, dots.  For the purpose of this paper, I will always refer to the number of dots on a board and not to the number of boxes unless otherwise stated (at times it will be useful to talk about the number of boxes, but as a general rule I will use the number of dots to reference the size of a board).  However, when referring to the size of a chain I will always use the number of boxes in that chain to denote it's size -- i.e. a $4$-chain is a chain of four boxes.

Finally, in each example you will notice that moves are color coded depending on which player made them.  If \emph{\textcolor{blue}{Player A}} makes a move then it/they will be colored blue, and if it is \emph{\textcolor{red}{Player B}} the edge(s) will be colored red.  There are three different styles of lines depending on when the moves were played.  A solid edge indicates that the move was played on a previous turn, a dotted line implies that it was the most recent move(s), and when a dashed line is used it indicates that the move is being considered or not yet played.



\chapter{Game Play and Elementary Strategy}

\section{The Game of Dots and Boxes}
The rules of Dots and Boxes are not complicated and it is because of this that the game reaches so many people of all ages. Anyone can quickly learn to play Dots and Boxes at a beginner's level and enjoy the game.  However like many games that can be quickly understood or taught, such as $Go$ or $Chess$, mathematical analysis has revealed an underlying wealth of strategies and tactics that can be used efficaciously to better play.  We will see that with each new mathematical insight gained, a new higher level of play is achieved wherein players on lower levels cannot hope to compete.


The game of Dots and Boxes is played with two players on a rectangular grid of dots\footnote[1]{All the games analyzed in this thesis will be square grid games, however the game works just as well with rectangles who's sides have different lengths.}.  Typically the game is played with paper and pencil, but recently electronic versions of the game have become quite popular.  The game consists of players taking turns joining two adjacent dots with a vertical or horizontal edge -- this act of joining two dots is known as a $move$.  If a player  connects the fourth edge of a $1$x$1$ box (also known as $capturing$ a box) he scores a point, writes his initial in the box, and must move again.  When all the edges have been drawn and dots connected the player with the most points or boxes wins.  In the event that both players capture the same number of boxes then the game is scored a tie.  The following is an example game played on a $3$x$3$ board:


\begin{multicols}{3}
\begin{figure}[H]
\centering
\begin{tikzpicture}
	\begin{pgfonlayer}{nodelayer}
		\node [style=node] (0) at (-2, -0) {};
		\node [style=node] (1) at (-1, -0) {};
		\node [style=node] (2) at (0, -0) {};
		\node [style=node] (3) at (-2, -1) {};
		\node [style=node] (4) at (-1, -1) {};
		\node [style=node] (5) at (0, -1) {};
		\node [style=node] (6) at (-2, -2) {};
		\node [style=node] (7) at (-1, -2) {};
		\node [style=node] (8) at (0, -2) {};
	\end{pgfonlayer}
	\begin{pgfonlayer}{edgelayer}
		\draw [style=apm] (0) to (1);
	\end{pgfonlayer}
\end{tikzpicture}
\caption{Turn 1. \textbf{\textbf{Score}}: 0-0.}
\end{figure}

\begin{figure}[H]
\centering
\begin{tikzpicture}
	\begin{pgfonlayer}{nodelayer}
		\node [style=node] (0) at (-2, -0) {};
		\node [style=node] (1) at (-1, -0) {};
		\node [style=node] (2) at (0, -0) {};
		\node [style=node] (3) at (-2, -1) {};
		\node [style=node] (4) at (-1, -1) {};
		\node [style=node] (5) at (0, -1) {};
		\node [style=node] (6) at (-2, -2) {};
		\node [style=node] (7) at (-1, -2) {};
		\node [style=node] (8) at (0, -2) {};
	\end{pgfonlayer}
	\begin{pgfonlayer}{edgelayer}
		\draw [style=dash] (0) to (1);
		\draw [style=bpm] (7) to (8);
	\end{pgfonlayer}
\end{tikzpicture}
\caption{Turn 2. \textbf{\textbf{Score}}: 0-0.}
\end{figure}

\begin{figure}[H]
\centering
\begin{tikzpicture}
	\begin{pgfonlayer}{nodelayer}
		\node [style=node] (0) at (-2, -0) {};
		\node [style=node] (1) at (-1, -0) {};
		\node [style=node] (2) at (0, -0) {};
		\node [style=node] (3) at (-2, -1) {};
		\node [style=node] (4) at (-1, -1) {};
		\node [style=node] (5) at (0, -1) {};
		\node [style=node] (6) at (-2, -2) {};
		\node [style=node] (7) at (-1, -2) {};
		\node [style=node] (8) at (0, -2) {};
	\end{pgfonlayer}
	\begin{pgfonlayer}{edgelayer}
		\draw [style=dash] (0) to (1);
		\draw [style=dot] (7) to (8);
		\draw [style=apm] (4) to (5);
	\end{pgfonlayer}
\end{tikzpicture}
\caption{Turn 3. \textbf{\textbf{Score}}: 0-0.}
\end{figure}
\end{multicols}


\begin{multicols}{3}
\begin{figure}[H]
\centering
\begin{tikzpicture}
	\begin{pgfonlayer}{nodelayer}
		\node [style=node] (0) at (-2, -0) {};
		\node [style=node] (1) at (-1, -0) {};
		\node [style=node] (2) at (0, -0) {};
		\node [style=node] (3) at (-2, -1) {};
		\node [style=node] (4) at (-1, -1) {};
		\node [style=node] (5) at (0, -1) {};
		\node [style=node] (6) at (-2, -2) {};
		\node [style=node] (7) at (-1, -2) {};
		\node [style=node] (8) at (0, -2) {};
	\end{pgfonlayer}
	\begin{pgfonlayer}{edgelayer}
		\draw [style=dash] (0) to (1);
		\draw [style=dot] (7) to (8);
		\draw [style=dash] (4) to (5);
		\draw [style=bpm] (0) to (3);
	\end{pgfonlayer}
\end{tikzpicture}
\caption{Turn 4. \textbf{\textbf{Score}}: 0-0.}
\end{figure}

\begin{figure}[H]
\centering
\begin{tikzpicture}
	\begin{pgfonlayer}{nodelayer}
		\node [style=node] (0) at (-2, -0) {};
		\node [style=node] (1) at (-1, -0) {};
		\node [style=node] (2) at (0, -0) {};
		\node [style=node] (3) at (-2, -1) {};
		\node [style=node] (4) at (-1, -1) {};
		\node [style=node] (5) at (0, -1) {};
		\node [style=node] (6) at (-2, -2) {};
		\node [style=node] (7) at (-1, -2) {};
		\node [style=node] (8) at (0, -2) {};
	\end{pgfonlayer}
	\begin{pgfonlayer}{edgelayer}
		\draw [style=dash] (0) to (1);
		\draw [style=dot] (7) to (8);
		\draw [style=dash] (4) to (5);
		\draw [style=dot] (0) to (3);
		\draw [style=apm] (1) to (2);
	\end{pgfonlayer}
\end{tikzpicture}
\caption{Turn 5. \textbf{\textbf{Score}}: 0-0.}
\end{figure}

\begin{figure}[H]
\centering
\begin{tikzpicture}
	\begin{pgfonlayer}{nodelayer}
		\node [style=node] (0) at (-2, -0) {};
		\node [style=node] (1) at (-1, -0) {};
		\node [style=node] (2) at (0, -0) {};
		\node [style=node] (3) at (-2, -1) {};
		\node [style=node] (4) at (-1, -1) {};
		\node [style=node] (5) at (0, -1) {};
		\node [style=node] (6) at (-2, -2) {};
		\node [style=node] (7) at (-1, -2) {};
		\node [style=node] (8) at (0, -2) {};
	\end{pgfonlayer}
	\begin{pgfonlayer}{edgelayer}
		\draw [style=dash] (0) to (1);
		\draw [style=dot] (7) to (8);
		\draw [style=dash] (4) to (5);
		\draw [style=dot] (0) to (3);
		\draw [style=dash] (1) to (2);
		\draw [style=bpm] (3) to (6);
	\end{pgfonlayer}
\end{tikzpicture}
\caption{Turn 6. \textbf{\textbf{Score}}: 0-0.}
\end{figure}
\end{multicols}

\begin{multicols}{3}
\begin{figure}[H]
\centering
\begin{tikzpicture}
	\begin{pgfonlayer}{nodelayer}
		\node [style=node] (0) at (-2, -0) {};
		\node [style=node] (1) at (-1, -0) {};
		\node [style=node] (2) at (0, -0) {};
		\node [style=node] (3) at (-2, -1) {};
		\node [style=node] (4) at (-1, -1) {};
		\node [style=node] (5) at (0, -1) {};
		\node [style=node] (6) at (-2, -2) {};
		\node [style=node] (7) at (-1, -2) {};
		\node [style=node] (8) at (0, -2) {};
	\end{pgfonlayer}
	\begin{pgfonlayer}{edgelayer}
		\draw [style=dash] (0) to (1);
		\draw [style=dot] (7) to (8);
		\draw [style=dash] (4) to (5);
		\draw [style=dot] (0) to (3);
		\draw [style=dash] (1) to (2);
		\draw [style=dot] (3) to (6);
		\draw [style=apm] (6) to (7);
	\end{pgfonlayer}
\end{tikzpicture}
\caption{Turn 7. \textbf{\textbf{Score}}: 0-0.}
\end{figure}

\begin{figure}[H]
\centering
\begin{tikzpicture}
	\begin{pgfonlayer}{nodelayer}
		\node [style=node] (0) at (-2, -0) {};
		\node [style=node] (1) at (-1, -0) {};
		\node [style=node] (2) at (0, -0) {};
		\node [style=node] (3) at (-2, -1) {};
		\node [style=node] (4) at (-1, -1) {};
		\node [style=node] (5) at (0, -1) {};
		\node [style=node] (6) at (-2, -2) {};
		\node [style=node] (7) at (-1, -2) {};
		\node [style=node] (8) at (0, -2) {};
	\end{pgfonlayer}
	\begin{pgfonlayer}{edgelayer}
		\draw [style=dash] (0) to (1);
		\draw [style=dot] (7) to (8);
		\draw [style=dash] (4) to (5);
		\draw [style=dot] (0) to (3);
		\draw [style=dash] (1) to (2);
		\draw [style=dot] (3) to (6);
		\draw [style=dash] (6) to (7);
		\draw [style=bpm] (2) to (5);
	\end{pgfonlayer}
\end{tikzpicture}
\caption{Turn 8. \textbf{\textbf{Score}}: 0-0.}
\end{figure}

\begin{figure}[H]
\centering
\begin{tikzpicture}
	\begin{pgfonlayer}{nodelayer}
		\node [style=node] (0) at (-2, -0) {};
		\node [style=node] (1) at (-1, -0) {};
		\node [style=node] (2) at (0, -0) {};
		\node [style=node] (3) at (-2, -1) {};
		\node [style=node] (4) at (-1, -1) {};
		\node [style=node] (5) at (0, -1) {};
		\node [style=node] (6) at (-2, -2) {};
		\node [style=node] (7) at (-1, -2) {};
		\node [style=node] (8) at (0, -2) {};
		\node (9) at (-1.5, -0.5) {\textcolor{blue}{\textit{A}}};
		\node (10) at (-0.5, -0.5) {\textcolor{blue}{\textit{A}}};
		\node (11) at (-1.5, -1.5) {\textcolor{blue}{\textit{A}}};
		\node (12) at (-0.5, -1.5) {\textcolor{blue}{\textit{A}}};
	\end{pgfonlayer}
	\begin{pgfonlayer}{edgelayer}
		\draw [style=dash] (0) to (1);
		\draw [style=dot] (7) to (8);
		\draw [style=dash] (4) to (5);
		\draw [style=dot] (0) to (3);
		\draw [style=dash] (1) to (2);
		\draw [style=dot] (3) to (6);
		\draw [style=dash] (6) to (7);
		\draw [style=dot] (2) to (5);
		\draw [style=apm] (1) to (4);
		\draw [style=apm] (3) to (4);
		\draw [style=apm] (4) to (7);
		\draw [style=apm] (5) to (8);
	\end{pgfonlayer}
\end{tikzpicture}
\caption{Turn 9. Player A wins. \textbf{\textbf{Score}}: 4-0.}
\end{figure}
\end{multicols}


\subsection{Immediate Consequences of the Rules of Play}
There are two immediate implications that follow from the rules which need to be unpacked and discussed before further analysis is possible.

\subsubsection{Turn Parity}
Firstly, one should note that in a single turn more than one move can be played -- evident in the above game on the ninth turn.  This is due to the fact that a player is obligated to move again if he has captured a box.  In fact there can be as many as $n$ moves in a single turn, where $n$ is the total number of boxes on the board.  This result follows directly from the definition of turn.

\begin{mydef}[Turn]
A \textbf{turn} is one complete set of consecutive moves by a player.  During every non-final turn a player makes one move for each box captured\footnote[2]{As we will see later this is not completely accurate.  With the introduction of the double-cross move it will be possible to take two boxes with one move.  This implies that the number of moves in a non-final turn is equal to the number of boxes captured without the use of a double-cross, plus the number of double-crosses played, plus one. As before, the final turn will have one less move than this.} plus one additional move.  On the final turn of the game the number of moves will be equal to the number of boxes captured on that turn.
\end{mydef}

\noindent
While more than one move can be played in a given turn, the important part is that the parity of one's turn never changes.  Therefore even though Player A may make an odd or even number of moves during his turn, he will always be making moves on the $odd$ turns.  Similarly Player B will always play moves on the $even$ turns.  As we will see beginning in Chapter 2, turn parity is fundamental in all Dots and Boxes strategies.


\subsubsection{Voluntary Capture}
In the game of Dots and Boxes it is not always best to complete every available box.  While this may seem counterintuitive -- given the goal is to capture as many boxes as possible -- there are many situations that arise in which it is advantageous to sacrifice boxes that otherwise could have been taken in the course of a turn.  Nowhere in the rules does it state that one is obligated to capture a box when it is possible to do so, and thus we get the concept of a voluntary capture.  When to take a box, and when to sacrifice a box is not always obvious.  But as we will see shortly the option is extremely powerful.

As with turn parity however, this heuristic becomes useful only after a deeper analysis of the game.


\section{Chains}
As two players set out to play a game of Dots and Boxes, both attempting to make moves which will prevent their opponent from capturing a box, one quickly sees that chains of boxes form.  Whether it is a $2$-chain or a long-chain, the creation of chains inevitably occurs in every game, no matter what skill levels the players have.  Intentionally manipulating the number and size of chains on a given board is the governing tactic in expert Dots and Boxes strategy.

\begin{mydef}[Chain]
A \textbf{chain} is a connected string of boxes wherein any move made in the chain allows the other player to capture all the connected boxes.
\end{mydef}

\noindent
Below are several example chains that come up in typical Dots and Boxes games:

\begin{multicols}{3}
\begin{figure}[H]
\centering
\begin{tikzpicture}
	\begin{pgfonlayer}{nodelayer}
		\node [style=node] (0) at (-2, -0) {};
		\node [style=node] (1) at (-1, -0) {};
		\node [style=node] (2) at (0, -0) {};
		\node [style=node] (3) at (-2, -1) {};
		\node [style=node] (4) at (-1, -1) {};
		\node [style=node] (5) at (0, -1) {};
		\node (6) at (-1, 1) {};
	\end{pgfonlayer}
	\begin{pgfonlayer}{edgelayer}
		\draw [style=simple] (0) to (2);
		\draw [style=simple] (3) to (5);
	\end{pgfonlayer}
\end{tikzpicture}
\caption{$2$-chain.}
\end{figure}

\begin{figure}[H]
\centering
\begin{tikzpicture}
	\begin{pgfonlayer}{nodelayer}
		\node [style=node] (0) at (-3, -0) {};
		\node [style=node] (1) at (-2, -0) {};
		\node [style=node] (2) at (-1, -0) {};
		\node [style=node] (3) at (0, -0) {};
		\node [style=node] (4) at (0, 1) {};
		\node [style=node] (5) at (-1, 1) {};
		\node [style=node] (6) at (-1, -1) {};
		\node [style=node] (7) at (-2, -1) {};
		\node [style=node] (8) at (-3, -1) {};
		\node [style=node] (9) at (0, -1) {};
		\node [style=node] (10) at (-1, -0) {};
	\end{pgfonlayer}
	\begin{pgfonlayer}{edgelayer}
		\draw [style=simple] (8) to (9);
		\draw [style=simple] (9) to (4);
		\draw [style=simple] (5) to (2);
		\draw [style=simple] (2) to (0);
	\end{pgfonlayer}
\end{tikzpicture}
\caption{$4$-chain.}
\end{figure}


\begin{figure}[H]
\centering
\begin{tikzpicture}
	\begin{pgfonlayer}{nodelayer}
		\node [style=node] (0) at (-2, 1) {};
		\node [style=node] (1) at (-1, 1) {};
		\node [style=node] (2) at (0, 1) {};
		\node [style=node] (3) at (1, 1) {};
		\node [style=node] (4) at (-2, -0) {};
		\node [style=node] (5) at (-1, -0) {};
		\node [style=node] (6) at (0, -0) {};
		\node [style=node] (7) at (1, -0) {};
		\node [style=node] (8) at (-2, -1) {};
		\node [style=node] (9) at (-1, -1) {};
		\node [style=node] (10) at (0, -1) {};
		\node [style=node] (11) at (1, -1) {};
	\end{pgfonlayer}
	\begin{pgfonlayer}{edgelayer}
		\draw [style=simple] (0) to (3);
		\draw [style=simple] (5) to (7);
		\draw [style=simple] (8) to (11);
		\draw [style=simple] (0) to (8);
	\end{pgfonlayer}
\end{tikzpicture}
\caption{$6$-chain.}
\end{figure}
\end{multicols}

Let's examine one of the above chains to better understand how they work. Take Figure 1.11, the chain of length four.  No matter where Player A draws  an edge,  he has opened a box, allowing all the boxes in that chain to be captured on the next turn by Player B. Figure 1.13 illustrates this fact:

\begin{figure}[H]
\centering
\begin{tikzpicture}
	\begin{pgfonlayer}{nodelayer}
		\node [style=node] (0) at (-3, 1) {};
		\node [style=node] (1) at (-2, 1) {};
		\node [style=node] (2) at (-1, 1) {};
		\node [style=node] (3) at (0, 1) {};
		\node [style=node] (4) at (-1, 2) {};
		\node [style=node] (5) at (0, 2) {};
		\node [style=node] (6) at (0, -0) {};
		\node [style=node] (7) at (-1, -0) {};
		\node [style=node] (8) at (-2, -0) {};
		\node [style=node] (9) at (-3, -0) {};
		\node [style=node] (10) at (1, 1) {};
		\node [style=node] (11) at (1, -0) {};
		\node [style=node] (12) at (2, 1) {};
		\node [style=node] (13) at (2, -0) {};
		\node [style=node] (14) at (3, 1) {};
		\node [style=node] (15) at (3, -0) {};
		\node [style=node] (16) at (4, 1) {};
		\node [style=node] (17) at (4, -0) {};
		\node [style=node] (18) at (3, 2) {};
		\node [style=node] (19) at (4, 2) {};
		\node [style=node] (20) at (-4, 1) {};
		\node [style=node] (21) at (-4, -0) {};
		\node [style=node] (22) at (-5, -0) {};
		\node [style=node] (23) at (-5, 1) {};
		\node [style=node] (24) at (-5, 2) {};
		\node [style=node] (25) at (-4, 2) {};
		\node [style=node] (26) at (-6, 1) {};
		\node [style=node] (27) at (-7, 1) {};
		\node [style=node] (28) at (-7, -0) {};
		\node [style=node] (29) at (-6, -0) {};
		\node [style=node] (30) at (-5, -1) {};
		\node [style=node] (31) at (-4, -1) {};
		\node [style=node] (32) at (-4, -2) {};
		\node [style=node] (33) at (-5, -2) {};
		\node [style=node] (34) at (-4, -3) {};
		\node [style=node] (35) at (-5, -3) {};
		\node [style=node] (36) at (-6, -2) {};
		\node [style=node] (37) at (-6, -3) {};
		\node [style=node] (38) at (-7, -2) {};
		\node [style=node] (39) at (-7, -3) {};
		\node [style=node] (40) at (-3, -2) {};
		\node [style=node] (41) at (-3, -3) {};
		\node [style=node] (42) at (-2, -2) {};
		\node [style=node] (43) at (-2, -3) {};
		\node [style=node] (44) at (-1, -2) {};
		\node [style=node] (45) at (-1, -3) {};
		\node [style=node] (46) at (0, -3) {};
		\node [style=node] (47) at (0, -2) {};
		\node [style=node] (48) at (0, -1) {};
		\node [style=node] (49) at (-1, -1) {};
		\node [style=node] (50) at (-5, 1) {};
		\node (51) at (-6.5, 0.5) {\textcolor{red}{\textit{B}}};
		\node (52) at (-5.5, 0.5) {\textcolor{red}{\textit{B}}};
		\node (53) at (-4.5, 0.5) {\textcolor{red}{\textit{B}}};
		\node (54) at (-4.5, 1.5) {\textcolor{red}{\textit{B}}};
		\node (55) at (-2.5, 0.5) {\textcolor{red}{\textit{B}}};
		\node (56) at (-1.5, 0.5) {\textcolor{red}{\textit{B}}};
		\node (57) at (-0.5, 0.5) {\textcolor{red}{\textit{B}}};
		\node (58) at (-0.5, 1.5) {\textcolor{red}{\textit{B}}};
		\node (59) at (1.5, 0.5) {\textcolor{red}{\textit{B}}};
		\node (60) at (2.5, 0.5) {\textcolor{red}{\textit{B}}};
		\node (61) at (3.5, 0.5) {\textcolor{red}{\textit{B}}};
		\node (62) at (3.5, 1.5) {\textcolor{red}{\textit{B}}};
		\node (63) at (-6.5, -2.5) {\textcolor{red}{\textit{B}}};
		\node (64) at (-5.5, -2.5) {\textcolor{red}{\textit{B}}};
		\node (65) at (-4.5, -2.5) {\textcolor{red}{\textit{B}}};
		\node (66) at (-4.5, -1.5) {\textcolor{red}{\textit{B}}};
		\node (67) at (-2.5, -2.5) {\textcolor{red}{\textit{B}}};
		\node (68) at (-1.5, -2.5) {\textcolor{red}{\textit{B}}};
		\node (69) at (-0.5, -2.5) {\textcolor{red}{\textit{B}}};
		\node (70) at (-0.5, -1.5) {\textcolor{red}{\textit{B}}};
	\end{pgfonlayer}
	\begin{pgfonlayer}{edgelayer}
		\draw [style=simple] (27) to (23);
		\draw [style=simple] (28) to (21);
		\draw [style=simple] (25) to (21);
		\draw [style=simple] (24) to (23);
		\draw [style=simple] (0) to (2);
		\draw [style=simple] (9) to (6);
		\draw [style=simple] (5) to (6);
		\draw [style=simple] (4) to (2);
		\draw [style=simple] (10) to (14);
		\draw [style=simple] (11) to (17);
		\draw [style=simple] (19) to (17);
		\draw [style=simple] (18) to (14);
		\draw [style=simple] (38) to (33);
		\draw [style=simple] (39) to (34);
		\draw [style=simple] (31) to (34);
		\draw [style=simple] (30) to (33);
		\draw [style=simple] (40) to (44);
		\draw [style=simple] (41) to (46);
		\draw [style=simple] (48) to (46);
		\draw [style=simple] (49) to (44);
		\draw [style=dash] (38) to (39);
		\draw [style=dash] (42) to (43);
		\draw [style=dash] (23) to (22);
		\draw [style=dash] (2) to (3);
		\draw [style=dash] (18) to (19);
		\draw [style=bpm] (23) to (20);
		\draw [style=bpm] (24) to (25);
		\draw [style=bpm] (26) to (29);
		\draw [style=bpm] (27) to (28);
		\draw [style=bpm] (0) to (9);
		\draw [style=bpm] (1) to (8);
		\draw [style=bpm] (2) to (7);
		\draw [style=bpm] (4) to (5);
		\draw [style=bpm] (10) to (11);
		\draw [style=bpm] (12) to (13);
		\draw [style=bpm] (14) to (15);
		\draw [style=bpm] (14) to (16);
		\draw [style=bpm] (36) to (37);
		\draw [style=bpm] (33) to (35);
		\draw [style=bpm] (33) to (32);
		\draw [style=bpm] (30) to (31);
		\draw [style=bpm] (40) to (41);
		\draw [style=bpm] (44) to (45);
		\draw [style=bpm] (44) to (47);
		\draw [style=bpm] (49) to (48);
	\end{pgfonlayer}
\end{tikzpicture}
\caption{The completion of Player B's turn given Player A makes any available move.}
\end{figure}

Not all chains are created equal, and as we will see in the section on long-chains the size of a chain determines how it can be used effectively to manipulate play.  But for now it is good to understand that chains form in every Dots and Boxes game, and their existence dictates strategic decisions on every turn.

\section{Cycles}
Cycles are a special case of chains, and like chains play a central role in the theory of Dots and Boxes.  No matter where an edge is drawn in a cycle all the remaining boxes become available on the next turn.  However unlike chains, cycles occur much less often, and analysis of their effects can be slightly more complicated.  This is not to say that cycles are any less important than chains -- and in fact, in order to master the game one must have a full understanding of \emph{both} cycles and chains, and the various options available when confronted with either.

\begin{mydef}[Cycle]
A \textbf{cycle} is a closed loop of four or more boxes in which exactly two edges have been drawn for each box contained in the loop.
\end{mydef}

\begin{multicols}{3}
\begin{figure}[H]
\centering
\begin{tikzpicture}
	\begin{pgfonlayer}{nodelayer}
		\node [style=node] (0) at (-2, -0) {};
		\node [style=node] (1) at (-1, -0) {};
		\node [style=node] (2) at (0, -0) {};
		\node [style=node] (3) at (0, -1) {};
		\node [style=node] (4) at (0, -2) {};
		\node [style=node] (5) at (-1, -2) {};
		\node [style=node] (6) at (-2, -2) {};
		\node [style=node] (7) at (-2, -1) {};
		\node [style=node] (8) at (-1, -1) {};
		\node (9) at (-1, 1) {};
	\end{pgfonlayer}
	\begin{pgfonlayer}{edgelayer}
		\draw [style=simple] (0) to (2);
		\draw [style=simple] (2) to (4);
		\draw [style=simple] (4) to (6);
		\draw [style=simple] (6) to (0);
	\end{pgfonlayer}
\end{tikzpicture}
\caption{$4$-Cycle.}
\end{figure}

\begin{figure}[H]
\centering
\begin{tikzpicture}
	\begin{pgfonlayer}{nodelayer}
		\node [style=node] (0) at (-5, -0) {};
		\node [style=node] (1) at (-4, -0) {};
		\node [style=node] (2) at (-3, -0) {};
		\node [style=node] (3) at (-2, -0) {};
		\node [style=node] (4) at (-1, -0) {};
		\node [style=node] (5) at (0, -0) {};
		\node [style=node] (6) at (-5, -1) {};
		\node [style=node] (7) at (-5, -2) {};
		\node [style=node] (8) at (-4, -2) {};
		\node [style=node] (9) at (-3, -2) {};
		\node [style=node] (10) at (-2, -2) {};
		\node [style=node] (11) at (-1, -2) {};
		\node [style=node] (12) at (0, -2) {};
		\node [style=node] (13) at (0, -1) {};
		\node [style=node] (14) at (-4, -1) {};
		\node [style=node] (15) at (-3, -1) {};
		\node [style=node] (16) at (-2, -1) {};
		\node [style=node] (17) at (-1, -1) {};
		\node (9) at (-1, 1) {};
	\end{pgfonlayer}
	\begin{pgfonlayer}{edgelayer}
		\draw [style=simple] (0) to (5);
		\draw [style=simple] (5) to (12);
		\draw [style=simple] (0) to (7);
		\draw [style=simple] (7) to (12);
		\draw [style=simple] (14) to (17);
	\end{pgfonlayer}
\end{tikzpicture}
\caption{$10$-Cycle}
\end{figure}

\begin{figure}[H]
\centering
\begin{tikzpicture}
	\begin{pgfonlayer}{nodelayer}
		\node [style=node] (0) at (-3, -0) {};
		\node [style=node] (1) at (-3, -1) {};
		\node [style=node] (2) at (-3, -2) {};
		\node [style=node] (3) at (-3, -3) {};
		\node [style=node] (4) at (-2, -3) {};
		\node [style=node] (5) at (-1, -3) {};
		\node [style=node] (6) at (0, -3) {};
		\node [style=node] (7) at (0, -2) {};
		\node [style=node] (8) at (-1, -2) {};
		\node [style=node] (9) at (-2, -2) {};
		\node [style=node] (10) at (-2, -1) {};
		\node [style=node] (11) at (-1, -1) {};
		\node [style=node] (12) at (-2, -0) {};
		\node [style=node] (13) at (-1, -0) {};
		\node [style=node] (14) at (0, -0) {};
		\node [style=node] (15) at (0, -1) {};
		\node (16) at (-1.5, -1.5) {\textcolor{blue}{\textit{A}}};
	\end{pgfonlayer}
	\begin{pgfonlayer}{edgelayer}
		\draw [style=simple] (0) to (3);
		\draw [style=simple] (3) to (6);
		\draw [style=simple] (6) to (14);
		\draw [style=simple] (0) to (14);
		\draw [style=simple] (10) to (9);
		\draw [style=simple] (10) to (11);
		\draw [style=simple] (11) to (8);
		\draw [style=simple] (9) to (8);
	\end{pgfonlayer}
\end{tikzpicture}
\caption{$8$-Cycle.}
\end{figure}
\end{multicols}

The above figures illustrate several possible cycles that can form in a game of Dots and Boxes.  It is not hard for one to see that any move made in one of these cycles completes the third edge of a box, creating a chain reaction, that enables their opponent to capture every box in the cycle on the following turn if desired.

More often than not cycles function like chains when determining the `correct' course of play.  Because of this it is helpful to introduce both at the same time, but one must always be aware that they can have different effects on the outcome of a game depending on how they are used.  Many players will occasionally blunder when confronted with a cycle due to this fact.

\section{The Greedy Strategy}
Now that the rules have been established and the basic patterns of play determined (i.e. edges are drawn and chains and/or cycles formed), it is time to begin looking at strategies of play.  We start by examining the most basic strategy employed by novice players. 

When two amateurs set out to play Dots and Boxes the game goes something like this.  Each player takes turns making moves with one goal in mind -- don't give up a box until it is the only move available.  So initial moves appear random.  Players connect dots forming chains or cycles being careful not to draw the third edge of any box which would give their opponent the opportunity to capture a point.  This continues until the board is saturated with chains and the only moves left are those which will complete the third edge of a box.  At this point, whoever is forced to open the first chain usually studies the board and determines which chain will give his opponent the fewest boxes.  After this quick analysis he then opens the smallest chain giving away the fewest boxes, and in his mind, maximizing the number of boxes he will get in return -- as every remaining chain has at least that number of boxes and probably more.  In this fashion the two amateur players trade chains from smallest to largest until the final turn is played and the score is reckoned.

The following example from \emph{Winning Ways} shows how this strategy plays out on a $6$x$6$ board.  It is Player B's turn, and the only moves available open a chain for Player A.  The size of each chain is labeled on the first board to indicate the order of chains from smallest to largest.

\begin{multicols}{2}
\begin{figure}[H]
\centering
\begin{tikzpicture}
	\begin{pgfonlayer}{nodelayer}
		\node [style=node] (0) at (-0.5, -0.5) {};
		\node [style=node] (1) at (0.5, -0.5) {};
		\node [style=node] (2) at (1.5, -0.5) {};
		\node [style=node] (3) at (2.5, -0.5) {};
		\node [style=node] (4) at (2.5, 0.5) {};
		\node [style=node] (5) at (1.5, 0.5) {};
		\node [style=node] (6) at (0.5, 0.5) {};
		\node [style=node] (7) at (-0.5, 0.5) {};
		\node [style=node] (8) at (-0.5, -1.5) {};
		\node [style=node] (9) at (0.5, -1.5) {};
		\node [style=node] (10) at (1.5, -1.5) {};
		\node [style=node] (11) at (2.5, -1.5) {};
		\node [style=node] (12) at (2.5, -2.5) {};
		\node [style=node] (13) at (1.5, -2.5) {};
		\node [style=node] (14) at (0.5, -2.5) {};
		\node [style=node] (15) at (-0.5, -2.5) {};
		\node [style=node] (16) at (-1.5, -2.5) {};
		\node [style=node] (17) at (-2.5, -2.5) {};
		\node [style=node] (18) at (-2.5, -1.5) {};
		\node [style=node] (19) at (-1.5, -1.5) {};
		\node [style=node] (20) at (-2.5, -0.5) {};
		\node [style=node] (21) at (-1.5, -0.5) {};
		\node [style=node] (22) at (-2.5, 0.5) {};
		\node [style=node] (23) at (-1.5, 0.5) {};
		\node [style=node] (24) at (-2.5, 1.5) {};
		\node [style=node] (25) at (-1.5, 1.5) {};
		\node [style=node] (26) at (-0.5, 1.5) {};
		\node [style=node] (27) at (0.5, 1.5) {};
		\node [style=node] (28) at (1.5, 1.5) {};
		\node [style=node] (29) at (2.5, 1.5) {};
		\node [style=node] (30) at (2.5, 2.5) {};
		\node [style=node] (31) at (1.5, 2.5) {};
		\node [style=node] (32) at (0.5, 2.5) {};
		\node [style=node] (33) at (-0.5, 2.5) {};
		\node [style=node] (34) at (-1.5, 2.5) {};
		\node [style=node] (35) at (-2.5, 2.5) {};
		\node [style=node] (36) at (-1.5, 2.5) {};
		\node at (-1.5,2) {\textit{3-Boxes}};
		\node at (-.5,1) {\textit{9-Boxes}};
		\node at (.5,0) {\textit{8-Boxes}};
		\node at (1.5,-1) {\textit{5-Boxes}};
	\end{pgfonlayer}
	\begin{pgfonlayer}{edgelayer}
		\draw [style=simple] (35) to (34);
		\draw [style=simple] (35) to (24);
		\draw [style=simple] (22) to (23);
		\draw [style=simple] (23) to (25);
		\draw [style=simple] (25) to (26);
		\draw [style=simple] (26) to (33);
		\draw [style=simple] (33) to (31);
		\draw [style=simple] (31) to (28);
		\draw [style=simple] (30) to (4);
		\draw [style=simple] (7) to (4);
		\draw [style=simple] (27) to (6);
		\draw [style=simple] (7) to (0);
		\draw [style=simple] (20) to (0);
		\draw [style=simple] (20) to (17);
		\draw [style=simple] (17) to (16);
		\draw [style=simple] (8) to (15);
		\draw [style=simple] (19) to (9);
		\draw [style=simple] (9) to (1);
		\draw [style=simple] (1) to (3);
		\draw [style=simple] (3) to (12);
		\draw [style=simple] (10) to (13);
		\draw [style=simple] (13) to (14);
	\end{pgfonlayer}
\end{tikzpicture}
\caption{Player B's turn to move on a board where any move opens a chain.}
\end{figure}

\begin{figure}[H]
\centering
\begin{tikzpicture}
	\begin{pgfonlayer}{nodelayer}
		\node [style=node] (0) at (-5, -0) {};
		\node [style=node] (1) at (-4, -0) {};
		\node [style=node] (2) at (-3, -0) {};
		\node [style=node] (3) at (-2, -0) {};
		\node [style=node] (4) at (-1, -0) {};
		\node [style=node] (5) at (0, -0) {};
		\node [style=node] (6) at (-5, -1) {};
		\node [style=node] (7) at (-4, -1) {};
		\node [style=node] (8) at (-3, -1) {};
		\node [style=node] (9) at (-2, -1) {};
		\node [style=node] (10) at (-1, -1) {};
		\node [style=node] (11) at (0, -1) {};
		\node [style=node] (12) at (-5, -2) {};
		\node [style=node] (13) at (-4, -2) {};
		\node [style=node] (14) at (-3, -2) {};
		\node [style=node] (15) at (-2, -2) {};
		\node [style=node] (16) at (-1, -2) {};
		\node [style=node] (17) at (0, -2) {};
		\node [style=node] (18) at (-5, -3) {};
		\node [style=node] (19) at (-4, -3) {};
		\node [style=node] (20) at (-3, -3) {};
		\node [style=node] (21) at (-2, -3) {};
		\node [style=node] (22) at (-1, -3) {};
		\node [style=node] (23) at (0, -3) {};
		\node [style=node] (24) at (-5, -4) {};
		\node [style=node] (25) at (-4, -4) {};
		\node [style=node] (26) at (-3, -4) {};
		\node [style=node] (27) at (-2, -4) {};
		\node [style=node] (28) at (-1, -4) {};
		\node [style=node] (29) at (0, -4) {};
		\node [style=node] (30) at (-5, -5) {};
		\node [style=node] (31) at (-4, -5) {};
		\node [style=node] (32) at (-3, -5) {};
		\node [style=node] (33) at (-2, -5) {};
		\node [style=node] (34) at (-1, -5) {};
		\node [style=node] (35) at (0, -5) {};
	\end{pgfonlayer}
	\begin{pgfonlayer}{edgelayer}
		\draw [style=simple] (6) to (0);
		\draw [style=simple] (0) to (1);
		\draw [style=simple] (12) to (13);
		\draw [style=simple] (13) to (7);
		\draw [style=simple] (7) to (8);
		\draw [style=simple] (8) to (2);
		\draw [style=simple] (2) to (3);
		\draw [style=simple] (3) to (4);
		\draw [style=simple] (4) to (10);
		\draw [style=simple] (5) to (11);
		\draw [style=simple] (11) to (17);
		\draw [style=simple] (16) to (17);
		\draw [style=simple] (16) to (15);
		\draw [style=simple] (15) to (9);
		\draw [style=simple] (15) to (14);
		\draw [style=simple] (14) to (20);
		\draw [style=simple] (20) to (19);
		\draw [style=simple] (19) to (18);
		\draw [style=simple] (18) to (24);
		\draw [style=simple] (24) to (30);
		\draw [style=simple] (30) to (31);
		\draw [style=simple] (25) to (26);
		\draw [style=simple] (26) to (32);
		\draw [style=simple] (26) to (27);
		\draw [style=simple] (27) to (21);
		\draw [style=simple] (21) to (22);
		\draw [style=simple] (22) to (23);
		\draw [style=simple] (23) to (29);
		\draw [style=simple] (29) to (35);
		\draw [style=simple] (28) to (34);
		\draw [style=simple] (34) to (33);
		\draw [style=bpm] (6) to (12);
	\end{pgfonlayer}
\end{tikzpicture}
\caption{Player B opens the smallest chain first in accordance with the \emph{Greedy Strategy}. \textbf{\textbf{Score}}: 0 - 0.}
\end{figure}
\end{multicols}

\begin{multicols}{2}
\begin{figure}[H]
\centering
\begin{tikzpicture}
	\begin{pgfonlayer}{nodelayer}
		\node [style=node] (0) at (-5, -0) {};
		\node [style=node] (1) at (-4, -0) {};
		\node [style=node] (2) at (-3, -0) {};
		\node [style=node] (3) at (-2, -0) {};
		\node [style=node] (4) at (-1, -0) {};
		\node [style=node] (5) at (0, -0) {};
		\node [style=node] (6) at (-5, -1) {};
		\node [style=node] (7) at (-4, -1) {};
		\node [style=node] (8) at (-3, -1) {};
		\node [style=node] (9) at (-2, -1) {};
		\node [style=node] (10) at (-1, -1) {};
		\node [style=node] (11) at (0, -1) {};
		\node [style=node] (12) at (-5, -2) {};
		\node [style=node] (13) at (-4, -2) {};
		\node [style=node] (14) at (-3, -2) {};
		\node [style=node] (15) at (-2, -2) {};
		\node [style=node] (16) at (-1, -2) {};
		\node [style=node] (17) at (0, -2) {};
		\node [style=node] (18) at (-5, -3) {};
		\node [style=node] (19) at (-4, -3) {};
		\node [style=node] (20) at (-3, -3) {};
		\node [style=node] (21) at (-2, -3) {};
		\node [style=node] (22) at (-1, -3) {};
		\node [style=node] (23) at (0, -3) {};
		\node [style=node] (24) at (-5, -4) {};
		\node [style=node] (25) at (-4, -4) {};
		\node [style=node] (26) at (-3, -4) {};
		\node [style=node] (27) at (-2, -4) {};
		\node [style=node] (28) at (-1, -4) {};
		\node [style=node] (29) at (0, -4) {};
		\node [style=node] (30) at (-5, -5) {};
		\node [style=node] (31) at (-4, -5) {};
		\node [style=node] (32) at (-3, -5) {};
		\node [style=node] (33) at (-2, -5) {};
		\node [style=node] (34) at (-1, -5) {};
		\node [style=node] (35) at (0, -5) {};
		\node (36) at (-4.5, -1.5) {\textcolor{blue}{\textit{A}}};
		\node (37) at (-4.5, -0.5) {\textcolor{blue}{\textit{A}}};
		\node (38) at (-3.5, -0.5) {\textcolor{blue}{\textit{A}}};
	\end{pgfonlayer}
	\begin{pgfonlayer}{edgelayer}
		\draw [style=simple] (6) to (0);
		\draw [style=simple] (0) to (1);
		\draw [style=simple] (12) to (13);
		\draw [style=simple] (13) to (7);
		\draw [style=simple] (7) to (8);
		\draw [style=simple] (8) to (2);
		\draw [style=simple] (2) to (3);
		\draw [style=simple] (3) to (4);
		\draw [style=simple] (4) to (10);
		\draw [style=simple] (5) to (11);
		\draw [style=simple] (11) to (17);
		\draw [style=simple] (16) to (17);
		\draw [style=simple] (16) to (15);
		\draw [style=simple] (15) to (9);
		\draw [style=simple] (15) to (14);
		\draw [style=simple] (14) to (20);
		\draw [style=simple] (20) to (19);
		\draw [style=simple] (19) to (18);
		\draw [style=simple] (18) to (24);
		\draw [style=simple] (24) to (30);
		\draw [style=simple] (30) to (31);
		\draw [style=simple] (25) to (26);
		\draw [style=simple] (26) to (32);
		\draw [style=simple] (26) to (27);
		\draw [style=simple] (27) to (21);
		\draw [style=simple] (21) to (22);
		\draw [style=simple] (22) to (23);
		\draw [style=simple] (23) to (29);
		\draw [style=simple] (29) to (35);
		\draw [style=simple] (28) to (34);
		\draw [style=simple] (34) to (33);
		\draw [style=dot] (6) to (12);
		\draw [style=apm] (6) to (7);
		\draw [style=apm] (7) to (1);
		\draw [style=apm] (1) to (2);
		\draw [style=apm] (34) to (35);
	\end{pgfonlayer}
\end{tikzpicture}
\caption{Player A responds by taking all three boxes and opening the next smallest chain. \textbf{Score}: 3 - 0.}
\end{figure}


\begin{figure}[H]
\centering
\begin{tikzpicture}
	\begin{pgfonlayer}{nodelayer}
		\node [style=node] (0) at (-5, -0) {};
		\node [style=node] (1) at (-4, -0) {};
		\node [style=node] (2) at (-3, -0) {};
		\node [style=node] (3) at (-2, -0) {};
		\node [style=node] (4) at (-1, -0) {};
		\node [style=node] (5) at (0, -0) {};
		\node [style=node] (6) at (-5, -1) {};
		\node [style=node] (7) at (-4, -1) {};
		\node [style=node] (8) at (-3, -1) {};
		\node [style=node] (9) at (-2, -1) {};
		\node [style=node] (10) at (-1, -1) {};
		\node [style=node] (11) at (0, -1) {};
		\node [style=node] (12) at (-5, -2) {};
		\node [style=node] (13) at (-4, -2) {};
		\node [style=node] (14) at (-3, -2) {};
		\node [style=node] (15) at (-2, -2) {};
		\node [style=node] (16) at (-1, -2) {};
		\node [style=node] (17) at (0, -2) {};
		\node [style=node] (18) at (-5, -3) {};
		\node [style=node] (19) at (-4, -3) {};
		\node [style=node] (20) at (-3, -3) {};
		\node [style=node] (21) at (-2, -3) {};
		\node [style=node] (22) at (-1, -3) {};
		\node [style=node] (23) at (0, -3) {};
		\node [style=node] (24) at (-5, -4) {};
		\node [style=node] (25) at (-4, -4) {};
		\node [style=node] (26) at (-3, -4) {};
		\node [style=node] (27) at (-2, -4) {};
		\node [style=node] (28) at (-1, -4) {};
		\node [style=node] (29) at (0, -4) {};
		\node [style=node] (30) at (-5, -5) {};
		\node [style=node] (31) at (-4, -5) {};
		\node [style=node] (32) at (-3, -5) {};
		\node [style=node] (33) at (-2, -5) {};
		\node [style=node] (34) at (-1, -5) {};
		\node [style=node] (35) at (0, -5) {};
		\node (36) at (-4.5, -1.5) {\textcolor{blue}{\textit{A}}};
		\node (37) at (-4.5, -0.5) {\textcolor{blue}{\textit{A}}};
		\node (38) at (-3.5, -0.5) {\textcolor{blue}{\textit{A}}};
		\node (39) at (-2.5, -4.5) {\textcolor{red}{\textit{B}}};
		\node (40) at (-1.5, -4.5) {\textcolor{red}{\textit{B}}};
		\node (41) at (-0.5, -4.5) {\textcolor{red}{\textit{B}}};
		\node (42) at (-1.5, -3.5) {\textcolor{red}{\textit{B}}};
		\node (43) at (-0.5, -3.5) {\textcolor{red}{\textit{B}}};
	\end{pgfonlayer}
	\begin{pgfonlayer}{edgelayer}
		\draw [style=simple] (6) to (0);
		\draw [style=simple] (0) to (1);
		\draw [style=simple] (12) to (13);
		\draw [style=simple] (13) to (7);
		\draw [style=simple] (7) to (8);
		\draw [style=simple] (8) to (2);
		\draw [style=simple] (2) to (3);
		\draw [style=simple] (3) to (4);
		\draw [style=simple] (4) to (10);
		\draw [style=simple] (5) to (11);
		\draw [style=simple] (11) to (17);
		\draw [style=simple] (16) to (17);
		\draw [style=simple] (16) to (15);
		\draw [style=simple] (15) to (9);
		\draw [style=simple] (15) to (14);
		\draw [style=simple] (14) to (20);
		\draw [style=simple] (20) to (19);
		\draw [style=simple] (19) to (18);
		\draw [style=simple] (18) to (24);
		\draw [style=simple] (24) to (30);
		\draw [style=simple] (30) to (31);
		\draw [style=simple] (25) to (26);
		\draw [style=simple] (26) to (32);
		\draw [style=simple] (26) to (27);
		\draw [style=simple] (27) to (21);
		\draw [style=simple] (21) to (22);
		\draw [style=simple] (22) to (23);
		\draw [style=simple] (23) to (29);
		\draw [style=simple] (29) to (35);
		\draw [style=simple] (28) to (34);
		\draw [style=simple] (34) to (33);
		\draw [style=dot] (6) to (12);
		\draw [style=dash] (6) to (7);
		\draw [style=dash] (7) to (1);
		\draw [style=dash] (1) to (2);
		\draw [style=dash] (34) to (35);
		\draw [style=bpm] (28) to (29);
		\draw [style=bpm] (22) to (28);
		\draw [style=bpm] (27) to (28);
		\draw [style=bpm] (27) to (33);
		\draw [style=bpm] (32) to (33);
		\draw [style=bpm] (17) to (23);
	\end{pgfonlayer}
\end{tikzpicture}
\caption{Player B takes all five boxes and again opens the next smallest chain. \textbf{Score}: 3 - 5.}
\end{figure}
\end{multicols}

\begin{multicols}{2}
\begin{figure}[H]
\centering
\begin{tikzpicture}
	\begin{pgfonlayer}{nodelayer}
		\node [style=node] (0) at (-5, -0) {};
		\node [style=node] (1) at (-4, -0) {};
		\node [style=node] (2) at (-3, -0) {};
		\node [style=node] (3) at (-2, -0) {};
		\node [style=node] (4) at (-1, -0) {};
		\node [style=node] (5) at (0, -0) {};
		\node [style=node] (6) at (-5, -1) {};
		\node [style=node] (7) at (-4, -1) {};
		\node [style=node] (8) at (-3, -1) {};
		\node [style=node] (9) at (-2, -1) {};
		\node [style=node] (10) at (-1, -1) {};
		\node [style=node] (11) at (0, -1) {};
		\node [style=node] (12) at (-5, -2) {};
		\node [style=node] (13) at (-4, -2) {};
		\node [style=node] (14) at (-3, -2) {};
		\node [style=node] (15) at (-2, -2) {};
		\node [style=node] (16) at (-1, -2) {};
		\node [style=node] (17) at (0, -2) {};
		\node [style=node] (18) at (-5, -3) {};
		\node [style=node] (19) at (-4, -3) {};
		\node [style=node] (20) at (-3, -3) {};
		\node [style=node] (21) at (-2, -3) {};
		\node [style=node] (22) at (-1, -3) {};
		\node [style=node] (23) at (0, -3) {};
		\node [style=node] (24) at (-5, -4) {};
		\node [style=node] (25) at (-4, -4) {};
		\node [style=node] (26) at (-3, -4) {};
		\node [style=node] (27) at (-2, -4) {};
		\node [style=node] (28) at (-1, -4) {};
		\node [style=node] (29) at (0, -4) {};
		\node [style=node] (30) at (-5, -5) {};
		\node [style=node] (31) at (-4, -5) {};
		\node [style=node] (32) at (-3, -5) {};
		\node [style=node] (33) at (-2, -5) {};
		\node [style=node] (34) at (-1, -5) {};
		\node [style=node] (35) at (0, -5) {};
		\node (36) at (-4.5, -1.5) {\textcolor{blue}{\textit{A}}};
		\node (37) at (-4.5, -0.5) {\textcolor{blue}{\textit{A}}};
		\node (38) at (-3.5, -0.5) {\textcolor{blue}{\textit{A}}};
		\node (39) at (-2.5, -4.5) {\textcolor{red}{\textit{B}}};
		\node (40) at (-1.5, -4.5) {\textcolor{red}{\textit{B}}};
		\node (41) at (-0.5, -4.5) {\textcolor{red}{\textit{B}}};
		\node (42) at (-1.5, -3.5) {\textcolor{red}{\textit{B}}};
		\node (43) at (-0.5, -3.5) {\textcolor{red}{\textit{B}}};
		\node (44) at (-0.5, -2.5) {\textcolor{blue}{\textit{A}}};
		\node (45) at (-1.5, -2.5) {\textcolor{blue}{\textit{A}}};
		\node (46) at (-2.5, -2.5) {\textcolor{blue}{\textit{A}}};
		\node (47) at (-3.5, -3.5) {\textcolor{blue}{\textit{A}}};
		\node (48) at (-4.5, -3.5) {\textcolor{blue}{\textit{A}}};
		\node (49) at (-4.5, -4.5) {\textcolor{blue}{\textit{A}}};
		\node (50) at (-3.5, -4.5) {\textcolor{blue}{\textit{A}}};
		\node (51) at (-2.5, -3.5) {\textcolor{blue}{\textit{A}}};
	\end{pgfonlayer}
	\begin{pgfonlayer}{edgelayer}
		\draw [style=simple] (6) to (0);
		\draw [style=simple] (0) to (1);
		\draw [style=simple] (12) to (13);
		\draw [style=simple] (13) to (7);
		\draw [style=simple] (7) to (8);
		\draw [style=simple] (8) to (2);
		\draw [style=simple] (2) to (3);
		\draw [style=simple] (3) to (4);
		\draw [style=simple] (4) to (10);
		\draw [style=simple] (5) to (11);
		\draw [style=simple] (11) to (17);
		\draw [style=simple] (16) to (17);
		\draw [style=simple] (16) to (15);
		\draw [style=simple] (15) to (9);
		\draw [style=simple] (15) to (14);
		\draw [style=simple] (14) to (20);
		\draw [style=simple] (20) to (19);
		\draw [style=simple] (19) to (18);
		\draw [style=simple] (18) to (24);
		\draw [style=simple] (24) to (30);
		\draw [style=simple] (30) to (31);
		\draw [style=simple] (25) to (26);
		\draw [style=simple] (26) to (32);
		\draw [style=simple] (26) to (27);
		\draw [style=simple] (27) to (21);
		\draw [style=simple] (21) to (22);
		\draw [style=simple] (22) to (23);
		\draw [style=simple] (23) to (29);
		\draw [style=simple] (29) to (35);
		\draw [style=simple] (28) to (34);
		\draw [style=simple] (34) to (33);
		\draw [style=dot] (6) to (12);
		\draw [style=dash] (6) to (7);
		\draw [style=dash] (7) to (1);
		\draw [style=dash] (1) to (2);
		\draw [style=dash] (34) to (35);
		\draw [style=dot] (28) to (29);
		\draw [style=dot] (22) to (28);
		\draw [style=dot] (27) to (28);
		\draw [style=dot] (27) to (33);
		\draw [style=dot] (32) to (33);
		\draw [style=dot] (17) to (23);
		\draw [style=apm] (16) to (22);
		\draw [style=apm] (15) to (21);
		\draw [style=apm] (20) to (21);
		\draw [style=apm] (20) to (26);
		\draw [style=apm] (19) to (25);
		\draw [style=apm] (24) to (25);
		\draw [style=apm] (25) to (31);
		\draw [style=apm] (31) to (32);
		\draw [style=apm] (12) to (18);
	\end{pgfonlayer}
\end{tikzpicture}
\caption{Player A takes the eight boxes and moves into the final chain. \newline \textbf{Score}: 11 - 5.}
\end{figure}


\begin{figure}[H]
\centering
\begin{tikzpicture}
	\begin{pgfonlayer}{nodelayer}
		\node [style=node] (0) at (-5, -0) {};
		\node [style=node] (1) at (-4, -0) {};
		\node [style=node] (2) at (-3, -0) {};
		\node [style=node] (3) at (-2, -0) {};
		\node [style=node] (4) at (-1, -0) {};
		\node [style=node] (5) at (0, -0) {};
		\node [style=node] (6) at (-5, -1) {};
		\node [style=node] (7) at (-4, -1) {};
		\node [style=node] (8) at (-3, -1) {};
		\node [style=node] (9) at (-2, -1) {};
		\node [style=node] (10) at (-1, -1) {};
		\node [style=node] (11) at (0, -1) {};
		\node [style=node] (12) at (-5, -2) {};
		\node [style=node] (13) at (-4, -2) {};
		\node [style=node] (14) at (-3, -2) {};
		\node [style=node] (15) at (-2, -2) {};
		\node [style=node] (16) at (-1, -2) {};
		\node [style=node] (17) at (0, -2) {};
		\node [style=node] (18) at (-5, -3) {};
		\node [style=node] (19) at (-4, -3) {};
		\node [style=node] (20) at (-3, -3) {};
		\node [style=node] (21) at (-2, -3) {};
		\node [style=node] (22) at (-1, -3) {};
		\node [style=node] (23) at (0, -3) {};
		\node [style=node] (24) at (-5, -4) {};
		\node [style=node] (25) at (-4, -4) {};
		\node [style=node] (26) at (-3, -4) {};
		\node [style=node] (27) at (-2, -4) {};
		\node [style=node] (28) at (-1, -4) {};
		\node [style=node] (29) at (0, -4) {};
		\node [style=node] (30) at (-5, -5) {};
		\node [style=node] (31) at (-4, -5) {};
		\node [style=node] (32) at (-3, -5) {};
		\node [style=node] (33) at (-2, -5) {};
		\node [style=node] (34) at (-1, -5) {};
		\node [style=node] (35) at (0, -5) {};
		\node (36) at (-4.5, -1.5) {\textcolor{blue}{\textit{A}}};
		\node (37) at (-4.5, -0.5) {\textcolor{blue}{\textit{A}}};
		\node (38) at (-3.5, -0.5) {\textcolor{blue}{\textit{A}}};
		\node (39) at (-2.5, -4.5) {\textcolor{red}{\textit{B}}};
		\node (40) at (-1.5, -4.5) {\textcolor{red}{\textit{B}}};
		\node (41) at (-0.5, -4.5) {\textcolor{red}{\textit{B}}};
		\node (42) at (-1.5, -3.5) {\textcolor{red}{\textit{B}}};
		\node (43) at (-0.5, -3.5) {\textcolor{red}{\textit{B}}};
		\node (44) at (-0.5, -2.5) {\textcolor{blue}{\textit{A}}};
		\node (45) at (-1.5, -2.5) {\textcolor{blue}{\textit{A}}};
		\node (46) at (-2.5, -2.5) {\textcolor{blue}{\textit{A}}};
		\node (47) at (-3.5, -3.5) {\textcolor{blue}{\textit{A}}};
		\node (48) at (-4.5, -3.5) {\textcolor{blue}{\textit{A}}};
		\node (49) at (-4.5, -4.5) {\textcolor{blue}{\textit{A}}};
		\node (50) at (-3.5, -4.5) {\textcolor{blue}{\textit{A}}};
		\node (51) at (-2.5, -3.5) {\textcolor{blue}{\textit{A}}};
		\node (52) at (-4.5, -2.5) {\textcolor{red}{\textit{B}}};
		\node (53) at (-3.5, -2.5) {\textcolor{red}{\textit{B}}};
		\node (54) at (-3.5, -1.5) {\textcolor{red}{\textit{B}}};
		\node (55) at (-2.5, -1.5) {\textcolor{red}{\textit{B}}};
		\node (56) at (-2.5, -0.5) {\textcolor{red}{\textit{B}}};
		\node (57) at (-1.5, -0.5) {\textcolor{red}{\textit{B}}};
		\node (58) at (-1.5, -1.5) {\textcolor{red}{\textit{B}}};
		\node (59) at (-0.5, -0.5) {\textcolor{red}{\textit{B}}};
		\node (60) at (-0.5, -1.5) {\textcolor{red}{\textit{B}}};
	\end{pgfonlayer}
	\begin{pgfonlayer}{edgelayer}
		\draw [style=simple] (6) to (0);
		\draw [style=simple] (0) to (1);
		\draw [style=simple] (12) to (13);
		\draw [style=simple] (13) to (7);
		\draw [style=simple] (7) to (8);
		\draw [style=simple] (8) to (2);
		\draw [style=simple] (2) to (3);
		\draw [style=simple] (3) to (4);
		\draw [style=simple] (4) to (10);
		\draw [style=simple] (5) to (11);
		\draw [style=simple] (11) to (17);
		\draw [style=simple] (16) to (17);
		\draw [style=simple] (16) to (15);
		\draw [style=simple] (15) to (9);
		\draw [style=simple] (15) to (14);
		\draw [style=simple] (14) to (20);
		\draw [style=simple] (20) to (19);
		\draw [style=simple] (19) to (18);
		\draw [style=simple] (18) to (24);
		\draw [style=simple] (24) to (30);
		\draw [style=simple] (30) to (31);
		\draw [style=simple] (25) to (26);
		\draw [style=simple] (26) to (32);
		\draw [style=simple] (26) to (27);
		\draw [style=simple] (27) to (21);
		\draw [style=simple] (21) to (22);
		\draw [style=simple] (22) to (23);
		\draw [style=simple] (23) to (29);
		\draw [style=simple] (29) to (35);
		\draw [style=simple] (28) to (34);
		\draw [style=simple] (34) to (33);
		\draw [style=dot] (6) to (12);
		\draw [style=dash] (6) to (7);
		\draw [style=dash] (7) to (1);
		\draw [style=dash] (1) to (2);
		\draw [style=dash] (34) to (35);
		\draw [style=dot] (28) to (29);
		\draw [style=dot] (22) to (28);
		\draw [style=dot] (27) to (28);
		\draw [style=dot] (27) to (33);
		\draw [style=dot] (32) to (33);
		\draw [style=dot] (17) to (23);
		\draw [style=dash] (16) to (22);
		\draw [style=dash] (15) to (21);
		\draw [style=dash] (20) to (21);
		\draw [style=dash] (20) to (26);
		\draw [style=dash] (19) to (25);
		\draw [style=dash] (24) to (25);
		\draw [style=dash] (25) to (31);
		\draw [style=dash] (31) to (32);
		\draw [style=dash] (12) to (18);
		\draw [style=bpm] (13) to (19);
		\draw [style=bpm] (13) to (14);
		\draw [style=bpm] (14) to (8);
		\draw [style=bpm] (8) to (9);
		\draw [style=bpm] (9) to (3);
		\draw [style=bpm] (9) to (10);
		\draw [style=bpm] (10) to (16);
		\draw [style=bpm] (10) to (11);
		\draw [style=bpm] (4) to (5);
	\end{pgfonlayer}
\end{tikzpicture}
\caption{Player B then takes all the remaining boxes winning the game. \newline \textbf{Score}: 11 - 14.}
\end{figure}
\end{multicols}


With no knowledge of turn parity or voluntary capture our two amateur players greedily sacrifice the smallest chains they can and hope, in return, to receive more boxes than their opponent.  This basic strategy of Dots and Boxes, appropriately known as the Greedy Strategy, is by far the least intellectual approach to the game; requiring very little combinatorial game theory to analyze positions and their outcomes.  

We will quickly see that while Player A loses using the Greedy Strategy in this example\footnote[3]{Note, it is possible to win using the Greedy Strategy -- case-in-point Player B wins using the same strategy -- however it is impossible to control the outcome of the game which is needed to consistently win.}, he will win every time when intermediate tactics are employed given the same conditions.



\chapter{Understanding Control}
The first chapter outlined the rules of play and basic moves utilized in the game of Dots and Boxes.  This lead to our first explicit strategy -- the Greedy Strategy.  With this strategy we saw that there was very little one could do to control the outcome of the game, given there were no offensive moves forcing specific desirable outcomes.  Instead, one could only wait and hope that when it came time to trade chains they ended up with the majority of the boxes winning the game.

In this chapter we begin to see that there exist certain tactics which will guarantee favorable outcomes allowing one to take control of a game and keep it.  The main concept that allows one to do this is the before mentioned voluntary capture consequence of the rules of play, and in particular the resulting double-dealing move.  It is very powerful to be able to deny boxes in order to force your opponent into desired positions.

Through careful analysis we shall see how moves transform from being ends in themselves -- i.e. immediately capturing boxes or preventing boxes from being captured -- to a means for accomplishing an overarching winning strategy.  Henceforth moves will be analyzed as such; allowing multi-turn strategies to emerge and be studied.  We begin this chapter by categorizing these moves (based on their effects, limitations, and uses),  and end it with a review of control and how to keep it.


\section{The Double-dealing Move}
Here we examine one of the most crucial moves in Dots and Boxes known as the double-dealing move.  Relying on the notion of voluntary capture, this move is a central component of all intermediate and expert Dots and Boxes strategies enabling one to control the outcome of a game.

\begin{mydef}[Double-dealing Move]
A \textbf{double-dealing move} is the act of sacrificing the last two boxes of an opened chain by completing the third edge of the last box in the chain.
\end{mydef}

\noindent
Consider the following 3-chain that has just been opened by Player A in Figure 2.1.

\begin{multicols}{2}
\begin{figure}[H]
\centering
\begin{tikzpicture}
	\begin{pgfonlayer}{nodelayer}
		\node [style=node] (0) at (-3, -0) {};
		\node [style=node] (1) at (-2, -0) {};
		\node [style=node] (2) at (-1, -0) {};
		\node [style=node] (3) at (0, -0) {};
		\node [style=node] (4) at (-3, -1) {};
		\node [style=node] (5) at (-2, -1) {};
		\node [style=node] (6) at (-1, -1) {};
		\node [style=node] (7) at (0, -1) {};
	\end{pgfonlayer}
	\begin{pgfonlayer}{edgelayer}
		\draw [style=simple] (0) to (1);
		\draw [style=simple] (1) to (2);
		\draw [style=simple] (2) to (3);
		\draw [style=simple] (4) to (5);
		\draw [style=simple] (5) to (6);
		\draw [style=simple] (6) to (7);
		\draw [style=apm] (0) to (4);
	\end{pgfonlayer}
\end{tikzpicture}
\caption{}
\end{figure}

\begin{figure}[H]
\centering
\begin{tikzpicture}
	\begin{pgfonlayer}{nodelayer}
		\node [style=node] (0) at (-3, -0) {};
		\node [style=node] (1) at (-2, -0) {};
		\node [style=node] (2) at (-1, -0) {};
		\node [style=node] (3) at (0, -0) {};
		\node [style=node] (4) at (-3, -1) {};
		\node [style=node] (5) at (-2, -1) {};
		\node [style=node] (6) at (-1, -1) {};
		\node [style=node] (7) at (0, -1) {};
		\node (8) at (-2.5, -0.5) {\textcolor{red}{\textit{B}}};
	\end{pgfonlayer}
	\begin{pgfonlayer}{edgelayer}
		\draw [style=simple] (0) to (1);
		\draw [style=simple] (1) to (2);
		\draw [style=simple] (2) to (3);
		\draw [style=simple] (4) to (5);
		\draw [style=simple] (5) to (6);
		\draw [style=simple] (6) to (7);
		\draw [style=dash] (0) to (4);
		\draw [style=bpm] (1) to (5);
		\draw [style=bpm] (3) to (7);
	\end{pgfonlayer}
\end{tikzpicture}
\caption{}
\end{figure}
\end{multicols}

\noindent
Player B can now capture the box with three edges and end his turn with a double-dealing move by joining the two far right dots with a vertical edge -- Figure 2.2.  Given Player B has not completed a box with his final move, he does not have to move again, and in fact \emph{can't} make another move on this turn.  Player A will then take the two sacrificed boxes -- assuming he acts rationally -- and make a final move in another section of the board.

The act of capturing two boxes with one edge is known as a \textbf{double-cross move} and will always follow a double-dealing move under rational play.  While scoring two points by completing the sacrificed boxes is always a good thing (i.e. adds to your total score), moving again will almost always benefit your opponent if he has used the double-dealing move correctly.

When to play a double-dealing move will depend completely on an analysis of the game from that turn on, and in effect allows one to choose which side to play as for the remainder of the game.  We prove this result in the section on \emph{loony-moves} later in the chapter.

\subsubsection{Rational Play: Accepting Sacrificed Boxes}
\begin{theorem}
If a player is able to capture a box or a pair of boxes with the first move of his turn, then he should do so.
\end{theorem}

\begin{proof}
Assume Player A has made the third edge of a box with the final move of a turn.  Then Player B has two options: either he captures the box and moves again in another section of the board, or he denies the box and makes his first move somewhere else on the board.  Either way Player B will have to make the move in another section of the board, and thus there is no penalty for accepting the opened box.  If he does not accept the open box, then Player A will be able to do so on the following turn with no penalty using the same logic.  Therefore, while Player B is not \emph{forced} to take the box that Player A opened under the rules of play, he \emph{should} do so given it adds to his final score without any detriment.\footnote[1]{Given this theorem holds in any situation, we will assume from this point on that players will play rationally and accept sacrificed boxes whenever possible.}
\end{proof}

\subsection{The Cycle Double-dealing Move}
It is possible to play an equivalent form of the double-dealing move when faced with a cycle.  However instead of sacrificing two boxes, as with the chain, one has to sacrifice the last four boxes of a cycle in order to have the same effect.  This means that two double-cross moves will be played following a \textbf{cycle double-dealing move}.

In order to see how this works, consider a 6-cycle that has just been opened by Player A (Figure 2.3).

\begin{multicols}{2}
\begin{figure}[H]
\centering
\begin{tikzpicture}
	\begin{pgfonlayer}{nodelayer}
		\node [style=node] (0) at (-1.5, 1) {};
		\node [style=node] (1) at (-0.5, 1) {};
		\node [style=node] (2) at (0.5, 1) {};
		\node [style=node] (3) at (1.5, 1) {};
		\node [style=node] (4) at (-1.5, -0) {};
		\node [style=node] (5) at (-0.5, -0) {};
		\node [style=node] (6) at (0.5, -0) {};
		\node [style=node] (7) at (1.5, -0) {};
		\node [style=node] (8) at (-1.5, -1) {};
		\node [style=node] (9) at (-0.5, -1) {};
		\node [style=node] (10) at (0.5, -1) {};
		\node [style=node] (11) at (1.5, -1) {};
	\end{pgfonlayer}
	\begin{pgfonlayer}{edgelayer}
		\draw [style=simple] (0) to (3);
		\draw [style=simple] (3) to (11);
		\draw [style=simple] (8) to (11);
		\draw [style=simple] (0) to (8);
		\draw [style=simple] (5) to (6);
		\draw [style=apm] (1) to (5);
	\end{pgfonlayer}
\end{tikzpicture}
\caption{}
\end{figure}

\begin{figure}[H]
\centering
\begin{tikzpicture}
	\begin{pgfonlayer}{nodelayer}
		\node [style=node] (0) at (-1.5, 1) {};
		\node [style=node] (1) at (-0.5, 1) {};
		\node [style=node] (2) at (0.5, 1) {};
		\node [style=node] (3) at (1.5, 1) {};
		\node [style=node] (4) at (-1.5, -0) {};
		\node [style=node] (5) at (-0.5, -0) {};
		\node [style=node] (6) at (0.5, -0) {};
		\node [style=node] (7) at (1.5, -0) {};
		\node [style=node] (8) at (-1.5, -1) {};
		\node [style=node] (9) at (-0.5, -1) {};
		\node [style=node] (10) at (0.5, -1) {};
		\node [style=node] (11) at (1.5, -1) {};
		\node (12) at (0, 0.5) {\textcolor{red}{\textit{B}}};
		\node (13) at (1, 0.5) {\textcolor{red}{\textit{B}}};
	\end{pgfonlayer}
	\begin{pgfonlayer}{edgelayer}
		\draw [style=simple] (0) to (3);
		\draw [style=simple] (3) to (11);
		\draw [style=simple] (8) to (11);
		\draw [style=simple] (0) to (8);
		\draw [style=simple] (5) to (6);
		\draw [style=dash] (1) to (5);
		\draw [style=bpm] (2) to (6);
		\draw [style=bpm] (6) to (7);
		\draw [style=bpm] (5) to (9);
	\end{pgfonlayer}
\end{tikzpicture}
\caption{}
\end{figure}
\end{multicols}

From Figure 2.4 we see that Player B is able to able to produce the same results, as the double-dealing move for chains produces, with the cycle double-dealing move by leaving two double-cross moves for Player A.  Therefore the cycle double-dealing move can be analyzed exactly the same way as a double-dealing move is analyzed, with the added fact that an extra double-cross move will be played.


\section{Why 3 is Long: Short-Chain Moves}
Given our new understanding of voluntary capture and the double-dealing move we can now revisit chains and discuss why all chains are not created equal.  

Consider a chain of length three that Player A may be forced to open on his next turn.  No matter what move Player A chooses to play in that chain, there is always the possibility of Player B making a double-dealing final move.

\begin{theorem}
Given a player opens a chain of length at least three, his opponent always has an opportunity to play a double-dealing final move on his next turn.
\end{theorem}

\begin{proof}
Suppose there is a chain of length three that has just been opened by Player A.  Let the opening move be any of the four possible edges.  This means that a third edge of an end box (either the first box in the chain or the third box) has been played.  Therefore, Player B can complete the fourth edge of the end box and make his final move completing the third edge of the other end box; thus playing a double-dealing move.
\end{proof}

\noindent
This is true of all chains of three or more boxes\footnote[3]{The proof of this is the same as the 3-chain proof, once you realize that all chains longer than three boxes reduce to a 3-chain with an end box completed, as the chain is being captured.}, but once the chain is smaller than three boxes it is not always possible to end one's turn with a double-dealing move.  This yields the following definition:

\begin{mydef}[Long-Chain]
A chain consisting of three or more boxes.
\end{mydef}

Let us now consider a chain of length two to see why what we said above is true.  Suppose Player A must open a 2-chain and does not want to give Player B the option of playing a double-dealing move on his next turn.  Well then, Player A can always insert the middle edge of the 2-chain playing what is known as a \textbf{hard-hearted handout} (Figure 2.5), thereby eliminating the possibility of Player B making a double-dealing move in that chain. 

On the other hand, if Player A wants to give Player B the option of making a double-dealing move on his next turn he can always play what is called a \textbf{half-hearted handout} as is seen in Figure 2.6.  As we will see in the next section, however, this move is rarely played as it gives no tactical advantage to Player A.

\begin{figure}[H]
\centering
\begin{tikzpicture}
	\begin{pgfonlayer}{nodelayer}
		\node [style=node] (0) at (5, -0) {};
		\node [style=node] (1) at (5, -1) {};
		\node [style=node] (2) at (4, -0) {};
		\node [style=node] (3) at (4, -1) {};
		\node [style=node] (4) at (3, -0) {};
		\node [style=node] (5) at (3, -1) {};
		\node [style=node] (6) at (1, -0) {};
		\node [style=node] (7) at (1, -1) {};
		\node [style=node] (8) at (0, -0) {};
		\node [style=node] (9) at (0, -1) {};
		\node [style=node] (10) at (-1, -0) {};
		\node [style=node] (11) at (-1, -1) {};
		\node [style=node] (12) at (-3, -0) {};
		\node [style=node] (13) at (-3, -1) {};
		\node [style=node] (14) at (-4, -0) {};
		\node [style=node] (15) at (-4, -1) {};
		\node [style=node] (16) at (-5, -1) {};
		\node [style=node] (17) at (-5, -0) {};
	\end{pgfonlayer}
	\begin{pgfonlayer}{edgelayer}
		\draw [style=simple] (17) to (16);
		\draw [style=simple] (16) to (13);
		\draw [style=simple] (13) to (12);
		\draw [style=simple] (10) to (6);
		\draw [style=simple] (11) to (7);
		\draw [style=simple] (4) to (2);
		\draw [style=simple] (5) to (1);
		\draw [style=simple] (1) to (0);
		\draw [style=apm] (14) to (15);
		\draw [style=apm] (8) to (9);
		\draw [style=apm] (2) to (3);
	\end{pgfonlayer}
\end{tikzpicture}
\caption{Hard-hearted Handouts.}
\end{figure}

\begin{figure}[H]
\centering
\begin{tikzpicture}
	\begin{pgfonlayer}{nodelayer}
		\node [style=node] (0) at (5, 0) {};
		\node [style=node] (1) at (5, -1) {};
		\node [style=node] (2) at (4, 0) {};
		\node [style=node] (3) at (4, -1) {};
		\node [style=node] (4) at (3, 0) {};
		\node [style=node] (5) at (3, -1) {};
		\node [style=node] (6) at (1, 0) {};
		\node [style=node] (7) at (1, -1) {};
		\node [style=node] (8) at (0, 0) {};
		\node [style=node] (9) at (0, -1) {};
		\node [style=node] (10) at (-1, 0) {};
		\node [style=node] (11) at (-1, -1) {};
		\node [style=node] (12) at (-3, 0) {};
		\node [style=node] (13) at (-3, -1) {};
		\node [style=node] (14) at (-4, 0) {};
		\node [style=node] (15) at (-4, -1) {};
		\node [style=node] (16) at (-5, -1) {};
		\node [style=node] (17) at (-5, 0) {};
		\node [style=node] (18) at (-5, -1) {};
	\end{pgfonlayer}
	\begin{pgfonlayer}{edgelayer}
		\draw [style=simple] (17) to (16);
		\draw [style=simple] (16) to (13);
		\draw [style=simple] (13) to (12);
		\draw [style=simple] (10) to (6);
		\draw [style=simple] (11) to (7);
		\draw [style=simple] (4) to (2);
		\draw [style=simple] (5) to (1);
		\draw [style=simple] (1) to (0);
		\draw [style=apm] (14) to (12);
		\draw [style=apm] (10) to (11);
		\draw [style=apm] (4) to (5);
	\end{pgfonlayer}
\end{tikzpicture}
\caption{Half-hearted Handouts.}
\end{figure}

\subsection{Long-Cycles}
Unlike chains, it is possible to play a cycle double-dealing move given any cycle that arises in a game of Dots and Boxes.  This is due to the fact that no cycle can be created that is smaller than four boxes.  Thus every cycle in a Dots and Boxes game is considered long.

The proof of this is similar to the proof used to show why one can always play a double-dealing move given a long-chain, and is left as an exercise for the reader.

\section{Loony-Moves}
In the last few sections we have been discussing the double-dealing move and several situations when one can use or deny it, without much attention to why it is so effective.  In this section we discuss one of the reasons for it's effectiveness, and, more specifically, when one should play a double-dealing move if given the option.

\begin{mydef}[Loony-Move]
Any move after which the other player, on the next turn, can make a double-dealing move or cycle double-dealing move.
\end{mydef}

Thus, a loony-move is a move that leaves the board so that the opposing player has the \emph{option} of playing a double-dealing move on the following turn.  These moves include opening a long-chain or cycle and the half-hearted handout.  In all three cases the player who has made the loony-move has given their opponent the option of going first for the rest of the game, or going second for the rest of the game.  

To understand this consider what happens when a long-chain is opened.  You can either complete the entire chain and be the first person to move in the game from that point on.  Or you can play a double-dealing move and force your opponent to be the first person to move in the game from that point on.  The same argument works to show this result for the remaining two loony-moves.

\subsection{The Loony Theorem}
The above result leads to a central theorem in Dots and Boxes which I will call the Loony Theorem.

\begin{theorem}
\emph{ (The Loony Theorem) }
If a player has made a loony-move, then their opponent can score at least half of the unclaimed boxes remaining. 
\end{theorem}

\begin{proof}
Without loss of generality assume Player A has just made a loony-move.  Then, Player B has the option of capturing all the boxes offered and playing first in the rest of the game, or he can make a double-dealing move and play second in the rest of the game.  Thus all Player B has to do is decide which player -- either the first or second player in the rest of the game -- will score more boxes, and choose the appropriate final move.  Given Dots and Boxes is a game of perfect information it is always possible to perform an exhaustive analysis to determine the results under best play\footnote[4]{This can be shown using \emph{Zermelo's Theorem}, which states that in a game of perfect information with finite moves there exists a non-losing strategy for one of the players.  A proof of this can be carried out inducting on the set of all possible moves from every position.}.  In the case when the exhaustive analysis results in a tie, Player B can choose either side and score half the boxes.  Therefore, in all cases Player B can score at least half of the boxes in the rest of the game.
\end{proof}

\subsection{Consequences of The Loony Theorem}

\subsubsection{Constructive Use}
The proof of the Loony Theorem rests on the fact that one can determine which side will score more boxes in the remaining game, and while this is true, it is not always practical.  For instance we can imagine a game on a 10x10 board in which a half-hearted handout is played on the fifth move.  In this case, Player B should be able to score at least half the the 81 boxes according to the Loony Theorem.  However the number of permutations that Player B would have to consider in order to determine the results under best play is too large to analyze while playing a game, and thus is not a fruitful way to evaluate his position.  This implies that in \emph{practice}, the Loony Theorem can only be utilized when there are a relatively small number of moves to consider in the remaining game.

\subsubsection{Normal Play}
Given the results of playing a loony-move, it is not hard to see that one is rarely played until it is forced.  This suggests that in most cases when the first loony-move is played, the only moves left on the board are moves that will open long-chains or cycles.  This is an important consequence of the Loony Theorem and leads to what is known as Normal Play.  Normal play will be further discussed in Section 3.1.


\section{Control}
Now that we have clarified the types of tactical moves available in Dots and Boxes, and seen -- from the Loony Theorem -- a formal explanation on when to use a double-dealing move, we can explore the concept of control and its effects.

\begin{mydef}[Control]
A player is said to have \textbf{control} in a game if the opponent has made the first loony-move on a board that has only long-chains or cycles left.
\end{mydef}

Given that the definition of control states there are only long-chains or cycles left on the board, we can infer that whoever has control can keep it by playing a double-dealing move.

\begin{theorem}
If a player has control in a game of Dots and Boxes it is possible to keep control until the end of that game. 
\end{theorem}

\begin{proof}
Suppose Player A has control in a game, and captures all the boxes in the opened chain but the last two by playing a double-dealing final move.  This will be followed immediately by a double-cross move by Player B -- capturing the last two boxes in the same chain.  At this point the only moves played have been in the initial open chain and the rest of the board is left unchanged.  Thus the remaining board has only long-chains or cycles, and Player B is first to move.  If Player B opens a cycle instead of a chain, the same argument applies.
\end{proof}

With this in mind, let us reexamine the game from Section 1.4 to see how Player A can take advantage of control and win the game.  As before it is Player B to move, and the only moves available open a long-chain:

\begin{multicols}{2}
\begin{figure}[H]
\centering
\begin{tikzpicture}
	\begin{pgfonlayer}{nodelayer}
		\node [style=node] (0) at (-5, -0) {};
		\node [style=node] (1) at (-4, -0) {};
		\node [style=node] (2) at (-3, -0) {};
		\node [style=node] (3) at (-2, -0) {};
		\node [style=node] (4) at (-1, -0) {};
		\node [style=node] (5) at (0, -0) {};
		\node [style=node] (6) at (-5, -1) {};
		\node [style=node] (7) at (-4, -1) {};
		\node [style=node] (8) at (-3, -1) {};
		\node [style=node] (9) at (-2, -1) {};
		\node [style=node] (10) at (-1, -1) {};
		\node [style=node] (11) at (0, -1) {};
		\node [style=node] (12) at (-5, -2) {};
		\node [style=node] (13) at (-4, -2) {};
		\node [style=node] (14) at (-3, -2) {};
		\node [style=node] (15) at (-2, -2) {};
		\node [style=node] (16) at (-1, -2) {};
		\node [style=node] (17) at (0, -2) {};
		\node [style=node] (18) at (-5, -3) {};
		\node [style=node] (19) at (-4, -3) {};
		\node [style=node] (20) at (-3, -3) {};
		\node [style=node] (21) at (-2, -3) {};
		\node [style=node] (22) at (-1, -3) {};
		\node [style=node] (23) at (0, -3) {};
		\node [style=node] (24) at (-5, -4) {};
		\node [style=node] (25) at (-4, -4) {};
		\node [style=node] (26) at (-3, -4) {};
		\node [style=node] (27) at (-2, -4) {};
		\node [style=node] (28) at (-1, -4) {};
		\node [style=node] (29) at (0, -4) {};
		\node [style=node] (30) at (-5, -5) {};
		\node [style=node] (31) at (-4, -5) {};
		\node [style=node] (32) at (-3, -5) {};
		\node [style=node] (33) at (-2, -5) {};
		\node [style=node] (34) at (-1, -5) {};
		\node [style=node] (35) at (0, -5) {};
	\end{pgfonlayer}
	\begin{pgfonlayer}{edgelayer}
		\draw [style=simple] (6) to (0);
		\draw [style=simple] (0) to (1);
		\draw [style=simple] (12) to (13);
		\draw [style=simple] (13) to (7);
		\draw [style=simple] (7) to (8);
		\draw [style=simple] (8) to (2);
		\draw [style=simple] (2) to (3);
		\draw [style=simple] (3) to (4);
		\draw [style=simple] (4) to (10);
		\draw [style=simple] (5) to (11);
		\draw [style=simple] (11) to (17);
		\draw [style=simple] (16) to (17);
		\draw [style=simple] (16) to (15);
		\draw [style=simple] (15) to (9);
		\draw [style=simple] (15) to (14);
		\draw [style=simple] (14) to (20);
		\draw [style=simple] (20) to (19);
		\draw [style=simple] (19) to (18);
		\draw [style=simple] (18) to (24);
		\draw [style=simple] (24) to (30);
		\draw [style=simple] (30) to (31);
		\draw [style=simple] (25) to (26);
		\draw [style=simple] (26) to (32);
		\draw [style=simple] (26) to (27);
		\draw [style=simple] (27) to (21);
		\draw [style=simple] (21) to (22);
		\draw [style=simple] (22) to (23);
		\draw [style=simple] (23) to (29);
		\draw [style=simple] (29) to (35);
		\draw [style=simple] (28) to (34);
		\draw [style=simple] (34) to (33);
		\draw [style=bpm] (6) to (12);
	\end{pgfonlayer}
\end{tikzpicture}
\caption{Player B opens the smallest chain available as before. \textbf{\textbf{Score}}: 0 - 0.}
\end{figure}


\begin{figure}[H]
\centering
\begin{tikzpicture}
	\begin{pgfonlayer}{nodelayer}
		\node [style=node] (0) at (-5, -0) {};
		\node [style=node] (1) at (-4, -0) {};
		\node [style=node] (2) at (-3, -0) {};
		\node [style=node] (3) at (-2, -0) {};
		\node [style=node] (4) at (-1, -0) {};
		\node [style=node] (5) at (0, -0) {};
		\node [style=node] (6) at (-5, -1) {};
		\node [style=node] (7) at (-4, -1) {};
		\node [style=node] (8) at (-3, -1) {};
		\node [style=node] (9) at (-2, -1) {};
		\node [style=node] (10) at (-1, -1) {};
		\node [style=node] (11) at (0, -1) {};
		\node [style=node] (12) at (-5, -2) {};
		\node [style=node] (13) at (-4, -2) {};
		\node [style=node] (14) at (-3, -2) {};
		\node [style=node] (15) at (-2, -2) {};
		\node [style=node] (16) at (-1, -2) {};
		\node [style=node] (17) at (0, -2) {};
		\node [style=node] (18) at (-5, -3) {};
		\node [style=node] (19) at (-4, -3) {};
		\node [style=node] (20) at (-3, -3) {};
		\node [style=node] (21) at (-2, -3) {};
		\node [style=node] (22) at (-1, -3) {};
		\node [style=node] (23) at (0, -3) {};
		\node [style=node] (24) at (-5, -4) {};
		\node [style=node] (25) at (-4, -4) {};
		\node [style=node] (26) at (-3, -4) {};
		\node [style=node] (27) at (-2, -4) {};
		\node [style=node] (28) at (-1, -4) {};
		\node [style=node] (29) at (0, -4) {};
		\node [style=node] (30) at (-5, -5) {};
		\node [style=node] (31) at (-4, -5) {};
		\node [style=node] (32) at (-3, -5) {};
		\node [style=node] (33) at (-2, -5) {};
		\node [style=node] (34) at (-1, -5) {};
		\node [style=node] (35) at (0, -5) {};
		\node (36) at (-4.5, -1.5) {\textcolor{blue}{\textit{A}}};
	\end{pgfonlayer}
	\begin{pgfonlayer}{edgelayer}
		\draw [style=simple] (6) to (0);
		\draw [style=simple] (0) to (1);
		\draw [style=simple] (12) to (13);
		\draw [style=simple] (13) to (7);
		\draw [style=simple] (7) to (8);
		\draw [style=simple] (8) to (2);
		\draw [style=simple] (2) to (3);
		\draw [style=simple] (3) to (4);
		\draw [style=simple] (4) to (10);
		\draw [style=simple] (5) to (11);
		\draw [style=simple] (11) to (17);
		\draw [style=simple] (16) to (17);
		\draw [style=simple] (16) to (15);
		\draw [style=simple] (15) to (9);
		\draw [style=simple] (15) to (14);
		\draw [style=simple] (14) to (20);
		\draw [style=simple] (20) to (19);
		\draw [style=simple] (19) to (18);
		\draw [style=simple] (18) to (24);
		\draw [style=simple] (24) to (30);
		\draw [style=simple] (30) to (31);
		\draw [style=simple] (25) to (26);
		\draw [style=simple] (26) to (32);
		\draw [style=simple] (26) to (27);
		\draw [style=simple] (27) to (21);
		\draw [style=simple] (21) to (22);
		\draw [style=simple] (22) to (23);
		\draw [style=simple] (23) to (29);
		\draw [style=simple] (29) to (35);
		\draw [style=simple] (28) to (34);
		\draw [style=simple] (34) to (33);
		\draw [style=dot] (6) to (12);
		\draw [style=apm] (6) to (7);
		\draw [style=apm] (1) to (2);
	\end{pgfonlayer}
\end{tikzpicture}
\caption{Now, Player A takes the first box and plays a double-dealing move instead. \textbf{\textbf{Score}}: 1 - 0.}
\end{figure}
\end{multicols}

\newpage
\begin{multicols}{2}
\begin{figure}[H]
\centering
\begin{tikzpicture}
	\begin{pgfonlayer}{nodelayer}
		\node [style=node] (0) at (-5, -0) {};
		\node [style=node] (1) at (-4, -0) {};
		\node [style=node] (2) at (-3, -0) {};
		\node [style=node] (3) at (-2, -0) {};
		\node [style=node] (4) at (-1, -0) {};
		\node [style=node] (5) at (0, -0) {};
		\node [style=node] (6) at (-5, -1) {};
		\node [style=node] (7) at (-4, -1) {};
		\node [style=node] (8) at (-3, -1) {};
		\node [style=node] (9) at (-2, -1) {};
		\node [style=node] (10) at (-1, -1) {};
		\node [style=node] (11) at (0, -1) {};
		\node [style=node] (12) at (-5, -2) {};
		\node [style=node] (13) at (-4, -2) {};
		\node [style=node] (14) at (-3, -2) {};
		\node [style=node] (15) at (-2, -2) {};
		\node [style=node] (16) at (-1, -2) {};
		\node [style=node] (17) at (0, -2) {};
		\node [style=node] (18) at (-5, -3) {};
		\node [style=node] (19) at (-4, -3) {};
		\node [style=node] (20) at (-3, -3) {};
		\node [style=node] (21) at (-2, -3) {};
		\node [style=node] (22) at (-1, -3) {};
		\node [style=node] (23) at (0, -3) {};
		\node [style=node] (24) at (-5, -4) {};
		\node [style=node] (25) at (-4, -4) {};
		\node [style=node] (26) at (-3, -4) {};
		\node [style=node] (27) at (-2, -4) {};
		\node [style=node] (28) at (-1, -4) {};
		\node [style=node] (29) at (0, -4) {};
		\node [style=node] (30) at (-5, -5) {};
		\node [style=node] (31) at (-4, -5) {};
		\node [style=node] (32) at (-3, -5) {};
		\node [style=node] (33) at (-2, -5) {};
		\node [style=node] (34) at (-1, -5) {};
		\node [style=node] (35) at (0, -5) {};
		\node (36) at (-4.5, -1.5) {\textcolor{blue}{\textit{A}}};
		\node (37) at (-4.5, -0.5) {\textcolor{red}{\textit{B}}};
		\node (38) at (-3.5, -0.5) {\textcolor{red}{\textit{B}}};
	\end{pgfonlayer}
	\begin{pgfonlayer}{edgelayer}
		\draw [style=simple] (6) to (0);
		\draw [style=simple] (0) to (1);
		\draw [style=simple] (12) to (13);
		\draw [style=simple] (13) to (7);
		\draw [style=simple] (7) to (8);
		\draw [style=simple] (8) to (2);
		\draw [style=simple] (2) to (3);
		\draw [style=simple] (3) to (4);
		\draw [style=simple] (4) to (10);
		\draw [style=simple] (5) to (11);
		\draw [style=simple] (11) to (17);
		\draw [style=simple] (16) to (17);
		\draw [style=simple] (16) to (15);
		\draw [style=simple] (15) to (9);
		\draw [style=simple] (15) to (14);
		\draw [style=simple] (14) to (20);
		\draw [style=simple] (20) to (19);
		\draw [style=simple] (19) to (18);
		\draw [style=simple] (18) to (24);
		\draw [style=simple] (24) to (30);
		\draw [style=simple] (30) to (31);
		\draw [style=simple] (25) to (26);
		\draw [style=simple] (26) to (32);
		\draw [style=simple] (26) to (27);
		\draw [style=simple] (27) to (21);
		\draw [style=simple] (21) to (22);
		\draw [style=simple] (22) to (23);
		\draw [style=simple] (23) to (29);
		\draw [style=simple] (29) to (35);
		\draw [style=simple] (28) to (34);
		\draw [style=simple] (34) to (33);
		\draw [style=dot] (6) to (12);
		\draw [style=dash] (6) to (7);
		\draw [style=dash] (1) to (2);
		\draw [style=bpm] (1) to (7);
		\draw [style=bpm] (34) to (35);
	\end{pgfonlayer}
\end{tikzpicture}
\caption{This forces Player B to take the two boxes and make another loony-move. \textbf{\textbf{Score}}: 1 - 2.}
\end{figure}



\begin{figure}[H]
\centering
\begin{tikzpicture}
	\begin{pgfonlayer}{nodelayer}
		\node [style=node] (0) at (-5, -0) {};
		\node [style=node] (1) at (-4, -0) {};
		\node [style=node] (2) at (-3, -0) {};
		\node [style=node] (3) at (-2, -0) {};
		\node [style=node] (4) at (-1, -0) {};
		\node [style=node] (5) at (0, -0) {};
		\node [style=node] (6) at (-5, -1) {};
		\node [style=node] (7) at (-4, -1) {};
		\node [style=node] (8) at (-3, -1) {};
		\node [style=node] (9) at (-2, -1) {};
		\node [style=node] (10) at (-1, -1) {};
		\node [style=node] (11) at (0, -1) {};
		\node [style=node] (12) at (-5, -2) {};
		\node [style=node] (13) at (-4, -2) {};
		\node [style=node] (14) at (-3, -2) {};
		\node [style=node] (15) at (-2, -2) {};
		\node [style=node] (16) at (-1, -2) {};
		\node [style=node] (17) at (0, -2) {};
		\node [style=node] (18) at (-5, -3) {};
		\node [style=node] (19) at (-4, -3) {};
		\node [style=node] (20) at (-3, -3) {};
		\node [style=node] (21) at (-2, -3) {};
		\node [style=node] (22) at (-1, -3) {};
		\node [style=node] (23) at (0, -3) {};
		\node [style=node] (24) at (-5, -4) {};
		\node [style=node] (25) at (-4, -4) {};
		\node [style=node] (26) at (-3, -4) {};
		\node [style=node] (27) at (-2, -4) {};
		\node [style=node] (28) at (-1, -4) {};
		\node [style=node] (29) at (0, -4) {};
		\node [style=node] (30) at (-5, -5) {};
		\node [style=node] (31) at (-4, -5) {};
		\node [style=node] (32) at (-3, -5) {};
		\node [style=node] (33) at (-2, -5) {};
		\node [style=node] (34) at (-1, -5) {};
		\node [style=node] (35) at (0, -5) {};
		\node (36) at (-4.5, -1.5) {\textcolor{blue}{\textit{A}}};
		\node (37) at (-4.5, -0.5) {\textcolor{red}{\textit{B}}};
		\node (38) at (-3.5, -0.5) {\textcolor{red}{\textit{B}}};
		\node (39) at (-0.5, -4.5) {\textcolor{blue}{\textit{A}}};
		\node (40) at (-0.5, -3.5) {\textcolor{blue}{\textit{A}}};
		\node (41) at (-1.5, -3.5) {\textcolor{blue}{\textit{A}}};
	\end{pgfonlayer}
	\begin{pgfonlayer}{edgelayer}
		\draw [style=simple] (6) to (0);
		\draw [style=simple] (0) to (1);
		\draw [style=simple] (12) to (13);
		\draw [style=simple] (13) to (7);
		\draw [style=simple] (7) to (8);
		\draw [style=simple] (8) to (2);
		\draw [style=simple] (2) to (3);
		\draw [style=simple] (3) to (4);
		\draw [style=simple] (4) to (10);
		\draw [style=simple] (5) to (11);
		\draw [style=simple] (11) to (17);
		\draw [style=simple] (16) to (17);
		\draw [style=simple] (16) to (15);
		\draw [style=simple] (15) to (9);
		\draw [style=simple] (15) to (14);
		\draw [style=simple] (14) to (20);
		\draw [style=simple] (20) to (19);
		\draw [style=simple] (19) to (18);
		\draw [style=simple] (18) to (24);
		\draw [style=simple] (24) to (30);
		\draw [style=simple] (30) to (31);
		\draw [style=simple] (25) to (26);
		\draw [style=simple] (26) to (32);
		\draw [style=simple] (26) to (27);
		\draw [style=simple] (27) to (21);
		\draw [style=simple] (21) to (22);
		\draw [style=simple] (22) to (23);
		\draw [style=simple] (23) to (29);
		\draw [style=simple] (29) to (35);
		\draw [style=simple] (28) to (34);
		\draw [style=simple] (34) to (33);
		\draw [style=dot] (6) to (12);
		\draw [style=dash] (6) to (7);
		\draw [style=dash] (1) to (2);
		\draw [style=dot] (1) to (7);
		\draw [style=dot] (34) to (35);
		\draw [style=apm] (28) to (29);
		\draw [style=apm] (22) to (28);
		\draw [style=apm] (27) to (28);
		\draw [style=apm] (32) to (33);
	\end{pgfonlayer}
\end{tikzpicture}
\caption{Player A takes three boxes and again uses a double-dealing move. \textbf{\textbf{Score}}: 4 - 2.}
\end{figure}
\end{multicols}


\begin{multicols}{2}
\begin{figure}[H]
\centering
\begin{tikzpicture}
	\begin{pgfonlayer}{nodelayer}
		\node [style=node] (0) at (-5, -0) {};
		\node [style=node] (1) at (-4, -0) {};
		\node [style=node] (2) at (-3, -0) {};
		\node [style=node] (3) at (-2, -0) {};
		\node [style=node] (4) at (-1, -0) {};
		\node [style=node] (5) at (0, -0) {};
		\node [style=node] (6) at (-5, -1) {};
		\node [style=node] (7) at (-4, -1) {};
		\node [style=node] (8) at (-3, -1) {};
		\node [style=node] (9) at (-2, -1) {};
		\node [style=node] (10) at (-1, -1) {};
		\node [style=node] (11) at (0, -1) {};
		\node [style=node] (12) at (-5, -2) {};
		\node [style=node] (13) at (-4, -2) {};
		\node [style=node] (14) at (-3, -2) {};
		\node [style=node] (15) at (-2, -2) {};
		\node [style=node] (16) at (-1, -2) {};
		\node [style=node] (17) at (0, -2) {};
		\node [style=node] (18) at (-5, -3) {};
		\node [style=node] (19) at (-4, -3) {};
		\node [style=node] (20) at (-3, -3) {};
		\node [style=node] (21) at (-2, -3) {};
		\node [style=node] (22) at (-1, -3) {};
		\node [style=node] (23) at (0, -3) {};
		\node [style=node] (24) at (-5, -4) {};
		\node [style=node] (25) at (-4, -4) {};
		\node [style=node] (26) at (-3, -4) {};
		\node [style=node] (27) at (-2, -4) {};
		\node [style=node] (28) at (-1, -4) {};
		\node [style=node] (29) at (0, -4) {};
		\node [style=node] (30) at (-5, -5) {};
		\node [style=node] (31) at (-4, -5) {};
		\node [style=node] (32) at (-3, -5) {};
		\node [style=node] (33) at (-2, -5) {};
		\node [style=node] (34) at (-1, -5) {};
		\node [style=node] (35) at (0, -5) {};
		\node (36) at (-4.5, -1.5) {\textcolor{blue}{\textit{A}}};
		\node (37) at (-4.5, -0.5) {\textcolor{red}{\textit{B}}};
		\node (38) at (-3.5, -0.5) {\textcolor{red}{\textit{B}}};
		\node (39) at (-0.5, -4.5) {\textcolor{blue}{\textit{A}}};
		\node (40) at (-0.5, -3.5) {\textcolor{blue}{\textit{A}}};
		\node (41) at (-1.5, -3.5) {\textcolor{blue}{\textit{A}}};
		\node (42) at (-2.5, -4.5) {\textcolor{red}{\textit{B}}};
		\node (43) at (-1.5, -4.5) {\textcolor{red}{\textit{B}}};
	\end{pgfonlayer}
	\begin{pgfonlayer}{edgelayer}
		\draw [style=simple] (6) to (0);
		\draw [style=simple] (0) to (1);
		\draw [style=simple] (12) to (13);
		\draw [style=simple] (13) to (7);
		\draw [style=simple] (7) to (8);
		\draw [style=simple] (8) to (2);
		\draw [style=simple] (2) to (3);
		\draw [style=simple] (3) to (4);
		\draw [style=simple] (4) to (10);
		\draw [style=simple] (5) to (11);
		\draw [style=simple] (11) to (17);
		\draw [style=simple] (16) to (17);
		\draw [style=simple] (16) to (15);
		\draw [style=simple] (15) to (9);
		\draw [style=simple] (15) to (14);
		\draw [style=simple] (14) to (20);
		\draw [style=simple] (20) to (19);
		\draw [style=simple] (19) to (18);
		\draw [style=simple] (18) to (24);
		\draw [style=simple] (24) to (30);
		\draw [style=simple] (30) to (31);
		\draw [style=simple] (25) to (26);
		\draw [style=simple] (26) to (32);
		\draw [style=simple] (26) to (27);
		\draw [style=simple] (27) to (21);
		\draw [style=simple] (21) to (22);
		\draw [style=simple] (22) to (23);
		\draw [style=simple] (23) to (29);
		\draw [style=simple] (29) to (35);
		\draw [style=simple] (28) to (34);
		\draw [style=simple] (34) to (33);
		\draw [style=dot] (6) to (12);
		\draw [style=dash] (6) to (7);
		\draw [style=dash] (1) to (2);
		\draw [style=dot] (1) to (7);
		\draw [style=dot] (34) to (35);
		\draw [style=dash] (28) to (29);
		\draw [style=dash] (22) to (28);
		\draw [style=dash] (27) to (28);
		\draw [style=dash] (32) to (33);
		\draw [style=bpm] (27) to (33);
		\draw [style=bpm] (17) to (23);
	\end{pgfonlayer}
\end{tikzpicture}
\caption{Player B takes the two boxes and opens the next smallest chain. \textbf{\textbf{Score}}: 4 - 4.}
\end{figure}

\begin{figure}[H]
\centering
\begin{tikzpicture}
	\begin{pgfonlayer}{nodelayer}
		\node [style=node] (0) at (-5, -0) {};
		\node [style=node] (1) at (-4, -0) {};
		\node [style=node] (2) at (-3, -0) {};
		\node [style=node] (3) at (-2, -0) {};
		\node [style=node] (4) at (-1, -0) {};
		\node [style=node] (5) at (0, -0) {};
		\node [style=node] (6) at (-5, -1) {};
		\node [style=node] (7) at (-4, -1) {};
		\node [style=node] (8) at (-3, -1) {};
		\node [style=node] (9) at (-2, -1) {};
		\node [style=node] (10) at (-1, -1) {};
		\node [style=node] (11) at (0, -1) {};
		\node [style=node] (12) at (-5, -2) {};
		\node [style=node] (13) at (-4, -2) {};
		\node [style=node] (14) at (-3, -2) {};
		\node [style=node] (15) at (-2, -2) {};
		\node [style=node] (16) at (-1, -2) {};
		\node [style=node] (17) at (0, -2) {};
		\node [style=node] (18) at (-5, -3) {};
		\node [style=node] (19) at (-4, -3) {};
		\node [style=node] (20) at (-3, -3) {};
		\node [style=node] (21) at (-2, -3) {};
		\node [style=node] (22) at (-1, -3) {};
		\node [style=node] (23) at (0, -3) {};
		\node [style=node] (24) at (-5, -4) {};
		\node [style=node] (25) at (-4, -4) {};
		\node [style=node] (26) at (-3, -4) {};
		\node [style=node] (27) at (-2, -4) {};
		\node [style=node] (28) at (-1, -4) {};
		\node [style=node] (29) at (0, -4) {};
		\node [style=node] (30) at (-5, -5) {};
		\node [style=node] (31) at (-4, -5) {};
		\node [style=node] (32) at (-3, -5) {};
		\node [style=node] (33) at (-2, -5) {};
		\node [style=node] (34) at (-1, -5) {};
		\node [style=node] (35) at (0, -5) {};
		\node (36) at (-4.5, -1.5) {\textcolor{blue}{\textit{A}}};
		\node (37) at (-4.5, -0.5) {\textcolor{red}{\textit{B}}};
		\node (38) at (-3.5, -0.5) {\textcolor{red}{\textit{B}}};
		\node (39) at (-0.5, -4.5) {\textcolor{blue}{\textit{A}}};
		\node (40) at (-0.5, -3.5) {\textcolor{blue}{\textit{A}}};
		\node (41) at (-1.5, -3.5) {\textcolor{blue}{\textit{A}}};
		\node (42) at (-2.5, -4.5) {\textcolor{red}{\textit{B}}};
		\node (43) at (-1.5, -4.5) {\textcolor{red}{\textit{B}}};
		\node (44) at (-0.5, -2.5) {\textcolor{blue}{\textit{A}}};
		\node (45) at (-1.5, -2.5) {\textcolor{blue}{\textit{A}}};
		\node (46) at (-2.5, -2.5) {\textcolor{blue}{\textit{A}}};
		\node (47) at (-2.5, -3.5) {\textcolor{blue}{\textit{A}}};
		\node (48) at (-3.5, -3.5) {\textcolor{blue}{\textit{A}}};
		\node (49) at (-4.5, -3.5) {\textcolor{blue}{\textit{A}}};
	\end{pgfonlayer}
	\begin{pgfonlayer}{edgelayer}
		\draw [style=simple] (6) to (0);
		\draw [style=simple] (0) to (1);
		\draw [style=simple] (12) to (13);
		\draw [style=simple] (13) to (7);
		\draw [style=simple] (7) to (8);
		\draw [style=simple] (8) to (2);
		\draw [style=simple] (2) to (3);
		\draw [style=simple] (3) to (4);
		\draw [style=simple] (4) to (10);
		\draw [style=simple] (5) to (11);
		\draw [style=simple] (11) to (17);
		\draw [style=simple] (16) to (17);
		\draw [style=simple] (16) to (15);
		\draw [style=simple] (15) to (9);
		\draw [style=simple] (15) to (14);
		\draw [style=simple] (14) to (20);
		\draw [style=simple] (20) to (19);
		\draw [style=simple] (19) to (18);
		\draw [style=simple] (18) to (24);
		\draw [style=simple] (24) to (30);
		\draw [style=simple] (30) to (31);
		\draw [style=simple] (25) to (26);
		\draw [style=simple] (26) to (32);
		\draw [style=simple] (26) to (27);
		\draw [style=simple] (27) to (21);
		\draw [style=simple] (21) to (22);
		\draw [style=simple] (22) to (23);
		\draw [style=simple] (23) to (29);
		\draw [style=simple] (29) to (35);
		\draw [style=simple] (28) to (34);
		\draw [style=simple] (34) to (33);
		\draw [style=dot] (6) to (12);
		\draw [style=dash] (6) to (7);
		\draw [style=dash] (1) to (2);
		\draw [style=dot] (1) to (7);
		\draw [style=dot] (34) to (35);
		\draw [style=dash] (28) to (29);
		\draw [style=dash] (22) to (28);
		\draw [style=dash] (27) to (28);
		\draw [style=dash] (32) to (33);
		\draw [style=dot] (27) to (33);
		\draw [style=dot] (17) to (23);
		\draw [style=apm] (16) to (22);
		\draw [style=apm] (15) to (21);
		\draw [style=apm] (20) to (21);
		\draw [style=apm] (20) to (26);
		\draw [style=apm] (19) to (25);
		\draw [style=apm] (24) to (25);
		\draw [style=apm] (31) to (32);
	\end{pgfonlayer}
\end{tikzpicture}
\caption{For the last time, Player A takes all but the last two boxes. \newline \textbf{\textbf{Score}}: 10 - 4.}
\end{figure}
\end{multicols}

\begin{multicols}{2}
\begin{figure}[H]
\centering
\begin{tikzpicture}
	\begin{pgfonlayer}{nodelayer}
		\node [style=node] (0) at (-5, -0) {};
		\node [style=node] (1) at (-4, -0) {};
		\node [style=node] (2) at (-3, -0) {};
		\node [style=node] (3) at (-2, -0) {};
		\node [style=node] (4) at (-1, -0) {};
		\node [style=node] (5) at (0, -0) {};
		\node [style=node] (6) at (-5, -1) {};
		\node [style=node] (7) at (-4, -1) {};
		\node [style=node] (8) at (-3, -1) {};
		\node [style=node] (9) at (-2, -1) {};
		\node [style=node] (10) at (-1, -1) {};
		\node [style=node] (11) at (0, -1) {};
		\node [style=node] (12) at (-5, -2) {};
		\node [style=node] (13) at (-4, -2) {};
		\node [style=node] (14) at (-3, -2) {};
		\node [style=node] (15) at (-2, -2) {};
		\node [style=node] (16) at (-1, -2) {};
		\node [style=node] (17) at (0, -2) {};
		\node [style=node] (18) at (-5, -3) {};
		\node [style=node] (19) at (-4, -3) {};
		\node [style=node] (20) at (-3, -3) {};
		\node [style=node] (21) at (-2, -3) {};
		\node [style=node] (22) at (-1, -3) {};
		\node [style=node] (23) at (0, -3) {};
		\node [style=node] (24) at (-5, -4) {};
		\node [style=node] (25) at (-4, -4) {};
		\node [style=node] (26) at (-3, -4) {};
		\node [style=node] (27) at (-2, -4) {};
		\node [style=node] (28) at (-1, -4) {};
		\node [style=node] (29) at (0, -4) {};
		\node [style=node] (30) at (-5, -5) {};
		\node [style=node] (31) at (-4, -5) {};
		\node [style=node] (32) at (-3, -5) {};
		\node [style=node] (33) at (-2, -5) {};
		\node [style=node] (34) at (-1, -5) {};
		\node [style=node] (35) at (0, -5) {};
		\node (36) at (-4.5, -1.5) {\textcolor{blue}{\textit{A}}};
		\node (37) at (-4.5, -0.5) {\textcolor{red}{\textit{B}}};
		\node (38) at (-3.5, -0.5) {\textcolor{red}{\textit{B}}};
		\node (39) at (-0.5, -4.5) {\textcolor{blue}{\textit{A}}};
		\node (40) at (-0.5, -3.5) {\textcolor{blue}{\textit{A}}};
		\node (41) at (-1.5, -3.5) {\textcolor{blue}{\textit{A}}};
		\node (42) at (-2.5, -4.5) {\textcolor{red}{\textit{B}}};
		\node (43) at (-1.5, -4.5) {\textcolor{red}{\textit{B}}};
		\node (44) at (-0.5, -2.5) {\textcolor{blue}{\textit{A}}};
		\node (45) at (-1.5, -2.5) {\textcolor{blue}{\textit{A}}};
		\node (46) at (-2.5, -2.5) {\textcolor{blue}{\textit{A}}};
		\node (47) at (-2.5, -3.5) {\textcolor{blue}{\textit{A}}};
		\node (48) at (-3.5, -3.5) {\textcolor{blue}{\textit{A}}};
		\node (49) at (-4.5, -3.5) {\textcolor{blue}{\textit{A}}};
		\node (50) at (-4.5, -4.5) {\textcolor{red}{\textit{B}}};
		\node (51) at (-3.5, -4.5) {\textcolor{red}{\textit{B}}};
	\end{pgfonlayer}
	\begin{pgfonlayer}{edgelayer}
		\draw [style=simple] (6) to (0);
		\draw [style=simple] (0) to (1);
		\draw [style=simple] (12) to (13);
		\draw [style=simple] (13) to (7);
		\draw [style=simple] (7) to (8);
		\draw [style=simple] (8) to (2);
		\draw [style=simple] (2) to (3);
		\draw [style=simple] (3) to (4);
		\draw [style=simple] (4) to (10);
		\draw [style=simple] (5) to (11);
		\draw [style=simple] (11) to (17);
		\draw [style=simple] (16) to (17);
		\draw [style=simple] (16) to (15);
		\draw [style=simple] (15) to (9);
		\draw [style=simple] (15) to (14);
		\draw [style=simple] (14) to (20);
		\draw [style=simple] (20) to (19);
		\draw [style=simple] (19) to (18);
		\draw [style=simple] (18) to (24);
		\draw [style=simple] (24) to (30);
		\draw [style=simple] (30) to (31);
		\draw [style=simple] (25) to (26);
		\draw [style=simple] (26) to (32);
		\draw [style=simple] (26) to (27);
		\draw [style=simple] (27) to (21);
		\draw [style=simple] (21) to (22);
		\draw [style=simple] (22) to (23);
		\draw [style=simple] (23) to (29);
		\draw [style=simple] (29) to (35);
		\draw [style=simple] (28) to (34);
		\draw [style=simple] (34) to (33);
		\draw [style=dot] (6) to (12);
		\draw [style=dash] (6) to (7);
		\draw [style=dash] (1) to (2);
		\draw [style=dot] (1) to (7);
		\draw [style=dot] (34) to (35);
		\draw [style=dash] (28) to (29);
		\draw [style=dash] (22) to (28);
		\draw [style=dash] (27) to (28);
		\draw [style=dash] (32) to (33);
		\draw [style=dot] (27) to (33);
		\draw [style=dot] (17) to (23);
		\draw [style=dash] (16) to (22);
		\draw [style=dash] (15) to (21);
		\draw [style=dash] (20) to (21);
		\draw [style=dash] (20) to (26);
		\draw [style=dash] (19) to (25);
		\draw [style=dash] (24) to (25);
		\draw [style=dash] (31) to (32);
		\draw [style=bpm] (25) to (31);
		\draw [style=bpm] (12) to (18);
	\end{pgfonlayer}
\end{tikzpicture}
\caption{Player B takes the two boxes and opens the last chain. \newline \textbf{\textbf{Score}}: 10 - 6.}
\end{figure}


\begin{figure}[H]
\centering
\begin{tikzpicture}
	\begin{pgfonlayer}{nodelayer}
		\node [style=node] (0) at (-5, -0) {};
		\node [style=node] (1) at (-4, -0) {};
		\node [style=node] (2) at (-3, -0) {};
		\node [style=node] (3) at (-2, -0) {};
		\node [style=node] (4) at (-1, -0) {};
		\node [style=node] (5) at (0, -0) {};
		\node [style=node] (6) at (-5, -1) {};
		\node [style=node] (7) at (-4, -1) {};
		\node [style=node] (8) at (-3, -1) {};
		\node [style=node] (9) at (-2, -1) {};
		\node [style=node] (10) at (-1, -1) {};
		\node [style=node] (11) at (0, -1) {};
		\node [style=node] (12) at (-5, -2) {};
		\node [style=node] (13) at (-4, -2) {};
		\node [style=node] (14) at (-3, -2) {};
		\node [style=node] (15) at (-2, -2) {};
		\node [style=node] (16) at (-1, -2) {};
		\node [style=node] (17) at (0, -2) {};
		\node [style=node] (18) at (-5, -3) {};
		\node [style=node] (19) at (-4, -3) {};
		\node [style=node] (20) at (-3, -3) {};
		\node [style=node] (21) at (-2, -3) {};
		\node [style=node] (22) at (-1, -3) {};
		\node [style=node] (23) at (0, -3) {};
		\node [style=node] (24) at (-5, -4) {};
		\node [style=node] (25) at (-4, -4) {};
		\node [style=node] (26) at (-3, -4) {};
		\node [style=node] (27) at (-2, -4) {};
		\node [style=node] (28) at (-1, -4) {};
		\node [style=node] (29) at (0, -4) {};
		\node [style=node] (30) at (-5, -5) {};
		\node [style=node] (31) at (-4, -5) {};
		\node [style=node] (32) at (-3, -5) {};
		\node [style=node] (33) at (-2, -5) {};
		\node [style=node] (34) at (-1, -5) {};
		\node [style=node] (35) at (0, -5) {};
		\node (36) at (-4.5, -1.5) {\textcolor{blue}{\textit{A}}};
		\node (37) at (-4.5, -0.5) {\textcolor{red}{\textit{B}}};
		\node (38) at (-3.5, -0.5) {\textcolor{red}{\textit{B}}};
		\node (39) at (-0.5, -4.5) {\textcolor{blue}{\textit{A}}};
		\node (40) at (-0.5, -3.5) {\textcolor{blue}{\textit{A}}};
		\node (41) at (-1.5, -3.5) {\textcolor{blue}{\textit{A}}};
		\node (42) at (-2.5, -4.5) {\textcolor{red}{\textit{B}}};
		\node (43) at (-1.5, -4.5) {\textcolor{red}{\textit{B}}};
		\node (44) at (-0.5, -2.5) {\textcolor{blue}{\textit{A}}};
		\node (45) at (-1.5, -2.5) {\textcolor{blue}{\textit{A}}};
		\node (46) at (-2.5, -2.5) {\textcolor{blue}{\textit{A}}};
		\node (47) at (-2.5, -3.5) {\textcolor{blue}{\textit{A}}};
		\node (48) at (-3.5, -3.5) {\textcolor{blue}{\textit{A}}};
		\node (49) at (-4.5, -3.5) {\textcolor{blue}{\textit{A}}};
		\node (50) at (-4.5, -4.5) {\textcolor{red}{\textit{B}}};
		\node (51) at (-3.5, -4.5) {\textcolor{red}{\textit{B}}};
		\node (52) at (-4.5, -2.5) {\textcolor{blue}{\textit{A}}};
		\node (53) at (-3.5, -2.5) {\textcolor{blue}{\textit{A}}};
		\node (54) at (-3.5, -1.5) {\textcolor{blue}{\textit{A}}};
		\node (55) at (-2.5, -1.5) {\textcolor{blue}{\textit{A}}};
		\node (56) at (-2.5, -0.5) {\textcolor{blue}{\textit{A}}};
		\node (57) at (-1.5, -0.5) {\textcolor{blue}{\textit{A}}};
		\node (58) at (-1.5, -1.5) {\textcolor{blue}{\textit{A}}};
		\node (59) at (-0.5, -1.5) {\textcolor{blue}{\textit{A}}};
		\node (60) at (-0.5, -0.5) {\textcolor{blue}{\textit{A}}};
	\end{pgfonlayer}
	\begin{pgfonlayer}{edgelayer}
		\draw [style=simple] (6) to (0);
		\draw [style=simple] (0) to (1);
		\draw [style=simple] (12) to (13);
		\draw [style=simple] (13) to (7);
		\draw [style=simple] (7) to (8);
		\draw [style=simple] (8) to (2);
		\draw [style=simple] (2) to (3);
		\draw [style=simple] (3) to (4);
		\draw [style=simple] (4) to (10);
		\draw [style=simple] (5) to (11);
		\draw [style=simple] (11) to (17);
		\draw [style=simple] (16) to (17);
		\draw [style=simple] (16) to (15);
		\draw [style=simple] (15) to (9);
		\draw [style=simple] (15) to (14);
		\draw [style=simple] (14) to (20);
		\draw [style=simple] (20) to (19);
		\draw [style=simple] (19) to (18);
		\draw [style=simple] (18) to (24);
		\draw [style=simple] (24) to (30);
		\draw [style=simple] (30) to (31);
		\draw [style=simple] (25) to (26);
		\draw [style=simple] (26) to (32);
		\draw [style=simple] (26) to (27);
		\draw [style=simple] (27) to (21);
		\draw [style=simple] (21) to (22);
		\draw [style=simple] (22) to (23);
		\draw [style=simple] (23) to (29);
		\draw [style=simple] (29) to (35);
		\draw [style=simple] (28) to (34);
		\draw [style=simple] (34) to (33);
		\draw [style=dot] (6) to (12);
		\draw [style=dash] (6) to (7);
		\draw [style=dash] (1) to (2);
		\draw [style=dot] (1) to (7);
		\draw [style=dot] (34) to (35);
		\draw [style=dash] (28) to (29);
		\draw [style=dash] (22) to (28);
		\draw [style=dash] (27) to (28);
		\draw [style=dash] (32) to (33);
		\draw [style=dot] (27) to (33);
		\draw [style=dot] (17) to (23);
		\draw [style=dash] (16) to (22);
		\draw [style=dash] (15) to (21);
		\draw [style=dash] (20) to (21);
		\draw [style=dash] (20) to (26);
		\draw [style=dash] (19) to (25);
		\draw [style=dash] (24) to (25);
		\draw [style=dash] (31) to (32);
		\draw [style=dot] (25) to (31);
		\draw [style=dot] (12) to (18);
		\draw [style=apm] (13) to (19);
		\draw [style=apm] (13) to (14);
		\draw [style=apm] (14) to (8);
		\draw [style=apm] (8) to (9);
		\draw [style=apm] (9) to (3);
		\draw [style=apm] (9) to (10);
		\draw [style=apm] (10) to (16);
		\draw [style=apm] (10) to (11);
		\draw [style=apm] (4) to (5);
	\end{pgfonlayer}
\end{tikzpicture}
\caption{Given this is the last chain Player A does not need to keep control and thus takes all nine boxes remaining. \textbf{\textbf{Score}}: 19 - 6.}
\end{figure}
\end{multicols}

Having control is a very powerful position in the game of Dots and Boxes in light of the double-dealing move and consequential Loony Theorem.  As in the above game, it allows Player A to completely destroy his opponent.  When there are sufficiently many large long-chains or cycles about it is easy to see that keeping control is the best course of action as you will clearly capture more boxes than the number that must be sacrificed.  However it becomes more difficult to decide when to keep control in a game with relatively small long-chains and cycles left.  In the next section we will explore some examples of when to lose control.

\subsection{Exceptions to Keeping Control}
It is not hard to imagine a game in which keeping control until the very end results in a loss.  Consider a 6x6 board in which twelve 3-chains have formed.  Then, if the player who has control (assume it is Player A) keeps it for the entire game he will lose 14-22, due to the fact that he must sacrifice two boxes for each chain but the last.  This implies that at some point Player A should have decided to capture all the boxes in an opened chain, and transfer control to Player B.  But since Player A always has the option of doing so, and we know how many boxes will be captured in either case, we can infer the following rule:
\\\\
\emph{Assume Player A has control.  Let TB be the total number of boxes on the board. Let LC be the total number of long-chains remaining. Let C be the total number of cycles remaining.  And let B be Player B's score prior to Player A having control.  Then Player A should not keep control until the end if:}

\begin{equation}
\emph{TB} - (2 \times (\emph{LC} - 1) + 4\emph{C} + \emph{B}) < \frac{\emph{TB}}{2}
\end{equation}

\noindent
\emph{when there are long-chains on the board. And if:}

\begin{equation}
\emph{TB} - (4 \times (\emph{C} - 1) + \emph{B}) < \frac{\emph{TB}}{2}
\end{equation}

\noindent
\emph{when only cycles remain.}
\\\\
\noindent
\emph{Also, if you ever find that the right sides of the above equations are greater than the left sides, you should keep control until the end as you can force a win.  If it is Player B who has control we must replace B with A, where A is Player A's score prior to Player B having control, and then follow the same rule.}\\\\
While this is clearly true, It does not tell us what to do in every situation.  It should also be noted that when the left side of the above equations equals the right side we know that a tie can be forced, but not that a win does not exist, a more exhaustive analysis is needed to determine if the tie is the best case or not.

In general it is difficult to pin down exactly when to lose control and when to keep it, but this rule does work as a heuristic when playing the game.  In order to solve the general question one would have to determine all the possible permutations and choose the best path. 




\chapter{Intermediate Strategy}
It is almost always a benefit to have control in a game of Dots and Boxes and, in most games, secures a win.  As we saw, this is due to the fact that gaining control means your opponent has just made a loony-move, which allows you to choose which side you prefer for the remainder of the game.  And even though there are game positions in which you should give up control, they are rare and can be analyzed on a case by case basis.  Thus, Dots and Boxes strategy really centers around making sure you get control and that there are long enough chains around to keep it.

In this chapter we will be considering a simplified version of the game -- a Normal Game -- in order to derive concrete strategies for accomplishing the goal of obtaining control.  Later we will develop general methods for acquiring control, but since a majority of games played are Normal Games, it is useful to assume this fact and then adapt when exceptions occur. 


\section{Normal Play}
Normal Play was a direct consequence of the Loony Theorem and is the result of playing what is called a Normal Game.

\begin{mydef}[Normal Game]
A \textbf{Normal Game} is one in which neither player makes a loony-move until all that remains are long-chains or cycles.  Thus there are no loony-moves until one player gains control.  Also, whoever has control in a Normal Game keeps it until the end of the game.
\end{mydef}

Once we assume that a game of Dots and Boxes is a Normal Game we are able to develop to some very insightful theorems, and in particular, proceed to one of the most useful observations in the game.  It is known as the Chain Rule.  But first let us consider two immediate consequences of playing a Normal Game.

\begin{lemma}
If there are an odd number of turns in a Normal Game, then Player B opened the first long-chain giving Player A control. The opposite is true if there are an even number of turns.
\end{lemma}

\begin{proof}
In a Normal Game the player who moves on the last turn of the game is also the player who had control.  This immediately follows from the definition.  Thus, given Player A always plays on the odd turns and Player B always plays on the even turns we can conclude \emph{Lemma 1}.
\end{proof}

\begin{lemma}
If there are long-chains in a Normal Game, then DC = (LC - 1) + 2C.  Where DC is the total number of double-cross moves, LC is the total number of long-chains, and C is the total number of cycles.  In the case when there are no long-chains we get: DC = 2 $\times$ (C - 1).
\end{lemma}

\begin{proof}
Assume there are long-chains in a Normal Game and Player A will get control.  Then we know no loony-move has been made until all that remains are long-chains and cycles (if there are any).  At this point Player B will first move into cycles until they are gone and then move into long-chains -- given Player A must sacrifice four boxes for each cycle and only two for each long-chain.  Therefore there will be two double-cross moves for each cycle and one for each long-chain but the last one, in which Player A takes all the boxes.  This gives us \emph{DC} = (\emph{LC} - 1) + 2\emph{C} for when there are long-chains about.  The case when there are no long-chains can be proven in a similar fashion.
\end{proof}

\section{The Chain Rule}
From Lemma 1 we know that if Player A wants to force his opponent to give him control there must be an odd number of turns in the game.  It turns out that in order to do this Player A should try to make the number of long-chains odd if there are an odd number of dots, and even if there are an even number of dots.  And this is exactly the Chain Rule.

\begin{theorem}
\emph{ (The Chain Rule) }
In order to have control in a Normal Game, Player A should make the number of long-chains odd if there are an odd number of dots and even if there are an even number of dots. The opposite is true for Player B.
\end{theorem}

This may not be obvious at first, and that is because we haven't thought about the relationship between the number of turns and the number of double-cross moves, which ultimately tells us about the relationship between the number of turns and the number of long-chains.  Thus in order to prove the Chain Rule we must first prove the following theorem:

\begin{theorem}
The total number of turns is equal to the total number of dots plus the total number of double-cross moves.\footnote[1]{The proof of this theorem does not assume a square board, but clearly holds when the length of both sides are equal.}
\end{theorem}

\begin{proof}
Consider an $m$ x $n$ board where \emph{m} and \emph{n} are the number of boxes in each row or column respectively; this implies that there are $(m + 1) (n + 1)$ dots. Immediately we know the following:
\begin{equation}
(m + 1) n = \textit{total number of horizontal moves,}
\end{equation}
\begin{equation}
(n + 1) m = \textit{total number of vertical moves.}
\end{equation}
Thus before introducing the rule of having to move again once a box is completed we know:
\begin{equation}
2mn + m + n = \textit{total number of moves possible.}
\end{equation}
Now, since after any box is completed you must move again on the same turn, except when the last box is taken in which the game ends, we must subtract $(mn - 1)$ moves from our previous total moves giving us:
\begin{equation}
(m + 1) (n + 1) = \textit{total number of turns}\footnote[2]{Up until this point in the proof moves and turns were equal.  However now that we are accounting for more than one move per turn, we are able to talk about the total number of turns rather than moves.} \textit{ before double-cross moves.}
\end{equation}
In the last step we subtracted one move for each box captured, except the last box, in order to determine the number of turns in a game.  But this did not account for moves which capture two boxes with one edge -- namely the double-cross move.  Thus when counting the total number of turns we must add one back for each double-cross move that is played.  And this results in what we want:
\begin{equation}
\textit{total number of turns} = (m + 1) (n + 1) + \textit{double-cross moves.}
\end{equation}
\begin{equation}
\textit{total number of turns} = \textit{dots} + \textit{double-cross moves.}
\end{equation}
\end{proof}

Now that we know the relationship between double-cross moves and turns we can move on to the proof of the Chain Rule.

\begin{proof}[Proof of The Chain Rule]
Assume there are an odd number of dots.  From Lemma 1 we know that in order for Player A to have control there must be an odd number of turns.  Thus, for there to be an odd number of turns the total number of double-cross moves must be even; given $odd + even = odd$ and $odd + odd = even$.  Therefore all Player A must do is force  (\emph{LC} - 1) + 2\emph{C} to be even according to Lemma 2.  In order for this to be even the number of long-chains must be odd ($even + even = even$), giving us the final step.  Therefore, an odd number of dots implies that Player A must make the number of long-chains odd and an even number of dots implies he should make an even number of long-chains.  The other side of the theorem is proved in exactly the same manner.
\end{proof}

The above proof relied on Lemma 2 which takes into account whether there are cycles or not.  But if you notice we did not have to worry about cycles in the proof of the Chain Rule.  This is due to the fact that no matter how many cycles there are on a board there will always be an even number of double-cross moves played for each cycle.  Therefore the number of cycles does not affect the parity of double-cross moves, and thus has no affect on the Chain Rule.

\subsection{Examples of the Chain Rule}
Given control is so key in the game of Dots and Boxes we now see why the Chain Rule is so useful.  In practice, it allows you to have a goal from the outset which will guarantee a favorable outcome, and this is extremely helpful.  Let us now look at a few examples to see how the Chain Rule is used in a Normal Game.

All the following games are played on 5x5 boards with 25 dots.  Therefore, Player A must make the number of long-chains odd in order to have control and Player B must make the number even:

\begin{multicols}{2}
\begin{figure}[H]
\centering
\begin{tikzpicture}
	\begin{pgfonlayer}{nodelayer}
		\node [style=node] (0) at (-2, 2) {};
		\node [style=node] (1) at (-1, 2) {};
		\node [style=node] (2) at (0, 2) {};
		\node [style=node] (3) at (1, 2) {};
		\node [style=node] (4) at (2, 2) {};
		\node [style=node] (5) at (-2, 1) {};
		\node [style=node] (6) at (-1, 1) {};
		\node [style=node] (7) at (0, 1) {};
		\node [style=node] (8) at (1, 1) {};
		\node [style=node] (9) at (2, 1) {};
		\node [style=node] (10) at (-2, -0) {};
		\node [style=node] (11) at (-1, -0) {};
		\node [style=node] (12) at (0, -0) {};
		\node [style=node] (13) at (1, -0) {};
		\node [style=node] (14) at (2, -0) {};
		\node [style=node] (15) at (-2, -1) {};
		\node [style=node] (16) at (-1, -1) {};
		\node [style=node] (17) at (0, -1) {};
		\node [style=node] (18) at (1, -1) {};
		\node [style=node] (19) at (2, -1) {};
		\node [style=node] (20) at (-2, -2) {};
		\node [style=node] (21) at (-1, -2) {};
		\node [style=node] (22) at (0, -2) {};
		\node [style=node] (23) at (2, -2) {};
		\node [style=node] (24) at (1, -2) {};
		\node [style=node] (25) at (1, -2) {};
	\end{pgfonlayer}
	\begin{pgfonlayer}{edgelayer}
		\draw [style=dash] (1) to (6);
		\draw [style=dash] (10) to (11);
		\draw [style=dash] (11) to (16);
		\draw [style=dash] (16) to (17);
		\draw [style=dash] (17) to (18);
		\draw [style=dash] (18) to (19);
		\draw [style=dash] (13) to (12);
		\draw [style=dot] (11) to (6);
		\draw [style=dot] (6) to (9);
		\draw [style=dot] (3) to (8);
		\draw [style=dot] (13) to (14);
		\draw [style=dot] (16) to (21);
		\draw [style=apm] (18) to (24);
	\end{pgfonlayer}
\end{tikzpicture}
\caption{One long-chain.  Player B to move.}
\end{figure}


\begin{figure}[H]
\centering
\begin{tikzpicture}
	\begin{pgfonlayer}{nodelayer}
		\node [style=node] (0) at (-2, 2) {};
		\node [style=node] (1) at (-1, 2) {};
		\node [style=node] (2) at (0, 2) {};
		\node [style=node] (3) at (1, 2) {};
		\node [style=node] (4) at (2, 2) {};
		\node [style=node] (5) at (-2, 1) {};
		\node [style=node] (6) at (-1, 1) {};
		\node [style=node] (7) at (0, 1) {};
		\node [style=node] (8) at (1, 1) {};
		\node [style=node] (9) at (2, 1) {};
		\node [style=node] (10) at (-2, -0) {};
		\node [style=node] (11) at (-1, -0) {};
		\node [style=node] (12) at (0, -0) {};
		\node [style=node] (13) at (1, -0) {};
		\node [style=node] (14) at (2, -0) {};
		\node [style=node] (15) at (-2, -1) {};
		\node [style=node] (16) at (-1, -1) {};
		\node [style=node] (17) at (0, -1) {};
		\node [style=node] (18) at (1, -1) {};
		\node [style=node] (19) at (2, -1) {};
		\node [style=node] (20) at (-2, -2) {};
		\node [style=node] (21) at (-1, -2) {};
		\node [style=node] (22) at (0, -2) {};
		\node [style=node] (23) at (1, -2) {};
		\node [style=node] (24) at (2, -2) {};
	\end{pgfonlayer}
	\begin{pgfonlayer}{edgelayer}
		\draw [style=dash] (4) to (1);
		\draw [style=dash] (1) to (6);
		\draw [style=dash] (7) to (8);
		\draw [style=dash] (12) to (13);
		\draw [style=dash] (10) to (11);
		\draw [style=dash] (15) to (17);
		\draw [style=dot] (6) to (11);
		\draw [style=dot] (11) to (12);
		\draw [style=dot] (8) to (9);
		\draw [style=dot] (13) to (14);
		\draw [style=dot] (13) to (23);
		\draw [style=dot] (21) to (23);
		\draw [style=bpm] (20) to (21);
	\end{pgfonlayer}
\end{tikzpicture}
\caption{Two long-chains.  Player A to move.}
\end{figure}
\end{multicols}

\begin{multicols}{2}
\begin{figure}[H]
\centering
\begin{tikzpicture}
	\begin{pgfonlayer}{nodelayer}
		\node [style=node] (0) at (-2, 2) {};
		\node [style=node] (1) at (-1, 2) {};
		\node [style=node] (2) at (0, 2) {};
		\node [style=node] (3) at (1, 2) {};
		\node [style=node] (4) at (2, 2) {};
		\node [style=node] (5) at (-2, 1) {};
		\node [style=node] (6) at (-1, 1) {};
		\node [style=node] (7) at (0, 1) {};
		\node [style=node] (8) at (1, 1) {};
		\node [style=node] (9) at (2, 1) {};
		\node [style=node] (10) at (-2, -0) {};
		\node [style=node] (11) at (-1, -0) {};
		\node [style=node] (12) at (0, -0) {};
		\node [style=node] (13) at (1, -0) {};
		\node [style=node] (14) at (2, -0) {};
		\node [style=node] (15) at (-2, -1) {};
		\node [style=node] (16) at (-1, -1) {};
		\node [style=node] (17) at (0, -1) {};
		\node [style=node] (18) at (1, -1) {};
		\node [style=node] (19) at (2, -1) {};
		\node [style=node] (20) at (-2, -2) {};
		\node [style=node] (21) at (-1, -2) {};
		\node [style=node] (22) at (0, -2) {};
		\node [style=node] (23) at (1, -2) {};
		\node [style=node] (24) at (2, -2) {};
	\end{pgfonlayer}
	\begin{pgfonlayer}{edgelayer}
		\draw [style=dot] (3) to (4);
		\draw [style=dot] (8) to (9);
		\draw [style=dot] (10) to (14);
		\draw [style=dot] (17) to (19);
		\draw [style=dot] (18) to (23);
		\draw [style=apm] (0) to (5);
		\draw [style=dash] (5) to (10);
		\draw [style=dash] (1) to (6);
		\draw [style=dash] (2) to (12);
		\draw [style=dash] (2) to (3);
		\draw [style=dash] (15) to (16);
		\draw [style=dash] (16) to (21);
		\draw [style=dash] (16) to (17);
	\end{pgfonlayer}
\end{tikzpicture}
\caption{Three long-chains.  Player B to move.}
\end{figure}


\begin{figure}[H]
\centering
\begin{tikzpicture}
	\begin{pgfonlayer}{nodelayer}
		\node [style=node] (0) at (-2, 2) {};
		\node [style=node] (1) at (-1, 2) {};
		\node [style=node] (2) at (0, 2) {};
		\node [style=node] (3) at (1, 2) {};
		\node [style=node] (4) at (2, 2) {};
		\node [style=node] (5) at (-2, 1) {};
		\node [style=node] (6) at (-1, 1) {};
		\node [style=node] (7) at (0, 1) {};
		\node [style=node] (8) at (1, 1) {};
		\node [style=node] (9) at (2, 1) {};
		\node [style=node] (10) at (-2, -0) {};
		\node [style=node] (11) at (-1, -0) {};
		\node [style=node] (12) at (0, -0) {};
		\node [style=node] (13) at (1, -0) {};
		\node [style=node] (14) at (2, -0) {};
		\node [style=node] (15) at (-2, -1) {};
		\node [style=node] (16) at (-1, -1) {};
		\node [style=node] (17) at (0, -1) {};
		\node [style=node] (18) at (1, -1) {};
		\node [style=node] (19) at (2, -1) {};
		\node [style=node] (20) at (-2, -2) {};
		\node [style=node] (21) at (-1, -2) {};
		\node [style=node] (22) at (0, -2) {};
		\node [style=node] (23) at (1, -2) {};
		\node [style=node] (24) at (2, -2) {};
	\end{pgfonlayer}
	\begin{pgfonlayer}{edgelayer}
		\draw [style=bpm] (4) to (9);
		\draw [style=dot] (2) to (12);
		\draw [style=dot] (12) to (14);
		\draw [style=dot] (3) to (8);
		\draw [style=dot] (18) to (19);
		\draw [style=dot] (22) to (24);
		\draw [style=dash] (15) to (16);
		\draw [style=dash] (20) to (22);
		\draw [style=dash] (22) to (12);
		\draw [style=dash] (12) to (10);
		\draw [style=dash] (10) to (0);
		\draw [style=dash] (1) to (6);
		\draw [style=dot] (9) to (14);
	\end{pgfonlayer}
\end{tikzpicture}
\caption{Four long-chains.  Player A to move.}
\end{figure}
\end{multicols}
\noindent
We see that in every one of these games no matter what moves are played, the player who will have control according the the Chain Rule is guaranteed a way of forcing their opponent to play the first loony-move.  And in all cases win by keeping control.  Once you have practiced using the Chain Rule you will see that it is an effective strategy against any opponent.

\section{Phases of the Game}
With the introduction of the Chain Rule, the game of Dots and Boxes transforms from being a fight for points, into a fight for the right number of long-chains (known as the \textbf{chain fight}).  As you begin to use the Chain Rule you will start to notice that each Normal Game can be divided into three main phases of play, each with different tactics for ultimately winning the chain fight.

\subsubsection{Openings}
Initially there are the opening moves.  These typically consist of moves that have been studied and memorized; they don't contribute directly to the chain count, but instead begin partitioning the board into desirable sections.  

For example, consider a 5x5 board.  Then, we know that Player A needs an odd number of long-chains.  So basic opening strategy tells us that Player A should either begin making one long-chain that runs through the center or attempt to split the board into three sections each with one chain.  And Player B tries to split the board in half creating two sections large enough for one long-chain in each or two long-chains in each.  

These rules can be extrapolated to different size boards, however in order to become an expert Dots and Boxes player one must rely on standard moves that have been exhaustively analyzed.

\subsubsection{Middle-Game}
The middle-game starts after players are done making standard opening moves and begin forcing the creation of chains in sections of the board.  Here, the players will fight for the correct chain count according to their turn parity.  Once the number of long-chains has been determined the middle-game is all about extending or limiting those chains depending on the outcome of the chain fight.  Therefore the player who has won the chain fight will attempt to make the long-chains as large as possible, while the player who has lost the chain fight will try to keep them as small as possible; this is due to the fact that control is most effective when there are large long-chains about.

\subsubsection{End-Game}
Once the number and length of all the long-chains has been established a game has reached the end-game.  The end-game is divided into three parts.  First there is the neutral phase, where players trade turns filling in edges without taking any boxes.  This is followed by the short-chain phase (often referred to as a saturated state).  In this part of the end-game players alternate capturing short-chains.  Lastly there is the final phase in which only long-chains or cycles remain.  The final phase is when a player will have control and keep it until the game is over.

\section{Beyond Normal Play}

\subsection{The Chain Rule in a Non-Normal Game}
In order for a game to be considered a Normal Game, there are two requirements; (1) there are no loony-moves until the final phase of the game, and (2) once a player has control he keeps it for the remainder of the game.  In most cases when a loony-move is played before the final phase of a game the player who gets control will still keep it until the last turn.  Thus the following theorem gives a rule for what to do in this situation:

\begin{theorem}
If there are an odd number of double-cross moves before the final phase of the game, then Player B must instead make an odd number of long-chains when playing on a board with an odd number of dots, and an even number long-chains when there are an even number of dots.  Again Player A must do the opposite.  And if there are an even number of double-cross moves before the final phase of the game then there is no affect on the Chain Rule.
\end{theorem}

\begin{proof}
Assume there are an odd number of dots.  Then, similar to the Chain Rule proof we must determine how many long-chains are needed based on the parity of double-cross moves.  Modifying Lemma 2 we get (\emph{LC} - 1) + 2\emph{C} + \emph{EDC}.  Where \emph{LC} and \emph{C} still stand for the number of long-chains and cycles respectively, but now we must count every early double-cross move as well -- \emph{EDC}.  As before, in order for there to be an odd number of turns, giving Player A control, there must be an even number of double-cross moves.  If there are an even number of early double-cross move then the parity of our new double-cross expression is the same as the parity of the old double-cross expression, and thus Player A must make an odd number of long-chains as before.  But if there are an odd number of early double-cross moves then Player A must now create an even number of long-chains in order for the number of double-cross moves to be even.  If there are an even number of dots the opposite is true.  The conditions for Player B are proved in the same fashion.
\end{proof}


\subsection{The Preemptive  Sacrifice}
As one studies high level Dots and Boxes play it becomes clear that even when a player has lost the chain fight it is possible to win or tie the game.  One such method for doing this is known as the Preemptive Sacrifice.  As the name implies, one must make an early loony-move when employing this strategy and therefore must already be up in boxes prior to the sacrifice in order to be effective\footnote[3]{If a player is not already up at least one box, then a preemptive sacrifice results in a loss.  This is Theorem 8, and is proven later in this section.}.  The conditions for when a player can use a preemptive sacrifice successfully are hard-fast and don't appear in every game.  However when the opportunity arises it is a very useful move.  The exact conditions will be fully explored in this section, but for now let's get a feel for how it works.  Consider the following game:

\begin{figure}[H]
\centering
\begin{tikzpicture}
	\begin{pgfonlayer}{nodelayer}
		\node [style=node] (0) at (-4, -0) {};
		\node [style=node] (1) at (-3, -0) {};
		\node [style=node] (2) at (-2, -0) {};
		\node [style=node] (3) at (-1, -0) {};
		\node [style=node] (4) at (0, -0) {};
		\node [style=node] (5) at (-4, -1) {};
		\node [style=node] (6) at (-3, -1) {};
		\node [style=node] (7) at (-2, -1) {};
		\node [style=node] (8) at (-1, -1) {};
		\node [style=node] (9) at (0, -1) {};
		\node [style=node] (10) at (-4, -2) {};
		\node [style=node] (11) at (-3, -2) {};
		\node [style=node] (12) at (-2, -2) {};
		\node [style=node] (13) at (-1, -2) {};
		\node [style=node] (14) at (0, -2) {};
		\node [style=node] (15) at (-4, -3) {};
		\node [style=node] (16) at (-3, -3) {};
		\node [style=node] (17) at (-2, -3) {};
		\node [style=node] (18) at (-1, -3) {};
		\node [style=node] (19) at (0, -3) {};
		\node [style=node] (20) at (-4, -4) {};
		\node [style=node] (21) at (-3, -4) {};
		\node [style=node] (22) at (-2, -4) {};
		\node [style=node] (23) at (-1, -4) {};
		\node [style=node] (24) at (0, -4) {};
		\node (25) at (-1.5, -1.5) {\textcolor{red}{\textit{B}}};
		\node (26) at (-1.5, -2.5) {\textcolor{red}{\textit{B}}};
	\end{pgfonlayer}
	\begin{pgfonlayer}{edgelayer}
		\draw [style=dash] (1) to (6);
		\draw [style=dash] (6) to (7);
		\draw [style=dash] (4) to (9);
		\draw [style=dash] (8) to (13);
		\draw [style=dash] (18) to (19);
		\draw [style=dash] (16) to (17);
		\draw [style=dash] (16) to (21);
		\draw [style=dash] (10) to (15);
		\draw [style=dot] (5) to (6);
		\draw [style=dot] (10) to (11);
		\draw [style=bpm] (7) to (8);
		\draw [style=dot] (12) to (13);
		\draw [style=dot] (7) to (12);
		\draw [style=dot] (12) to (17);
		\draw [style=dot] (17) to (18);
		\draw [style=dot] (13) to (18);
		\draw [style=dot] (3) to (4);
		\draw [style=dot] (9) to (14);
		\draw [style=dot] (22) to (23);
		\draw [style=dash] (23) to (24);
		\draw [style=dot] (20) to (21);
	\end{pgfonlayer}
\end{tikzpicture}
\caption{}
\end{figure}

Figure 3.5 shows a game in which Player B is to move and Player A has won the chain fight, forcing an odd number of chains on a $5$ x $5$ board.  In this situation it appears as though Player B has no option but to continue making the chain in the upper right and be the first to open a long-chain in the final phase, thereby losing the game.  However Player B decides to play the following move instead:


\begin{figure}[H]
\centering
\begin{tikzpicture}
	\begin{pgfonlayer}{nodelayer}
		\node [style=node] (0) at (-4, -0) {};
		\node [style=node] (1) at (-3, -0) {};
		\node [style=node] (2) at (-2, -0) {};
		\node [style=node] (3) at (-1, -0) {};
		\node [style=node] (4) at (0, -0) {};
		\node [style=node] (5) at (-4, -1) {};
		\node [style=node] (6) at (-3, -1) {};
		\node [style=node] (7) at (-2, -1) {};
		\node [style=node] (8) at (-1, -1) {};hard fast
		\node [style=node] (9) at (0, -1) {};
		\node [style=node] (10) at (-4, -2) {};
		\node [style=node] (11) at (-3, -2) {};
		\node [style=node] (12) at (-2, -2) {};
		\node [style=node] (13) at (-1, -2) {};
		\node [style=node] (14) at (0, -2) {};
		\node [style=node] (15) at (-4, -3) {};
		\node [style=node] (16) at (-3, -3) {};
		\node [style=node] (17) at (-2, -3) {};
		\node [style=node] (18) at (-1, -3) {};
		\node [style=node] (19) at (0, -3) {};
		\node [style=node] (20) at (-4, -4) {};
		\node [style=node] (21) at (-3, -4) {};
		\node [style=node] (22) at (-2, -4) {};
		\node [style=node] (23) at (-1, -4) {};
		\node [style=node] (24) at (0, -4) {};
		\node (25) at (-1.5, -1.5) {\textcolor{red}{\textit{B}}};
		\node (26) at (-1.5, -2.5) {\textcolor{red}{\textit{B}}};
	\end{pgfonlayer}
	\begin{pgfonlayer}{edgelayer}
		\draw [style=dash] (1) to (6);
		\draw [style=dash] (6) to (7);
		\draw [style=dash] (4) to (9);
		\draw [style=dash] (8) to (13);
		\draw [style=dash] (18) to (19);
		\draw [style=dash] (16) to (17);
		\draw [style=dash] (16) to (21);
		\draw [style=dash] (10) to (15);
		\draw [style=dot] (5) to (6);
		\draw [style=dot] (10) to (11);
		\draw [style=bpm] (7) to (8);
		\draw [style=dot] (12) to (13);
		\draw [style=dot] (7) to (12);
		\draw [style=dot] (12) to (17);
		\draw [style=dot] (17) to (18);
		\draw [style=dot] (13) to (18);
		\draw [style=dot] (3) to (4);
		\draw [style=dot] (9) to (14);
		\draw [style=dot] (22) to (23);
		\draw [style=dash] (23) to (24);
		\draw [style=dot] (20) to (21);
		\draw [style=bpm] (13) to (14);
	\end{pgfonlayer}
\end{tikzpicture}
\caption{}
\end{figure}

This is known as a preemptive sacrifice and enables Player B to force a tie, if Player A plays perfectly, or a win if he blunders. There are two routes of play after the disjoined box is captured: either Player A accepts the two boxes offered or he declines them.


\begin{multicols}{2}
\begin{figure}[H]
\centering
\begin{tikzpicture}
	\begin{pgfonlayer}{nodelayer}
		\node [style=node] (0) at (-4, -0) {};
		\node [style=node] (1) at (-3, -0) {};
		\node [style=node] (2) at (-2, -0) {};
		\node [style=node] (3) at (-1, -0) {};
		\node [style=node] (4) at (0, -0) {};
		\node [style=node] (5) at (-4, -1) {};
		\node [style=node] (6) at (-3, -1) {};
		\node [style=node] (7) at (-2, -1) {};
		\node [style=node] (8) at (-1, -1) {};
		\node [style=node] (9) at (0, -1) {};
		\node [style=node] (10) at (-4, -2) {};
		\node [style=node] (11) at (-3, -2) {};
		\node [style=node] (12) at (-2, -2) {};
		\node [style=node] (13) at (-1, -2) {};
		\node [style=node] (14) at (0, -2) {};
		\node [style=node] (15) at (-4, -3) {};
		\node [style=node] (16) at (-3, -3) {};
		\node [style=node] (17) at (-2, -3) {};
		\node [style=node] (18) at (-1, -3) {};
		\node [style=node] (19) at (0, -3) {};
		\node [style=node] (20) at (-4, -4) {};
		\node [style=node] (21) at (-3, -4) {};
		\node [style=node] (22) at (-2, -4) {};
		\node [style=node] (23) at (-1, -4) {};
		\node [style=node] (24) at (0, -4) {};
		\node (25) at (-1.5, -1.5) {\textcolor{red}{\textit{B}}};
		\node (26) at (-1.5, -2.5) {\textcolor{red}{\textit{B}}};
		\node (27) at (-0.5, -2.5) {\textcolor{blue}{\textit{A}}};
		\node (28) at (-0.5, -1.5) {\textcolor{blue}{\textit{A}}};
		\node (29) at (-0.5, -0.5) {\textcolor{blue}{\textit{A}}};
	\end{pgfonlayer}
	\begin{pgfonlayer}{edgelayer}
		\draw [style=dash] (1) to (6);
		\draw [style=dash] (6) to (7);
		\draw [style=dash] (4) to (9);
		\draw [style=dash] (8) to (13);
		\draw [style=dash] (18) to (19);
		\draw [style=dash] (16) to (17);
		\draw [style=dash] (16) to (21);
		\draw [style=dash] (10) to (15);
		\draw [style=dot] (5) to (6);
		\draw [style=dot] (10) to (11);
		\draw [style=dot] (7) to (8);
		\draw [style=dot] (12) to (13);
		\draw [style=dot] (7) to (12);
		\draw [style=dot] (12) to (17);
		\draw [style=dot] (17) to (18);
		\draw [style=dot] (13) to (18);
		\draw [style=dot] (3) to (4);
		\draw [style=dot] (9) to (14);
		\draw [style=dot] (22) to (23);
		\draw [style=dash] (23) to (24);
		\draw [style=dot] (20) to (21);
		\draw [style=dot] (13) to (14);
		\draw [style=apm] (14) to (19);
		\draw [style=apm] (8) to (9);
		\draw [style=apm] (3) to (8);
	\end{pgfonlayer}
\end{tikzpicture}
\caption{Accepts. Player A to move.}
\end{figure}

\begin{figure}[H]
\centering
\begin{tikzpicture}
	\begin{pgfonlayer}{nodelayer}
		\node [style=node] (0) at (-4, -0) {};
		\node [style=node] (1) at (-3, -0) {};
		\node [style=node] (2) at (-2, -0) {};
		\node [style=node] (3) at (-1, -0) {};
		\node [style=node] (4) at (0, -0) {};
		\node [style=node] (5) at (-4, -1) {};
		\node [style=node] (6) at (-3, -1) {};
		\node [style=node] (7) at (-2, -1) {};
		\node [style=node] (8) at (-1, -1) {};
		\node [style=node] (9) at (0, -1) {};
		\node [style=node] (10) at (-4, -2) {};
		\node [style=node] (11) at (-3, -2) {};
		\node [style=node] (12) at (-2, -2) {};
		\node [style=node] (13) at (-1, -2) {};
		\node [style=node] (14) at (0, -2) {};
		\node [style=node] (15) at (-4, -3) {};
		\node [style=node] (16) at (-3, -3) {};
		\node [style=node] (17) at (-2, -3) {};
		\node [style=node] (18) at (-1, -3) {};
		\node [style=node] (19) at (0, -3) {};
		\node [style=node] (20) at (-4, -4) {};
		\node [style=node] (21) at (-3, -4) {};
		\node [style=node] (22) at (-2, -4) {};
		\node [style=node] (23) at (-1, -4) {};
		\node [style=node] (24) at (0, -4) {};
		\node (25) at (-1.5, -1.5) {\textcolor{red}{\textit{B}}};
		\node (26) at (-1.5, -2.5) {\textcolor{red}{\textit{B}}};
		\node (27) at (-0.5, -2.5) {\textcolor{blue}{\textit{A}}};
	\end{pgfonlayer}
	\begin{pgfonlayer}{edgelayer}
		\draw [style=dash] (1) to (6);
		\draw [style=dash] (6) to (7);
		\draw [style=dash] (4) to (9);
		\draw [style=dash] (8) to (13);
		\draw [style=dash] (18) to (19);
		\draw [style=dash] (16) to (17);
		\draw [style=dash] (16) to (21);
		\draw [style=dash] (10) to (15);
		\draw [style=dot] (5) to (6);
		\draw [style=dot] (10) to (11);
		\draw [style=dot] (7) to (8);
		\draw [style=dot] (12) to (13);
		\draw [style=dot] (7) to (12);
		\draw [style=dot] (12) to (17);
		\draw [style=dot] (17) to (18);
		\draw [style=dot] (13) to (18);
		\draw [style=dot] (3) to (4);
		\draw [style=dot] (9) to (14);
		\draw [style=dot] (22) to (23);
		\draw [style=dash] (23) to (24);
		\draw [style=dot] (20) to (21);
		\draw [style=dot] (13) to (14);
		\draw [style=apm] (14) to (19);
		\draw [style=apm] (3) to (8);
	\end{pgfonlayer}
\end{tikzpicture}
\caption{Declines. Player B to move.}
\end{figure}
\end{multicols}

We see that in Figure 3.7 Player A has accepted the two boxes sacrificed and reduced the chain count to $2$ without leaving a double-cross move.  This implies that Player B has now won the chain fight given there are an even number of long-chains and an odd number of dots.  Therefore no matter what move Player A chooses to play his opponent will always be able to force him to be the first to open a long-chain, and the best score he can hope for is a 7-9 loss.  The following figures illustrate three possible moves Player A can make and what the end score would be if both sides play perfectly:


\begin{multicols}{2}
\begin{figure}[H]
\centering
\begin{tikzpicture}
	\begin{pgfonlayer}{nodelayer}
		\node [style=node] (0) at (-4, -0) {};
		\node [style=node] (1) at (-3, -0) {};
		\node [style=node] (2) at (-2, -0) {};
		\node [style=node] (3) at (-1, -0) {};
		\node [style=node] (4) at (0, -0) {};
		\node [style=node] (5) at (-4, -1) {};
		\node [style=node] (6) at (-3, -1) {};
		\node [style=node] (7) at (-2, -1) {};
		\node [style=node] (8) at (-1, -1) {};
		\node [style=node] (9) at (0, -1) {};
		\node [style=node] (10) at (-4, -2) {};
		\node [style=node] (11) at (-3, -2) {};
		\node [style=node] (12) at (-2, -2) {};
		\node [style=node] (13) at (-1, -2) {};
		\node [style=node] (14) at (0, -2) {};
		\node [style=node] (15) at (-4, -3) {};
		\node [style=node] (16) at (-3, -3) {};
		\node [style=node] (17) at (-2, -3) {};
		\node [style=node] (18) at (-1, -3) {};
		\node [style=node] (19) at (0, -3) {};
		\node [style=node] (20) at (-4, -4) {};
		\node [style=node] (21) at (-3, -4) {};
		\node [style=node] (22) at (-2, -4) {};
		\node [style=node] (23) at (-1, -4) {};
		\node [style=node] (24) at (0, -4) {};
		\node (25) at (-1.5, -1.5) {\textcolor{red}{\textit{B}}};
		\node (26) at (-1.5, -2.5) {\textcolor{red}{\textit{B}}};
		\node (27) at (-0.5, -2.5) {\textcolor{blue}{\textit{A}}};
		\node (28) at (-0.5, -1.5) {\textcolor{blue}{\textit{A}}};
		\node (29) at (-0.5, -0.5) {\textcolor{blue}{\textit{A}}};
	\end{pgfonlayer}
	\begin{pgfonlayer}{edgelayer}
		\draw [style=dash] (1) to (6);
		\draw [style=dash] (6) to (7);
		\draw [style=dash] (4) to (9);
		\draw [style=dash] (8) to (13);
		\draw [style=dash] (18) to (19);
		\draw [style=dash] (16) to (17);
		\draw [style=dash] (16) to (21);
		\draw [style=dash] (10) to (15);
		\draw [style=dot] (5) to (6);
		\draw [style=dot] (10) to (11);
		\draw [style=dot] (7) to (8);
		\draw [style=dot] (12) to (13);
		\draw [style=dot] (7) to (12);
		\draw [style=dot] (12) to (17);
		\draw [style=dot] (17) to (18);
		\draw [style=dot] (13) to (18);
		\draw [style=dot] (3) to (4);
		\draw [style=dot] (9) to (14);
		\draw [style=dot] (22) to (23);
		\draw [style=dash] (23) to (24);
		\draw [style=dot] (20) to (21);
		\draw [style=dot] (13) to (14);
		\draw [style=apm] (14) to (19);
		\draw [style=apm] (3) to (8);
		\draw [style=apm] (8) to (9);
		\draw [style=poss] (2) to (3);
		\draw [style=poss] (1) to (2);
	\end{pgfonlayer}
\end{tikzpicture}
\caption{Half-Hearted Handout. \\\textbf{\textbf{End Score}}: 6-10.}
\end{figure}



\begin{figure}[H]
\centering
\begin{tikzpicture}
	\begin{pgfonlayer}{nodelayer}
		\node [style=node] (0) at (-4, -0) {};
		\node [style=node] (1) at (-3, -0) {};
		\node [style=node] (2) at (-2, -0) {};
		\node [style=node] (3) at (-1, -0) {};
		\node [style=node] (4) at (0, -0) {};
		\node [style=node] (5) at (-4, -1) {};
		\node [style=node] (6) at (-3, -1) {};
		\node [style=node] (7) at (-2, -1) {};
		\node [style=node] (8) at (-1, -1) {};
		\node [style=node] (9) at (0, -1) {};
		\node [style=node] (10) at (-4, -2) {};
		\node [style=node] (11) at (-3, -2) {};
		\node [style=node] (12) at (-2, -2) {};
		\node [style=node] (13) at (-1, -2) {};
		\node [style=node] (14) at (0, -2) {};
		\node [style=node] (15) at (-4, -3) {};
		\node [style=node] (16) at (-3, -3) {};
		\node [style=node] (17) at (-2, -3) {};
		\node [style=node] (18) at (-1, -3) {};
		\node [style=node] (19) at (0, -3) {};
		\node [style=node] (20) at (-4, -4) {};
		\node [style=node] (21) at (-3, -4) {};
		\node [style=node] (22) at (-2, -4) {};
		\node [style=node] (23) at (-1, -4) {};
		\node [style=node] (24) at (0, -4) {};
		\node (25) at (-1.5, -1.5) {\textcolor{red}{\textit{B}}};
		\node (26) at (-1.5, -2.5) {\textcolor{red}{\textit{B}}};
		\node (27) at (-0.5, -2.5) {\textcolor{blue}{\textit{A}}};
		\node (28) at (-0.5, -1.5) {\textcolor{blue}{\textit{A}}};
		\node (29) at (-0.5, -0.5) {\textcolor{blue}{\textit{A}}};
	\end{pgfonlayer}
	\begin{pgfonlayer}{edgelayer}
		\draw [style=dash] (1) to (6);
		\draw [style=dash] (6) to (7);
		\draw [style=dash] (4) to (9);
		\draw [style=dash] (8) to (13);
		\draw [style=dash] (18) to (19);
		\draw [style=dash] (16) to (17);
		\draw [style=dash] (16) to (21);
		\draw [style=dash] (10) to (15);
		\draw [style=dot] (5) to (6);
		\draw [style=dot] (10) to (11);
		\draw [style=dot] (7) to (8);
		\draw [style=dot] (12) to (13);
		\draw [style=dot] (7) to (12);
		\draw [style=dot] (12) to (17);
		\draw [style=dot] (17) to (18);
		\draw [style=dot] (13) to (18);
		\draw [style=dot] (3) to (4);
		\draw [style=dot] (9) to (14);
		\draw [style=dot] (22) to (23);
		\draw [style=dash] (23) to (24);
		\draw [style=dot] (20) to (21);
		\draw [style=dot] (13) to (14);
		\draw [style=apm] (14) to (19);
		\draw [style=apm] (3) to (8);
		\draw [style=apm] (8) to (9);
		\draw [style=poss] (2) to (7);
	\end{pgfonlayer}
\end{tikzpicture}
\caption{Hard-Hearted Handout. \\\textbf{\textbf{End Score}}: 6-10.}
\end{figure}
\end{multicols}

\begin{figure}[H]
\centering
\begin{tikzpicture}
	\begin{pgfonlayer}{nodelayer}
		\node [style=node] (0) at (-4, -0) {};
		\node [style=node] (1) at (-3, -0) {};
		\node [style=node] (2) at (-2, -0) {};
		\node [style=node] (3) at (-1, -0) {};
		\node [style=node] (4) at (0, -0) {};
		\node [style=node] (5) at (-4, -1) {};
		\node [style=node] (6) at (-3, -1) {};
		\node [style=node] (7) at (-2, -1) {};
		\node [style=node] (8) at (-1, -1) {};
		\node [style=node] (9) at (0, -1) {};
		\node [style=node] (10) at (-4, -2) {};
		\node [style=node] (11) at (-3, -2) {};
		\node [style=node] (12) at (-2, -2) {};
		\node [style=node] (13) at (-1, -2) {};
		\node [style=node] (14) at (0, -2) {};
		\node [style=node] (15) at (-4, -3) {};
		\node [style=node] (16) at (-3, -3) {};
		\node [style=node] (17) at (-2, -3) {};
		\node [style=node] (18) at (-1, -3) {};
		\node [style=node] (19) at (0, -3) {};
		\node [style=node] (20) at (-4, -4) {};
		\node [style=node] (21) at (-3, -4) {};
		\node [style=node] (22) at (-2, -4) {};
		\node [style=node] (23) at (-1, -4) {};
		\node [style=node] (24) at (0, -4) {};
		\node (25) at (-1.5, -1.5) {\textcolor{red}{\textit{B}}};
		\node (26) at (-1.5, -2.5) {\textcolor{red}{\textit{B}}};
		\node (27) at (-0.5, -2.5) {\textcolor{blue}{\textit{A}}};
		\node (28) at (-0.5, -1.5) {\textcolor{blue}{\textit{A}}};
		\node (29) at (-0.5, -0.5) {\textcolor{blue}{\textit{A}}};
	\end{pgfonlayer}
	\begin{pgfonlayer}{edgelayer}
		\draw [style=dash] (1) to (6);
		\draw [style=dash] (6) to (7);
		\draw [style=dash] (4) to (9);
		\draw [style=dash] (8) to (13);
		\draw [style=dash] (18) to (19);
		\draw [style=dash] (16) to (17);
		\draw [style=dash] (16) to (21);
		\draw [style=dash] (10) to (15);
		\draw [style=dot] (5) to (6);
		\draw [style=dot] (10) to (11);
		\draw [style=dot] (7) to (8);
		\draw [style=dot] (12) to (13);
		\draw [style=dot] (7) to (12);
		\draw [style=dot] (12) to (17);
		\draw [style=dot] (17) to (18);
		\draw [style=dot] (13) to (18);
		\draw [style=dot] (3) to (4);
		\draw [style=dot] (9) to (14);
		\draw [style=dot] (22) to (23);
		\draw [style=dash] (23) to (24);
		\draw [style=dot] (20) to (21);
		\draw [style=dot] (13) to (14);
		\draw [style=apm] (14) to (19);
		\draw [style=apm] (3) to (8);
		\draw [style=apm] (8) to (9);
		\draw [style=poss] (0) to (1);
		\draw [style=poss] (0) to (5);
	\end{pgfonlayer}
\end{tikzpicture}
\caption{Open singleton. \textbf{\textbf{End Score}}: 7-9.}
\end{figure}


As it is now clear I'm sure, Player A must decline the two boxes and force his opponent to play a double-cross move.  From here as long as both players play perfectly Player B will be able to save the game and force a tie.  Player B must open the smallest chain available, the singleton, to maximize the boxes captured in the short chain phase.  This forces Player A to take and play a hard-hearted handout -- ensuring that no double-cross moves will be played until the final phase of the game.  And from here the game plays out as one would expect resulting in an 8-8 tie. Figures 3.12-17 illustrate how this works.

\newpage
\begin{multicols}{2}
\begin{figure}[H]
\centering
\begin{tikzpicture}
	\begin{pgfonlayer}{nodelayer}
		\node [style=node] (0) at (-4, -0) {};
		\node [style=node] (1) at (-3, -0) {};
		\node [style=node] (2) at (-2, -0) {};
		\node [style=node] (3) at (-1, -0) {};
		\node [style=node] (4) at (0, -0) {};
		\node [style=node] (5) at (-4, -1) {};
		\node [style=node] (6) at (-3, -1) {};
		\node [style=node] (7) at (-2, -1) {};
		\node [style=node] (8) at (-1, -1) {};
		\node [style=node] (9) at (0, -1) {};
		\node [style=node] (10) at (-4, -2) {};
		\node [style=node] (11) at (-3, -2) {};
		\node [style=node] (12) at (-2, -2) {};
		\node [style=node] (13) at (-1, -2) {};
		\node [style=node] (14) at (0, -2) {};
		\node [style=node] (15) at (-4, -3) {};
		\node [style=node] (16) at (-3, -3) {};
		\node [style=node] (17) at (-2, -3) {};
		\node [style=node] (18) at (-1, -3) {};
		\node [style=node] (19) at (0, -3) {};
		\node [style=node] (20) at (-4, -4) {};
		\node [style=node] (21) at (-3, -4) {};
		\node [style=node] (22) at (-2, -4) {};
		\node [style=node] (23) at (-1, -4) {};
		\node [style=node] (24) at (0, -4) {};
		\node (25) at (-1.5, -1.5) {\textcolor{red}{\textit{B}}};
		\node (26) at (-1.5, -2.5) {\textcolor{red}{\textit{B}}};
		\node (27) at (-0.5, -2.5) {\textcolor{blue}{\textit{A}}};
		\node (28) at (-0.5, -1.5) {\textcolor{red}{\textit{B}}};
		\node (29) at (-0.5, -0.5) {\textcolor{red}{\textit{B}}};
	\end{pgfonlayer}
	\begin{pgfonlayer}{edgelayer}
		\draw [style=dash] (1) to (6);
		\draw [style=dash] (6) to (7);
		\draw [style=dash] (4) to (9);
		\draw [style=dash] (8) to (13);
		\draw [style=dash] (18) to (19);
		\draw [style=dash] (16) to (17);
		\draw [style=dash] (16) to (21);
		\draw [style=dash] (10) to (15);
		\draw [style=dot] (5) to (6);
		\draw [style=dot] (10) to (11);
		\draw [style=dot] (7) to (8);
		\draw [style=dot] (12) to (13);
		\draw [style=dot] (7) to (12);
		\draw [style=dot] (12) to (17);
		\draw [style=dot] (17) to (18);
		\draw [style=dot] (13) to (18);
		\draw [style=dot] (3) to (4);
		\draw [style=dot] (9) to (14);
		\draw [style=dot] (22) to (23);
		\draw [style=dash] (23) to (24);
		\draw [style=dot] (20) to (21);
		\draw [style=dot] (13) to (14);
		\draw [style=dash] (14) to (19);
		\draw [style=dash] (3) to (8);
		\draw [style=bpm] (8) to (9);
		\draw [style=bpm] (0) to (1);
	\end{pgfonlayer}
\end{tikzpicture}
\caption{Player B accepts the two boxes with a double-cross move, and opens the singleton. \textbf{\textbf{Score}}: 1-4.}
\end{figure}

\begin{figure}[H]
\centering
\begin{tikzpicture}
	\begin{pgfonlayer}{nodelayer}
		\node [style=node] (0) at (-4, -0) {};
		\node [style=node] (1) at (-3, -0) {};
		\node [style=node] (2) at (-2, -0) {};
		\node [style=node] (3) at (-1, -0) {};
		\node [style=node] (4) at (0, -0) {};
		\node [style=node] (5) at (-4, -1) {};
		\node [style=node] (6) at (-3, -1) {};
		\node [style=node] (7) at (-2, -1) {};
		\node [style=node] (8) at (-1, -1) {};
		\node [style=node] (9) at (0, -1) {};
		\node [style=node] (10) at (-4, -2) {};
		\node [style=node] (11) at (-3, -2) {};
		\node [style=node] (12) at (-2, -2) {};
		\node [style=node] (13) at (-1, -2) {};
		\node [style=node] (14) at (0, -2) {};
		\node [style=node] (15) at (-4, -3) {};
		\node [style=node] (16) at (-3, -3) {};
		\node [style=node] (17) at (-2, -3) {};
		\node [style=node] (18) at (-1, -3) {};
		\node [style=node] (19) at (0, -3) {};
		\node [style=node] (20) at (-4, -4) {};
		\node [style=node] (21) at (-3, -4) {};
		\node [style=node] (22) at (-2, -4) {};
		\node [style=node] (23) at (-1, -4) {};
		\node [style=node] (24) at (0, -4) {};
		\node (25) at (-1.5, -1.5) {\textcolor{red}{\textit{B}}};
		\node (26) at (-1.5, -2.5) {\textcolor{red}{\textit{B}}};
		\node (27) at (-0.5, -2.5) {\textcolor{blue}{\textit{A}}};
		\node (28) at (-0.5, -1.5) {\textcolor{red}{\textit{B}}};
		\node (29) at (-0.5, -0.5) {\textcolor{red}{\textit{B}}};
		\node (30) at (-3.5, -0.5) {\textcolor{blue}{\textit{A}}};
	\end{pgfonlayer}
	\begin{pgfonlayer}{edgelayer}
		\draw [style=dash] (1) to (6);
		\draw [style=dash] (6) to (7);
		\draw [style=dash] (4) to (9);
		\draw [style=dash] (8) to (13);
		\draw [style=dash] (18) to (19);
		\draw [style=dash] (16) to (17);
		\draw [style=dash] (16) to (21);
		\draw [style=dash] (10) to (15);
		\draw [style=dot] (5) to (6);
		\draw [style=dot] (10) to (11);
		\draw [style=dot] (7) to (8);
		\draw [style=dot] (12) to (13);
		\draw [style=dot] (7) to (12);
		\draw [style=dot] (12) to (17);
		\draw [style=dot] (17) to (18);
		\draw [style=dot] (13) to (18);
		\draw [style=dot] (3) to (4);
		\draw [style=dot] (9) to (14);
		\draw [style=dot] (22) to (23);
		\draw [style=dash] (23) to (24);
		\draw [style=dot] (20) to (21);
		\draw [style=dot] (13) to (14);
		\draw [style=dash] (14) to (19);
		\draw [style=dash] (3) to (8);
		\draw [style=dot] (8) to (9);
		\draw [style=dot] (0) to (1);
		\draw [style=apm] (0) to (5);
		\draw [style=apm] (2) to (7);
	\end{pgfonlayer}
\end{tikzpicture}
\caption{Player A takes and then plays a hard-hearted handout.  \newline\textbf{\textbf{Score}}: 2-4.}
\end{figure}
\end{multicols}


\begin{multicols}{2}
\begin{figure}[H]
\centering
\begin{tikzpicture}
	\begin{pgfonlayer}{nodelayer}
		\node [style=node] (0) at (-4, -0) {};
		\node [style=node] (1) at (-3, -0) {};
		\node [style=node] (2) at (-2, -0) {};
		\node [style=node] (3) at (-1, -0) {};
		\node [style=node] (4) at (0, -0) {};
		\node [style=node] (5) at (-4, -1) {};
		\node [style=node] (6) at (-3, -1) {};
		\node [style=node] (7) at (-2, -1) {};
		\node [style=node] (8) at (-1, -1) {};
		\node [style=node] (9) at (0, -1) {};
		\node [style=node] (10) at (-4, -2) {};
		\node [style=node] (11) at (-3, -2) {};
		\node [style=node] (12) at (-2, -2) {};
		\node [style=node] (13) at (-1, -2) {};
		\node [style=node] (14) at (0, -2) {};
		\node [style=node] (15) at (-4, -3) {};
		\node [style=node] (16) at (-3, -3) {};
		\node [style=node] (17) at (-2, -3) {};
		\node [style=node] (18) at (-1, -3) {};
		\node [style=node] (19) at (0, -3) {};
		\node [style=node] (20) at (-4, -4) {};
		\node [style=node] (21) at (-3, -4) {};
		\node [style=node] (22) at (-2, -4) {};
		\node [style=node] (23) at (-1, -4) {};
		\node [style=node] (24) at (0, -4) {};
		\node (25) at (-1.5, -1.5) {\textcolor{red}{\textit{B}}};
		\node (26) at (-1.5, -2.5) {\textcolor{red}{\textit{B}}};
		\node (27) at (-0.5, -2.5) {\textcolor{blue}{\textit{A}}};
		\node (28) at (-0.5, -1.5) {\textcolor{red}{\textit{B}}};
		\node (29) at (-0.5, -0.5) {\textcolor{red}{\textit{B}}};
		\node (30) at (-3.5, -0.5) {\textcolor{blue}{\textit{A}}};
		\node (31) at (-2.5, -0.5) {\textcolor{red}{\textit{B}}};
		\node (32) at (-1.5, -0.5) {\textcolor{red}{\textit{B}}};
	\end{pgfonlayer}
	\begin{pgfonlayer}{edgelayer}
		\draw [style=dash] (1) to (6);
		\draw [style=dash] (6) to (7);
		\draw [style=dash] (4) to (9);
		\draw [style=dash] (8) to (13);
		\draw [style=dash] (18) to (19);
		\draw [style=dash] (16) to (17);
		\draw [style=dash] (16) to (21);
		\draw [style=dash] (10) to (15);
		\draw [style=dot] (5) to (6);
		\draw [style=dot] (10) to (11);
		\draw [style=dot] (7) to (8);
		\draw [style=dot] (12) to (13);
		\draw [style=dot] (7) to (12);
		\draw [style=dot] (12) to (17);
		\draw [style=dot] (17) to (18);
		\draw [style=dot] (13) to (18);
		\draw [style=dot] (3) to (4);
		\draw [style=dot] (9) to (14);
		\draw [style=dot] (22) to (23);
		\draw [style=dash] (23) to (24);
		\draw [style=dot] (20) to (21);
		\draw [style=dot] (13) to (14);
		\draw [style=dash] (14) to (19);
		\draw [style=dash] (3) to (8);
		\draw [style=dot] (8) to (9);
		\draw [style=dot] (0) to (1);
		\draw [style=dash] (0) to (5);
		\draw [style=dash] (2) to (7);
		\draw [style=bpm] (1) to (3);
		\draw [style=bpm] (19) to (24);
	\end{pgfonlayer}
\end{tikzpicture}
\caption{Player B takes and opens the first long-chain. \textbf{\textbf{Score}}: 2-6.}
\end{figure}

\begin{figure}[H]
\centering
\begin{tikzpicture}
	\begin{pgfonlayer}{nodelayer}
		\node [style=node] (0) at (-4, -0) {};
		\node [style=node] (1) at (-3, -0) {};
		\node [style=node] (2) at (-2, -0) {};
		\node [style=node] (3) at (-1, -0) {};
		\node [style=node] (4) at (0, -0) {};
		\node [style=node] (5) at (-4, -1) {};
		\node [style=node] (6) at (-3, -1) {};
		\node [style=node] (7) at (-2, -1) {};
		\node [style=node] (8) at (-1, -1) {};
		\node [style=node] (9) at (0, -1) {};
		\node [style=node] (10) at (-4, -2) {};
		\node [style=node] (11) at (-3, -2) {};
		\node [style=node] (12) at (-2, -2) {};
		\node [style=node] (13) at (-1, -2) {};
		\node [style=node] (14) at (0, -2) {};
		\node [style=node] (15) at (-4, -3) {};
		\node [style=node] (16) at (-3, -3) {};
		\node [style=node] (17) at (-2, -3) {};
		\node [style=node] (18) at (-1, -3) {};
		\node [style=node] (19) at (0, -3) {};
		\node [style=node] (20) at (-4, -4) {};
		\node [style=node] (21) at (-3, -4) {};
		\node [style=node] (22) at (-2, -4) {};
		\node [style=node] (23) at (-1, -4) {};
		\node [style=node] (24) at (0, -4) {};
		\node (25) at (-1.5, -1.5) {\textcolor{red}{\textit{B}}};
		\node (26) at (-1.5, -2.5) {\textcolor{red}{\textit{B}}};
		\node (27) at (-0.5, -2.5) {\textcolor{blue}{\textit{A}}};
		\node (28) at (-0.5, -1.5) {\textcolor{red}{\textit{B}}};
		\node (29) at (-0.5, -0.5) {\textcolor{red}{\textit{B}}};
		\node (30) at (-3.5, -0.5) {\textcolor{blue}{\textit{A}}};
		\node (31) at (-2.5, -0.5) {\textcolor{red}{\textit{B}}};
		\node (32) at (-1.5, -0.5) {\textcolor{red}{\textit{B}}};
		\node (33) at (-0.5, -3.5) {\textcolor{blue}{\textit{A}}};
	\end{pgfonlayer}
	\begin{pgfonlayer}{edgelayer}
		\draw [style=dash] (1) to (6);
		\draw [style=dash] (6) to (7);
		\draw [style=dash] (4) to (9);
		\draw [style=dash] (8) to (13);
		\draw [style=dash] (18) to (19);
		\draw [style=dash] (16) to (17);
		\draw [style=dash] (16) to (21);
		\draw [style=dash] (10) to (15);
		\draw [style=dot] (5) to (6);
		\draw [style=dot] (10) to (11);
		\draw [style=dot] (7) to (8);
		\draw [style=dot] (12) to (13);
		\draw [style=dot] (7) to (12);
		\draw [style=dot] (12) to (17);
		\draw [style=dot] (17) to (18);
		\draw [style=dot] (13) to (18);
		\draw [style=dot] (3) to (4);
		\draw [style=dot] (9) to (14);
		\draw [style=dot] (22) to (23);
		\draw [style=dash] (23) to (24);
		\draw [style=dot] (20) to (21);
		\draw [style=dot] (13) to (14);
		\draw [style=dash] (14) to (19);
		\draw [style=dash] (3) to (8);
		\draw [style=dot] (8) to (9);
		\draw [style=dot] (0) to (1);
		\draw [style=dash] (0) to (5);
		\draw [style=dash] (2) to (7);
		\draw [style=dot] (1) to (3);
		\draw [style=dot] (19) to (24);
		\draw [style=apm] (18) to (23);
		\draw [style=apm] (21) to (22);
	\end{pgfonlayer}
\end{tikzpicture}
\caption{Player A takes all but the last two boxes maintaining control. \textbf{\textbf{Score}}: 3-6.}
\end{figure}
\end{multicols}


\begin{multicols}{2}
\begin{figure}[H]
\centering
\begin{tikzpicture}
	\begin{pgfonlayer}{nodelayer}
		\node [style=node] (0) at (-4, -0) {};
		\node [style=node] (1) at (-3, -0) {};
		\node [style=node] (2) at (-2, -0) {};
		\node [style=node] (3) at (-1, -0) {};
		\node [style=node] (4) at (0, -0) {};
		\node [style=node] (5) at (-4, -1) {};
		\node [style=node] (6) at (-3, -1) {};
		\node [style=node] (7) at (-2, -1) {};
		\node [style=node] (8) at (-1, -1) {};
		\node [style=node] (9) at (0, -1) {};
		\node [style=node] (10) at (-4, -2) {};
		\node [style=node] (11) at (-3, -2) {};
		\node [style=node] (12) at (-2, -2) {};
		\node [style=node] (13) at (-1, -2) {};
		\node [style=node] (14) at (0, -2) {};
		\node [style=node] (15) at (-4, -3) {};
		\node [style=node] (16) at (-3, -3) {};
		\node [style=node] (17) at (-2, -3) {};
		\node [style=node] (18) at (-1, -3) {};
		\node [style=node] (19) at (0, -3) {};
		\node [style=node] (20) at (-4, -4) {};
		\node [style=node] (21) at (-3, -4) {};
		\node [style=node] (22) at (-2, -4) {};
		\node [style=node] (23) at (-1, -4) {};
		\node [style=node] (24) at (0, -4) {};
		\node (25) at (-1.5, -1.5) {\textcolor{red}{\textit{B}}};
		\node (26) at (-1.5, -2.5) {\textcolor{red}{\textit{B}}};
		\node (27) at (-0.5, -2.5) {\textcolor{blue}{\textit{A}}};
		\node (28) at (-0.5, -1.5) {\textcolor{red}{\textit{B}}};
		\node (29) at (-0.5, -0.5) {\textcolor{red}{\textit{B}}};
		\node (30) at (-3.5, -0.5) {\textcolor{blue}{\textit{A}}};
		\node (31) at (-2.5, -0.5) {\textcolor{red}{\textit{B}}};
		\node (32) at (-1.5, -0.5) {\textcolor{red}{\textit{B}}};
		\node (33) at (-0.5, -3.5) {\textcolor{blue}{\textit{A}}};
		\node (34) at (-2.5, -3.5) {\textcolor{red}{\textit{B}}};
		\node (35) at (-1.5, -3.5) {\textcolor{red}{\textit{B}}};
	\end{pgfonlayer}
	\begin{pgfonlayer}{edgelayer}
		\draw [style=dash] (1) to (6);
		\draw [style=dash] (6) to (7);
		\draw [style=dash] (4) to (9);
		\draw [style=dash] (8) to (13);
		\draw [style=dash] (18) to (19);
		\draw [style=dash] (16) to (17);
		\draw [style=dash] (16) to (21);
		\draw [style=dash] (10) to (15);
		\draw [style=dot] (5) to (6);
		\draw [style=dot] (10) to (11);
		\draw [style=dot] (7) to (8);
		\draw [style=dot] (12) to (13);
		\draw [style=dot] (7) to (12);
		\draw [style=dot] (12) to (17);
		\draw [style=dot] (17) to (18);
		\draw [style=dot] (13) to (18);
		\draw [style=dot] (3) to (4);
		\draw [style=dot] (9) to (14);
		\draw [style=dot] (22) to (23);
		\draw [style=dash] (23) to (24);
		\draw [style=dot] (20) to (21);
		\draw [style=dot] (13) to (14);
		\draw [style=dash] (14) to (19);
		\draw [style=dash] (3) to (8);
		\draw [style=dot] (8) to (9);
		\draw [style=dot] (0) to (1);
		\draw [style=dash] (0) to (5);
		\draw [style=dash] (2) to (7);
		\draw [style=dot] (1) to (3);
		\draw [style=dot] (19) to (24);
		\draw [style=dash] (18) to (23);
		\draw [style=dash] (21) to (22);
		\draw [style=bpm] (17) to (22);
		\draw [style=bpm] (15) to (20);
	\end{pgfonlayer}
\end{tikzpicture}
\caption{Player B takes the two boxes with a double-cross move and opens the last chain. \textbf{\textbf{Score}}: 3-8.}
\end{figure}

\begin{figure}[H]
\centering
\begin{tikzpicture}
	\begin{pgfonlayer}{nodelayer}
		\node [style=node] (0) at (-4, -0) {};
		\node [style=node] (1) at (-3, -0) {};
		\node [style=node] (2) at (-2, -0) {};
		\node [style=node] (3) at (-1, -0) {};
		\node [style=node] (4) at (0, -0) {};
		\node [style=node] (5) at (-4, -1) {};
		\node [style=node] (6) at (-3, -1) {};
		\node [style=node] (7) at (-2, -1) {};
		\node [style=node] (8) at (-1, -1) {};
		\node [style=node] (9) at (0, -1) {};
		\node [style=node] (10) at (-4, -2) {};
		\node [style=node] (11) at (-3, -2) {};
		\node [style=node] (12) at (-2, -2) {};
		\node [style=node] (13) at (-1, -2) {};
		\node [style=node] (14) at (0, -2) {};
		\node [style=node] (15) at (-4, -3) {};
		\node [style=node] (16) at (-3, -3) {};
		\node [style=node] (17) at (-2, -3) {};
		\node [style=node] (18) at (-1, -3) {};
		\node [style=node] (19) at (0, -3) {};
		\node [style=node] (20) at (-4, -4) {};
		\node [style=node] (21) at (-3, -4) {};
		\node [style=node] (22) at (-2, -4) {};
		\node [style=node] (23) at (-1, -4) {};
		\node [style=node] (24) at (0, -4) {};
		\node (25) at (-1.5, -1.5) {\textcolor{red}{\textit{B}}};
		\node (26) at (-1.5, -2.5) {\textcolor{red}{\textit{B}}};
		\node (27) at (-0.5, -2.5) {\textcolor{blue}{\textit{A}}};
		\node (28) at (-0.5, -1.5) {\textcolor{red}{\textit{B}}};
		\node (29) at (-0.5, -0.5) {\textcolor{red}{\textit{B}}};
		\node (30) at (-3.5, -0.5) {\textcolor{blue}{\textit{A}}};
		\node (31) at (-2.5, -0.5) {\textcolor{red}{\textit{B}}};
		\node (32) at (-1.5, -0.5) {\textcolor{red}{\textit{B}}};
		\node (33) at (-0.5, -3.5) {\textcolor{blue}{\textit{A}}};
		\node (34) at (-2.5, -3.5) {\textcolor{red}{\textit{B}}};
		\node (35) at (-1.5, -3.5) {\textcolor{red}{\textit{B}}};
		\node (36) at (-3.5, -3.5) {\textcolor{blue}{\textit{A}}};
		\node (37) at (-3.5, -2.5) {\textcolor{blue}{\textit{A}}};
		\node (38) at (-2.5, -2.5) {\textcolor{blue}{\textit{A}}};
		\node (39) at (-2.5, -1.5) {\textcolor{blue}{\textit{A}}};
		\node (40) at (-3.5, -1.5) {\textcolor{blue}{\textit{A}}};
	\end{pgfonlayer}
	\begin{pgfonlayer}{edgelayer}
		\draw [style=dash] (1) to (6);
		\draw [style=dash] (6) to (7);
		\draw [style=dash] (4) to (9);
		\draw [style=dash] (8) to (13);
		\draw [style=dash] (18) to (19);
		\draw [style=dash] (16) to (17);
		\draw [style=dash] (16) to (21);
		\draw [style=dash] (10) to (15);
		\draw [style=dot] (5) to (6);
		\draw [style=dot] (10) to (11);
		\draw [style=dot] (7) to (8);
		\draw [style=dot] (12) to (13);
		\draw [style=dot] (7) to (12);
		\draw [style=dot] (12) to (17);
		\draw [style=dot] (17) to (18);
		\draw [style=dot] (13) to (18);
		\draw [style=dot] (3) to (4);
		\draw [style=dot] (9) to (14);
		\draw [style=dot] (22) to (23);
		\draw [style=dash] (23) to (24);
		\draw [style=dot] (20) to (21);
		\draw [style=dot] (13) to (14);
		\draw [style=dash] (14) to (19);
		\draw [style=dash] (3) to (8);
		\draw [style=dot] (8) to (9);
		\draw [style=dot] (0) to (1);
		\draw [style=dash] (0) to (5);
		\draw [style=dash] (2) to (7);
		\draw [style=dot] (1) to (3);
		\draw [style=dot] (19) to (24);
		\draw [style=dash] (18) to (23);
		\draw [style=dash] (21) to (22);
		\draw [style=dot] (17) to (22);
		\draw [style=dot] (15) to (20);
		\draw [style=apm] (15) to (16);
		\draw [style=apm] (11) to (16);
		\draw [style=apm] (11) to (12);
		\draw [style=apm] (6) to (11);
		\draw [style=apm] (5) to (10);
	\end{pgfonlayer}
\end{tikzpicture}
\caption{Player A captures the five remaining boxes tying the game. \\\textbf{\textbf{Score}}: 8-8.}
\end{figure}
\end{multicols}

So, in the end Player B is able to force a tie out of his opponent with a preemptive sacrifice in a game that otherwise would have been a loss. 

Now as is true with every strategy in Dots and Boxes we must qualify the conditions for when it is valid to use the Preemptive Sacrifice.  In order to tie the game with a preemptive sacrifice when you have lost the chain fight, you must be at least one box up.  And to possibly win, you must be up by at least two boxes on a board with an odd number of boxes, and at least three boxes on a board with an even number of boxes.

\begin{theorem}
It is possible to tie a game using a preemptive sacrifice if and only if you are leading by at least one box prior to making the sacrifice. 
\end{theorem}

\begin{proof}
Assume Player A and Player B are tied in boxes, and Player A has made a preemptive sacrifice.  Then, given the shortest long-chain is of length three, Player B will capture at least one box prior to choosing a final move.  Player B can either play a double-dealing move and go second for the remaining game, or he can capture the last two boxes in the chain and go first for the remaining game.  Given the Loony Theorem, we know that this choice allows Player B to capture at least half of the boxes left on the board.  However he has already captured one box.  Therefore, Player B is guaranteed a win.
\end{proof}

\begin{theorem}
If there are an even number of boxes on the board, then it is possible for one to win using a preemptive sacrifice if and only if they are up by at least three boxes prior to the sacrifice.  And if there are an odd number of boxes, one must be up by at least two boxes prior to the sacrifice in order for a win to be possible.
\end{theorem}

\begin{proof}
Let $TB$ be the total number of boxes on the board, let $A$ and $B$ be the number of boxes Player A and Player B have, respectively, before the sacrifice is made, and let $C$ be the number of boxes in the sacrificed long-chain.  Assume Player B makes a preemptive sacrifice.  Then, either Player A will capture all the boxes in the sacrificed long-chain and move first for the remaining game, or he will capture all but the last two boxes and force Player B to go first for the rest of the game.  Let $M$ be the maximum number of boxes that can be captured by the player who moves first for the rest of the game.  Thus, if Player A takes all the boxes in the sacrificed long-chain we get Equation 3.7, otherwise we get Equation 3.8.
\begin{equation}
\emph{A} + \emph{C} + \emph{M} = \textit{total boxes captured by Player A.}
\end{equation}
\begin{equation}
\emph{B} + 2 + \emph{M} = \textit{total boxes captured by Player B.}
\end{equation}
In order for Player B to be guaranteed a win in both cases the following must be true:
\begin{equation}
\emph{B} + 2 + \emph{M} > \frac{TB}{2} > \emph{A} + \emph{C} + \emph{M}.
\end{equation}
Now, subtracting \emph{A}+\emph{M}+2 from the inequality we get:
\begin{equation}
\emph{B} - \emph{A} > \frac{TB}{2} - \emph{A} - \emph{M} - 2 > \emph{C} - 2.
\end{equation}
Given C is a long-chain, it must contain at least three boxes.  This yields the following:
\begin{equation}
\emph{B} - \emph{A} > \frac{TB}{2} - \emph{A} - \emph{M} - 2 > 1.
\end{equation}
Assume \emph{TB} is even.  Then the middle term will be an integer greater than one.  Therefore, since \emph{B}-\emph{A} is an integer, the difference between Player B's score and Player A's score prior to the sacrifice must be at least three.  However if \emph{TB} is odd, the smallest possible middle term is one and a half, in which case \emph{B}-\emph{A} can be two.
\end{proof}
The above theorem gives the conditions for when it is \emph{possible} to win using a preemptive sacrifice; but in order to determine if playing one \emph{will} lead to a win, one must use Equations 3.7 and 3.8 and conclude that in either case the opposing player will score less than half the total boxes.  The Preemptive Sacrifice is a useful strategy when you have lost the chain fight, but more often than not it can be prevented if you know what to look for.  Thus it is important to study these conditions and avoid them whenever you have won the chain fight.



\chapter{Nimdots and The Nimdots Method}
In this chapter we explore a general strategy for attaining control that does not require the framework of a Normal Game.  The Chain Rule provided a good heuristic for securing control in a Normal Game.  And in the last section of the previous chapter, we were able to modify the Chain Rule to work in Non-Normal Games, allowing for a more general approach to acquiring control.  In the end however, the Chain Rule had its limitations and thus did not give a complete picture for analyzing every possible position in Dots and Boxes.  This is where the Nimdots Method comes in.  As we will see, the Nimdots Method uses a Dots and Boxes variant, known as Nimdots, to evaluate subgames and determine who will acquire control, and the moves required to do so.

Ultimately though, the Nimdots Method is not without limitations of its own.  As we know, control does not necessarily imply a win.  This is one of the major limitations not just of the Nimdots Method, but of any strategy that is centered on acquiring control.  We must always be conscious of the number of boxes that must be sacrificed in order to achieve control, and the number of boxes that will be captured once we have control.  Beyond the issue of control, the Nimdots Method also relies on one's ability to recursively solve Nimdots games on the fly.  And even though this is always possible\footnote[1]{A complete description of how to do this is given in Section 4.3 \emph{Solving Nimdots Positions}.}, given enough time and commitment, it is not always feasible while playing a game.  This implies that often times players must rely on their memory of specific Nimdots positions and work towards \emph{these} positions, rather than general positions, in order to secure control.  This will be clarified later, but for now understand that it is one of the downsides to this strategy.  All this being said, the Nimdots Method is still a vastly useful strategy for both acquiring control and winning Dots and Boxes games.


\section{Subgames}
In order to properly use the Nimdots Method in Dots and Boxes we first need to understand the concept of a subgame.

\begin{mydef}[Subgame]
A \textbf{subgame} is any section of a Dots and Boxes game wherein moves made in that section only affect the position of that section, and have no affect on the positions of other subgames.  Every Dots and Boxes game is either a subgame or a composite game. 
\end{mydef}

\begin{mydef}[Composite Game]
A game $G$ is a \textbf{composite game} if  \newline $G = G_1 + G_2 + \cdots + G_n$, where $G_1 , G_2 , \dots , G_n$ are subgames and $n>1$.
\end{mydef}

Before a Dots and Boxes game begins, the only subgame is the entire board.  But as a game progresses, the board inevitably  becomes partitioned into more subgames.  The following is an example of a composite game and its subgames:

\begin{figure}[H]
\centering
\begin{tikzpicture}
	\begin{pgfonlayer}{nodelayer}
		\node [style=node] (0) at (0, -0) {};
		\node [style=node] (1) at (1, -0) {};
		\node [style=node] (2) at (2, -0) {};
		\node [style=node] (3) at (3, -0) {};
		\node [style=node] (4) at (0, -1) {};
		\node [style=node] (5) at (1, -1) {};
		\node [style=node] (6) at (2, -1) {};
		\node [style=node] (7) at (3, -1) {};
		\node [style=node] (8) at (0, -2) {};
		\node [style=node] (9) at (1, -2) {};
		\node [style=node] (10) at (2, -2) {};
		\node [style=node] (11) at (3, -2) {};
		\node [style=node] (12) at (0, -3) {};
		\node [style=node] (13) at (1, -3) {};
		\node [style=node] (14) at (2, -3) {};
		\node [style=node] (15) at (3, -3) {};
		\node [style=node] (16) at (5, -0) {};
		\node [style=node] (17) at (5, -1) {};
		\node [style=node] (18) at (5, -2) {};
		\node [style=node] (19) at (5, -3) {};
		\node [style=node] (20) at (6, -3) {};
		\node [style=node] (21) at (6, -2) {};
		\node [style=node] (22) at (6, -1) {};
		\node [style=node] (23) at (6, -0) {};
		\node [style=node] (24) at (8, -0) {};
		\node [style=node] (25) at (9, -0) {};
		\node [style=node] (26) at (10, -0) {};
		\node [style=node] (27) at (8, -1) {};
		\node [style=node] (28) at (9, -1) {};
		\node [style=node] (29) at (10, -1) {};
		\node [style=node] (30) at (8, -2) {};
		\node [style=node] (31) at (9, -2) {};
		\node [style=node] (32) at (10, -2) {};
		\node [style=node] (33) at (12, -2) {};
		\node [style=node] (34) at (13, -2) {};
		\node [style=node] (35) at (14, -2) {};
		\node [style=node] (36) at (12, -3) {};
		\node [style=node] (37) at (13, -3) {};
		\node [style=node] (38) at (14, -3) {};
		\node [style=node] (39) at (12, -2) {};
		\node (40) at (4, -1.5) {=};
		\node (41) at (7, -1.5) {+};
		\node (42) at (11, -1.5) {+};
		\node (43) at (1.5, -3.75) {$G$};
		\node (44) at (5.5, -3.75) {$G_1$};
		\node (45) at (9, -3.75) {$G_2$};
		\node (46) at (13, -3.75) {$G_3$};
	\end{pgfonlayer}
	\begin{pgfonlayer}{edgelayer}
		\draw [style=simple] (36) to (33);
		\draw [style=simple] (33) to (35);
		\draw [style=simple] (37) to (38);
		\draw [style=simple] (24) to (30);
		\draw [style=simple] (30) to (32);
		\draw [style=simple] (23) to (20);
		\draw [style=simple] (17) to (19);
		\draw [style=simple] (4) to (12);
		\draw [style=simple] (1) to (13);
		\draw [style=simple] (9) to (11);
		\draw [style=simple] (14) to (15);
		\draw [style=simple] (1) to (3);
		\draw [style=simple] (6) to (7);
		\draw [style=simple] (24) to (26);
		\draw [style=simple] (28) to (29);
	\end{pgfonlayer}
\end{tikzpicture}
\caption{4x4 board consisting of three subgames.}
\end{figure}

By constructing games in this fashion we are able to analyze each subgame individually and make conclusions about the game as a whole.


\section{The Game of Nimdots}
The game of Nimdots was invented by Berlekamp\footnote[2]{Actually Berlekamp invented a game called Nimstring, but the game of Nimdots is isomorphic to the game of Nimstring, and for our purposes it is better to consider the game of Nimdots.} in order to better study the game of Dots and Boxes.  Both Nimdots and Dots and Boxes are played on a rectangular grid of dots, and both consist of two players alternating turns each consisting of making moves joining two adjacent dots.  In fact, the only difference between Nimdots and Dots and Boxes is the winning condition -- all other rules are the same.  In Dots and Boxes, the player who captures the most boxes by the end of the game is declared the winner.  In Nimdots however, if a player makes the last possible move on a board then he has lost the game, and his opponent is declared the winner.  That is, if you take the last box in a game of Nimdots then you have lost.  This is known as the \textbf{Normal Play Rule}, not to be confused with Normal Play, and is a common winning condition for other impartial games. 

The motivation for inventing the game of Nimdots is twofold.  Firstly, as was mentioned earlier, Nimdots provides a strategy for gaining control.  Using Nimdots to attain control is known as the Nimdots Method and is discussed at length in Section 4.4.  Secondly, there has been substantial work done on impartial games that abide by the Normal Play Rule.  Specifically, it has been shown that every impartial game under the Normal Play Rule is reducible to a value\footnote[3]{This was shown independently by \emph{R. P. Sprague} in 1935, and \emph{P. M. Grundy} in 1939.  It is known as the Sprague-Grundy Theorem.} (in our case we will refer to this as the Nimdots value).  This, as we will see, is useful because it allows us to determine the the outcome of every Nimdots game under best play.  Thus, by imposing the Normal Play Rule on Dots and Boxes, we are able to solve any Nimdots game using its corresponding Nimdots values.


\section{Solving Nimdots Games}
Like Dots and Boxes, a Nimdots game is either a subgame or a composite game.  If the game is a subgame, then in order to find a solution, one must analyze every possible move and determine their respective values, or \textbf{nim-values}.  After the set of all nim-values is constructed, one can then compute the overall subgame value, at which point we say the subgame has been solved.  The value of a composite game, however, is not computed directly from the set of all nim-values, but rather is derived from the values of its subgames. Thus, in order to solve for the value of a general Nimdots position, known as a \textbf{Nimdots value}, we must first understand how to solve for the Nimdots value of a subgame.

\subsection{Solving Subgames}
To solve for the Nimdots value of a subgame we will define a recursive algorithm.  But before we look at the algorithm itself, we need to formally define the terms nim-value and Nimdots subgame value.  After which, we will be able to easily identify the recursive step and terminal case, producing the well defined recursive algorithm for solving Nimdots subgames.

A Nimdots subgame value is computed using a function known as the Minimal Excludant function, or MEX function, with nim-values as arguments.  Thus, we begin by introducing the MEX function.


\begin{mydef}[MEX Function]
Let $\{ x_1,x_2,\dots,x_n \} \subset \mathbb N^0$, such that  $x_1,x_2,\dots,x_n$ are nim-values.  Then, we define $mex \{ x_1,x_2,\dots,x_n \} = min \{ \mathbb N^0 \setminus \{ x_1,x_2,\dots,x_n \} \}$ and $mex \{ \rightmoon ,  x_1,x_2,\dots,x_n \} = mex \{  x_1,x_2,\dots,x_n \}$, where $\rightmoon$ is the nim-value of a loony-move.\footnote[4]{I use $\mathbb N^0$ to mean $\mathbb N \cup \{ 0 \}$.}
\end{mydef}

\noindent
\emph{E.g.}
\begin{equation}
mex \{ 0,1,2 \} = 3,
\end{equation}
\begin{equation}
mex \{ \rightmoon ,0,2,3 \} = 1,
\end{equation}
\begin{equation}
mex \{ 1,2,4,5,7 \} = 0.
\end{equation}

\noindent
From here we can now define Nimdots subgame value.recursive

\begin{mydef}[Nimdots Subgame Value]
Let $G_1$ be any subgame with nim-values $x_1,x_2,\dots,x_n \in \mathbb N^0$. Then, the \textbf{Nimdots subgame value} of  $G_1$ is defined as \newline $NV(G_1)=mex \{ x_1,x_2,\dots,x_n \}$.  By convention, all Nimdots values are prefixed by $``\ast"$ except 0.
\end{mydef}

\noindent
Lastly, we define a nim-value as follows:

\begin{mydef}[Nim-value]
The \textbf{nim-value} of a move is the the Nimdots value of the the subgame resulting from playing that move.  That is, given a move $m_1$ in a subgame $G_1$ we write the nim-value of $m_1$ as: $nv(m_1) = NV(G^{m_1}_1)$, where $G^{m_1}_1$ is the subgame $G_1$ after move $m_1$ has been played.  Note, we define $nv(\mbox{loony-move}) = \rightmoon$.\footnote[5]{A loony-move in Nimdots is given the special nim-value of $\rightmoon$ because as soon as one is played it is an automatic win for the opponent.  This is due to the fact that there are no ties in Nimdots.  Thus, as soon as a player makes a loony-move his opponent can decide which side will win and play accordingly.}
\end{mydef}

By defining the nim-value of a move in the above fashion we have established a recursive process for finding the Nimdots value of any subgame.  Thus, we must now determine the terminal case in order for the recursion to be well defined.


Since the recursive step adds a move to a subgame with each iteration, we know that it will terminate when every move has been filled in (this is referred to as a completed subgame, and is denoted $G^\ast_1$ for any subgame $G_1$).  Therefore, in order to establish the terminal case we must assign a Nimdots value to a completed subgame.   By convention, we say that the $NV(G^\ast_1) = \ast 1$ for any subgame $G_1$.  We are now ready to introduce the desired recursive algorithm.


\begin{mydef}[Recursive Algorithm for Nimdots Subgame Values]
Let $G_1$ be any subgame with remaining moves $m_1,m_2,\dots,m_n$. Then, the recursive algorithm for computing the Nimdots value of $G_1$ is:

\begin{equation}
\begin{array}{lcl} 
NV(G^\ast_1) & = & \ast 1, \\
NV(G_1) & = & mex \{ NV(G^{m_1}_1),NV(G^{m_2}_1),\dots,NV(G^{m_n}_1) \}.
\end{array}
\end{equation}

\end{mydef}

\subsubsection{Examples of Finding Nimdots Subgame Values}

\underline{\smash{Example 1.}}

\noindent
Let us first look at the smallest possible subgame, the one box subgame, to get a feel for how the recursive algorithm works.

\begin{figure}[H]
\centering
\begin{tikzpicture}
	\begin{pgfonlayer}{nodelayer}
		\node [style=node] (0) at (0, -0) {};
		\node [style=node] (1) at (1, -0) {};
		\node [style=node] (2) at (0, -1) {};
		\node [style=node] (3) at (1, -1) {};
		\node (4) at (0, -0.5) {$m_1$};
		\node (5) at (0.5, -0) {$m_2$};
		\node (6) at (1, -0.5) {$m_3$};
		\node (7) at (0.5, -1) {$m_4$};
	\end{pgfonlayer}
\end{tikzpicture}
\caption{Subgame $G_1$ with moves $m_1$-$m_4$.}
\end{figure}

\noindent
In order to solve for the Nimdots value of $G_1$ we need to solve for the nim-value of each available move.  In this example, however, every move produces the same subgame, and thus all four moves have the same nim-value.  Therefore, it suffices to solve for the the nim-value of one of the moves.  Let that move be $m_1$.

\begin{figure}[H]
\centering
\begin{tikzpicture}
	\begin{pgfonlayer}{nodelayer}
		\node [style=node] (0) at (0, -0) {};
		\node [style=node] (1) at (1, -0) {};
		\node [style=node] (2) at (0, -1) {};
		\node [style=node] (3) at (1, -1) {};
		\node (4) at (0.5, -0) {$n_1$};
		\node (5) at (1, -0.5) {$n_2$};
		\node (6) at (0.5, -1) {$n_3$};
	\end{pgfonlayer}
	\begin{pgfonlayer}{edgelayer}
		\draw [style=simple] (0) to (2);
	\end{pgfonlayer}
\end{tikzpicture}
\caption{Subgame $G^{m_1}_1$ with moves $n_1$-$n_3$.}
\end{figure}

\noindent
After analyzing the subgame $G^{m_1}_1$, and the three possible moves $n_1$-$n_3$, we notice that again all three moves are equivalent.  With most subgame positions this will not be the case -- normally one must deduce every nim-value for each iteration.  But given the simplicity of the one box subgame it is not hard to see that this is true not only of subgames  $G_1$ and  $G^{m_1}_1$, but of every subgame in each step of the recursion.  Therefore, we get the following sequence of subgames by playing the first move in each step:

\begin{figure}[H]
\centering
\begin{tikzpicture}
	\begin{pgfonlayer}{nodelayer}
		\node [style=node] (0) at (0, -0) {};
		\node [style=node] (1) at (1, -0) {};
		\node [style=node] (2) at (0, -1) {};
		\node [style=node] (3) at (1, -1) {};
		\node [style=node] (4) at (3, -0) {};
		\node [style=node] (5) at (3, -1) {};
		\node [style=node] (6) at (4, -0) {};
		\node [style=node] (7) at (4, -1) {};
		\node [style=node] (8) at (6, -0) {};
		\node [style=node] (9) at (6, -1) {};
		\node [style=node] (10) at (7, -0) {};
		\node [style=node] (11) at (7, -1) {};
		\node [style=node] (12) at (9, -0) {};
		\node [style=node] (13) at (9, -1) {};
		\node [style=node] (14) at (10, -0) {};
		\node [style=node] (15) at (10, -1) {};
		\node (16) at (0.5, -0) {$n_1$};
		\node (17) at (1, -0.5) {$n_2$};
		\node (18) at (0.5, -1) {$n_3$};
		\node (19) at (4, -0.5) {$j_1$};
		\node (20) at (3.5, -1) {$j_2$};
		\node (21) at (6.5, -1) {$k_1$};
		\node (22) at (10.25, -1.25) {\textbf{$\ast$1}};
		\node (23) at (0.5, -2) {$G^{m_1}_1$};
		\node (24) at (3.5, -2) {$G^{m_1 n_1}_1$};
		\node (25) at (6.5, -2) {$G^{m_1 n_1 j_1}_1$};
		\node (26) at (9.5, -2) {$G^\ast_1$};
	\end{pgfonlayer}
	\begin{pgfonlayer}{edgelayer}
		\draw [style=simple] (0) to (2);
		\draw [style=simple] (4) to (5);
		\draw [style=simple] (4) to (6);
		\draw [style=simple] (8) to (9);
		\draw [style=simple] (8) to (10);
		\draw [style=simple] (10) to (11);
		\draw [style=simple] (13) to (12);
		\draw [style=simple] (12) to (14);
		\draw [style=simple] (14) to (15);
		\draw [style=simple] (15) to (13);
	\end{pgfonlayer}
\end{tikzpicture}
\caption{}
\end{figure}

\noindent
We can now work backwards from $G^\ast_1$ to fill in the nim-values and subsequent Nimdots subgame values.

\begin{figure}[H]
\centering
\begin{tikzpicture}
	\begin{pgfonlayer}{nodelayer}
		\node [style=node] (0) at (0, -0) {};
		\node [style=node] (1) at (1, -0) {};
		\node [style=node] (2) at (0, -1) {};
		\node [style=node] (3) at (1, -1) {};
		\node [style=node] (4) at (3, -0) {};
		\node [style=node] (5) at (3, -1) {};
		\node [style=node] (6) at (4, -0) {};
		\node [style=node] (7) at (4, -1) {};
		\node [style=node] (8) at (6, -0) {};
		\node [style=node] (9) at (6, -1) {};
		\node [style=node] (10) at (7, -0) {};
		\node [style=node] (11) at (7, -1) {};
		\node [style=node] (12) at (9, -0) {};
		\node [style=node] (13) at (9, -1) {};
		\node [style=node] (14) at (10, -0) {};
		\node [style=node] (15) at (10, -1) {};
		\node (16) at (0.5, -0) {1};
		\node (17) at (1, -0.5) {1};
		\node (18) at (0.5, -1) {1};
		\node (19) at (4, -0.5) {0};
		\node (20) at (3.5, -1) {0};
		\node (21) at (6.5, -1) {1};
		\node (22) at (10.25, -1.25) {\textbf{$\ast$1}};
		\node (23) at (0.5, -2) {$G^{m_1}_1$};
		\node (24) at (3.5, -2) {$G^{m_1 n_1}_1$};
		\node (25) at (6.5, -2) {$G^{m_1 n_1 j_1}_1$};
		\node (26) at (9.5, -2) {$G^\ast_1$};
		\node (27) at (1.25, -1.25) {\textbf{0}};
		\node (28) at (4.25, -1.25) {\textbf{$\ast$1}};
		\node (29) at (7.25, -1.25) {\textbf{0}};
	\end{pgfonlayer}
	\begin{pgfonlayer}{edgelayer}
		\draw [style=simple] (0) to (2);
		\draw [style=simple] (4) to (5);
		\draw [style=simple] (4) to (6);
		\draw [style=simple] (8) to (9);
		\draw [style=simple] (8) to (10);
		\draw [style=simple] (10) to (11);
		\draw [style=simple] (13) to (12);
		\draw [style=simple] (12) to (14);
		\draw [style=simple] (14) to (15);
		\draw [style=simple] (15) to (13);
	\end{pgfonlayer}
\end{tikzpicture}
\caption{}
\end{figure}

\noindent
Thus, we get: 

\begin{multicols}{2}
\begin{equation*}
\begin{array}{rcl}
NV(G_1) & = & mex \{ G^{m_1}_1 , G^{m_2}_1 , G^{m_3}_1 , G^{m_4}_1 \} \\
& = & mex \{ G^{m_1}_1 \} \\
& = & mex \{ 0 \} \\
& = & \ast 1 \\
\\
\\
\end{array}
\end{equation*}

\begin{figure}[H]
\centering
\begin{tikzpicture}
	\begin{pgfonlayer}{nodelayer}
		\node [style=node] (0) at (0, -0) {};
		\node [style=node] (1) at (1, -0) {};
		\node [style=node] (2) at (0, -1) {};
		\node [style=node] (3) at (1, -1) {};
		\node (4) at (0.5, -0) {0};
		\node (5) at (1, -0.5) {0};
		\node (6) at (0.5, -1) {0};
		\node (7) at (0, -0.5) {0};
		\node (8) at (1.25, -1.25) {\textbf{$\ast$1}};
	\end{pgfonlayer}
\end{tikzpicture}
\caption{Subgame $G_1$ solved.}
\end{figure}
\end{multicols}


\noindent
\underline{\smash{Example 2.}}

\noindent
Consider the subgame $G_1$ introduced earlier in Figure 4.1:

\begin{figure}[H]
\centering
\begin{tikzpicture}
	\begin{pgfonlayer}{nodelayer}
		\node [style=node] (0) at (0, -0) {};
		\node [style=node] (1) at (1, -0) {};
		\node [style=node] (2) at (2, -0) {};
		\node [style=node] (3) at (3, -0) {};
		\node [style=node] (4) at (0, -1) {};
		\node [style=node] (5) at (1, -1) {};
		\node [style=node] (6) at (2, -1) {};
		\node [style=node] (7) at (3, -1) {};
		\node (8) at (0, -0.5) {\emph{$m_1$}};
		\node (9) at (1, -0.5) {\emph{$m_2$}};
		\node (10) at (2, -0.5) {\emph{$m_3$}};
		\node (11) at (2.5, -0) {\emph{$m_4$}};
		\node (12) at (3, -0.5) {\emph{$m_5$}};
	\end{pgfonlayer}
	\begin{pgfonlayer}{edgelayer}
		\draw [style=simple] (0) to (2);
		\draw [style=simple] (4) to (7);
	\end{pgfonlayer}
\end{tikzpicture}
\caption{Subgame $G_1$ with moves $m_1$-$m_5$.}
\end{figure}

\noindent
Right away we see that two of the moves are loony-moves.  Thus, moves $m_1$ and $m_3$ can be labeled without calculating the Nimdots values of the resulting positions.  Therefore, we get the following: 

\begin{figure}[H]
\centering
\begin{tikzpicture}
	\begin{pgfonlayer}{nodelayer}
		\node [style=node] (0) at (0, -0) {};
		\node [style=node] (1) at (1, -0) {};
		\node [style=node] (2) at (2, -0) {};
		\node [style=node] (3) at (3, -0) {};
		\node [style=node] (4) at (0, -1) {};
		\node [style=node] (5) at (1, -1) {};
		\node [style=node] (6) at (2, -1) {};
		\node [style=node] (7) at (3, -1) {};
		\node (8) at (0, -0.5) {$\rightmoon$};
		\node (9) at (1, -0.5) {$m_2$};
		\node (10) at (2, -0.5) {$\rightmoon$};
		\node (11) at (2.5, -0) {$m_4$};
		\node (12) at (3, -0.5) {$m_5$};
	\end{pgfonlayer}
	\begin{pgfonlayer}{edgelayer}
		\draw [style=simple] (0) to (2);
		\draw [style=simple] (4) to (7);
	\end{pgfonlayer}
\end{tikzpicture}
\caption{Subgame $G_1$ with moves $m_2$, $m_4$, and $m_5$ unsolved.}
\end{figure}

\noindent
At this point we need to begin making moves and analyzing the resulting subgames.  We start with move $m_2$.  By making move $m_2$ we have created two boxes that are capturable with the first moves of the next turn.  Thus, according to Theorem 1 on accepting sacrifices, we know that the value of the position is unchanged by first capturing the two boxes.  Once the two boxes are captured we are left with a subgame that was solved in the previous example which had nim-values $\{ 0 , 0 \}$ and a Nimdots value of $\ast 1$.  Therefore, we know $nv(m_2) = NV(G^{m_2}_1) = 1$.

\begin{multicols}{2}
\begin{figure}[H]
\centering
\begin{tikzpicture}
	\begin{pgfonlayer}{nodelayer}
		\node [style=node] (0) at (0, -0) {};
		\node [style=node] (1) at (1, -0) {};
		\node [style=node] (2) at (2, -0) {};
		\node [style=node] (3) at (3, -0) {};
		\node [style=node] (4) at (0, -1) {};
		\node [style=node] (5) at (1, -1) {};
		\node [style=node] (6) at (2, -1) {};
		\node [style=node] (7) at (3, -1) {};
		\node (9) at (2.5, -0) {0};
		\node (10) at (3, -0.5) {0};
		\node (11) at (3.25, -1.25) {\textbf{$\ast$1}};
	\end{pgfonlayer}
	\begin{pgfonlayer}{edgelayer}
		\draw [style=simple] (0) to (2);
		\draw [style=simple] (4) to (7);
		\draw [style=simple] (1) to (5);
		\draw [style=simple] (0) to (4);
		\draw [style=simple] (2) to (6);
	\end{pgfonlayer}
\end{tikzpicture}
\caption{Subgame $G^{m_2}_1$.}
\end{figure}


\begin{figure}[H]
\centering
\begin{tikzpicture}
	\begin{pgfonlayer}{nodelayer}
		\node [style=node] (0) at (0, -0) {};
		\node [style=node] (1) at (1, -0) {};
		\node [style=node] (2) at (2, -0) {};
		\node [style=node] (3) at (3, -0) {};
		\node [style=node] (4) at (0, -1) {};
		\node [style=node] (5) at (1, -1) {};
		\node [style=node] (6) at (2, -1) {};
		\node [style=node] (7) at (3, -1) {};
		\node (8) at (0, -0.5) {$\rightmoon$};
		\node (9) at (1, -0.5) {1};
		\node (10) at (2, -0.5) {$\rightmoon$};
		\node (11) at (2.5, -0) {$m_4$};
		\node (12) at (3, -0.5) {$m_5$};
	\end{pgfonlayer}
	\begin{pgfonlayer}{edgelayer}
		\draw [style=simple] (0) to (2);
		\draw [style=simple] (4) to (7);
	\end{pgfonlayer}
\end{tikzpicture}
\caption{Subgame $G_1$ with moves $m_4$ and $m_5$ unsolved.}
\end{figure}
\end{multicols}

\noindent
Now, the only two moves left to consider are moves $m_4$ and $m_5$.  However, both moves create a long-chain, and any move in a long-chain is a loony-move.  So not only do we know that both moves are equivalent, we also know that the subgame after either move is played has a Nimdots value of $mex \{ \rightmoon , \rightmoon , \rightmoon , \rightmoon \} = 0$.  Thus, they each have a nim-value of 0.


\begin{figure}[H]
\centering
\begin{tikzpicture}
	\begin{pgfonlayer}{nodelayer}
		\node [style=node] (0) at (0, -0) {};
		\node [style=node] (1) at (1, -0) {};
		\node [style=node] (2) at (2, -0) {};
		\node [style=node] (3) at (3, -0) {};
		\node [style=node] (4) at (0, -1) {};
		\node [style=node] (5) at (1, -1) {};
		\node [style=node] (6) at (2, -1) {};
		\node [style=node] (7) at (3, -1) {};
		\node (8) at (2.25, -0) {};
		\node (9) at (3, -0.75) {};
		\node (10) at (0, -0.5) {$\rightmoon$};
		\node (11) at (1, -0.5) {$\rightmoon$};
		\node (12) at (2, -0.5) {$\rightmoon$};
		\node (13) at (2.75, -0.25) {$\rightmoon$};
		\node (14) at (3.25, -1.25) {\textbf{0}};
	\end{pgfonlayer}
	\begin{pgfonlayer}{edgelayer}
		\draw [style=simple] (0) to (2);
		\draw [style=simple] (4) to (7);
		\draw [style=yo] (8) to (3);
		\draw [style=yo] (3) to (9);
	\end{pgfonlayer}
\end{tikzpicture}
\caption{The result of playing either move $m_4$ or $m_5$.\protect\footnotemark[6]}
\end{figure}
\footnotetext[6]{When diagramming equivalent moves, I use a three quarter dashed line to mean either move but not both.  The nim-value of the the move not played is then placed in the in the respective corner.}

\noindent
We have now recursively solved for all the nim-values in subgame $G_1$, and can determine $NV(G_1)$.  Thus, $NV(G_1) = mex \{ \rightmoon , 1 , \rightmoon , 0 , 0 \} = \ast 2$.

\begin{figure}[H]
\centering
\begin{tikzpicture}
	\begin{pgfonlayer}{nodelayer}
		\node [style=node] (0) at (0, -0) {};
		\node [style=node] (1) at (1, -0) {};
		\node [style=node] (2) at (2, -0) {};
		\node [style=node] (3) at (3, -0) {};
		\node [style=node] (4) at (0, -1) {};
		\node [style=node] (5) at (1, -1) {};
		\node [style=node] (6) at (2, -1) {};
		\node [style=node] (7) at (3, -1) {};
		\node (8) at (0, -0.5) {$\rightmoon$};
		\node (9) at (1, -0.5) {1};
		\node (10) at (2, -0.5) {$\rightmoon$};
		\node (11) at (2.5, -0) {0};
		\node (12) at (3, -0.5) {0};
		\node (13) at (3.25, -1.25) {\textbf{$\ast$2}};
	\end{pgfonlayer}
	\begin{pgfonlayer}{edgelayer}
		\draw [style=simple] (0) to (2);
		\draw [style=simple] (4) to (7);
	\end{pgfonlayer}
\end{tikzpicture}
\caption{Subgame $G_1$ solved.}
\end{figure}


\subsubsection{Consequences of Solving Nimdots Subgames}

\begin{theorem}
If a move has a nim-value of 0, then either it is the move preceding the losing move or  
\end{theorem}

\begin{proof}

\end{proof}

\begin{theorem}
Let $G_1$ be any subgame.  Then, the player who is next to move can win the subgame $G_1$ if and only if $NV(G_1)>0$.
\end{theorem}

\begin{proof}
Let $G_1$ be a subgame, let $NV(G_1)>0$, and let $\mathcal{V}$ be the set of of all nim-values in $G_1$.  Assume it is Player A to move.  Then, in order for $NV(G_1)>0$ we know $0 \in \mathcal{V}$, given $NV(G_1) = min \{ \mathbb N^0 \setminus \mathcal{V} \}$.  Therefore, since $\mathcal{V}$ is the set of all nim-values there must be a move in $G_1$ with a nim-value of 0, i.e. there must be a winning move for Player A.  Now assume $NV(G_1)=0$ with Player A still to move.  Then, we know $0 \notin \mathcal{V}$ by the same line of reasoning.  But if Player A can't make a winning move, then he must make a move with a nim-value greater than 0.  This implies that the resulting subgame will have a Nimdots value greater than 0.  Therefore, there exists a winning move for Player B.
\end{proof}


\subsection{Solving Composite Games}
Now that we can solve any Nimdots subgame, we turn to a description of how to solve composite games.  To do this, we need an operation known as the bitwise exclusive-or operation.

\begin{mydef}[Bitwise Exclusive-or Operation]
The \textbf{Bitwise Exclusive-or Operation}, denoted $\oplus$, is defined as binary addition without carries.
\end{mydef}

\noindent
\emph{E.g.}
\begin{equation}
\begin{array}{rlcl} 
& 010_2 & = & 2_{10} \\
\oplus & 011_2 & = & 3_{10} \\
& 111_2 & = & 7_{10} \\
\hline
& 110_2 & = & 6_{10}
\end{array}
\end{equation}



\begin{mydef}[Nimdots Composite Game Value]
Let $G$ be a composite game such that $G = G_1 + G_2 + \cdots + G_n$.  Then $NV(G) = G_1 \oplus G_2 \oplus \cdots \oplus G_n$.
\end{mydef}

Let us consider the game $G$ in Figure 4.1 on page 42 to see how this works.  In the last section we solved for the Nimdots value of the first subgame, $G_1$, and concluded that the $NV(G_1) = \ast 2$.  Therefore, we need only solve the two remaining subgames.  Given subgame $G_2$ is a long-chain, we know that every move is a loony-move, and thus $NV(G_2) = 0$.  To solve subgame $G_3$ however, we need to employ the recursive algorithm for solving subgames.

\begin{multicols}{2}
\begin{figure}[H]
\centering
\begin{tikzpicture}
	\begin{pgfonlayer}{nodelayer}
		\node [style=node] (0) at (0, -0) {};
		\node [style=node] (1) at (1, -0) {};
		\node [style=node] (2) at (2, -0) {};
		\node [style=node] (3) at (0, -1) {};
		\node [style=node] (4) at (1, -1) {};
		\node [style=node] (5) at (2, -1) {};
		\node (6) at (0.5, -1) {$m_1$};
		\node (7) at (1, -0.5) {$m_2$};
		\node (8) at (2, -0.5) {$m_3$};
	\end{pgfonlayer}
	\begin{pgfonlayer}{edgelayer}
		\draw [style=simple] (3) to (0);
		\draw [style=simple] (0) to (2);
		\draw [style=simple] (4) to (5);
	\end{pgfonlayer}
\end{tikzpicture}
\caption{Subgame $G_3$ unsolved.}
\end{figure}

\begin{figure}[H]
\centering
\begin{tikzpicture}
	\begin{pgfonlayer}{nodelayer}
		\node [style=node] (0) at (0, -0) {};
		\node [style=node] (1) at (1, -0) {};
		\node [style=node] (2) at (2, -0) {};
		\node [style=node] (3) at (0, -1) {};
		\node [style=node] (4) at (1, -1) {};
		\node [style=node] (5) at (2, -1) {};
		\node (6) at (0.5, -1) {\rightmoon};
		\node (7) at (1, -0.5) {0};
		\node (8) at (2, -0.5) {\rightmoon};
		\node (9) at (2.25, -1.25) {\textbf{$\ast$1}};
	\end{pgfonlayer}
	\begin{pgfonlayer}{edgelayer}
		\draw [style=simple] (3) to (0);
		\draw [style=simple] (0) to (2);
		\draw [style=simple] (4) to (5);
	\end{pgfonlayer}
\end{tikzpicture}
\caption{Subgame $G_3$ solved.}
\end{figure}
\end{multicols}

Looking at the three possible moves it is easy to see that moves $m_1$ and $m_3$ are loony-moves, and move $m_2$ leads to a win.  Therefore, we get $NV(G_3) = mex \{ \rightmoon , 0 , \rightmoon \} = \ast 1$.  Thus, we have solved all three subgames and can now determine the Nimdots value of the game $G$.

\begin{multicols}{2}
\begin{equation*}
\begin{array}{rlcrcl} 
& 10_2 & = & \ast 2_{10} & = & NV(G_1) \\
\oplus & 00_2 & = & 0_{10} & = & NV(G_2) \\
& 01_2 & = & \ast 1_{10} & = & NV(G_3) \\
\hline
& 11_2 & = & \ast 3_{10} & = & NV(G) \\
\\
\\
\\
\end{array}
\end{equation*}

\begin{figure}[H]
\centering
\begin{tikzpicture}
	\begin{pgfonlayer}{nodelayer}
		\node [style=node] (0) at (0, -0) {};
		\node [style=node] (1) at (1, -0) {};
		\node [style=node] (2) at (2, -0) {};
		\node [style=node] (3) at (3, -0) {};
		\node [style=node] (4) at (0, -1) {};
		\node [style=node] (5) at (1, -1) {};
		\node [style=node] (6) at (2, -1) {};
		\node [style=node] (7) at (3, -1) {};
		\node [style=node] (8) at (0, -2) {};
		\node [style=node] (9) at (1, -2) {};
		\node [style=node] (10) at (2, -2) {};
		\node [style=node] (11) at (3, -2) {};
		\node [style=node] (12) at (0, -3) {};
		\node [style=node] (13) at (1, -3) {};
		\node [style=node] (14) at (2, -3) {};
		\node [style=node] (15) at (3, -3) {};
		\node (16) at (0.5, -1.5) {$\ast$2};
		\node (17) at (2, -0.5) {0};
		\node (18) at (2, -2.5) {$\ast$1};
		\node (19) at (3.25, -3.25) {\textbf{$\ast$3}};
	\end{pgfonlayer}
	\begin{pgfonlayer}{edgelayer}
		\draw [style=simple] (4) to (12);
		\draw [style=simple] (1) to (13);
		\draw [style=simple] (9) to (11);
		\draw [style=simple] (14) to (15);
		\draw [style=simple] (6) to (7);
		\draw [style=simple] (1) to (3);
	\end{pgfonlayer}
\end{tikzpicture}
\caption{}
\end{figure}
\end{multicols}

\subsubsection{Finding the Winning Move}
Given the Nimdots value of game $G$ does not equal 0, we know that there exists a winning move for the player to move next -- Player B.  As with subgames, Player B must make the Nimdots value of the game 0 by playing a single move.  However, with composite games this is not necessarily done by playing a move with a nim-value of 0.  In this case, we see that in order to make $NV(G) = 0$, either $NV(G_1)$ must be $\ast$1 or $NV(G_3)$ must be $\ast$2.  But since there is no move in subgame $G_3$ with a nim-value of 2, we can conclude that Player B must make a move in subgame $G_1$ with a nim-value of 1.  


\section{Nimdots Method}
The Nimdots Method for acquiring control relies on the fact that every Dots and Boxes game is comprised of subgames.  At some point in every Dots and Boxes game the board is partitioned into subgames of long-chains or cycles, and subgames that are not long-chains or cycles.  In the simplest example, the board is partitioned into subgames that are all long-chains or cycles except one.  In this case, the player who wins the Nimdots game in the one subgame that isn't a long-chain or cycle has clearly forced their opponent to move in one of the remaining subgames, and thus has won control.  When more than one subgame is not a long-chain or cycle however, we must consider the Nimdots game that is sum of all such subgames, and make conclusions about control based on the winner of that game.  This is fully explored later in the chapter, but for now, consider the following Dots and Boxes game to better understand how the Nimdots Method works:

\begin{figure}[H]
\centering
\begin{tikzpicture}
	\begin{pgfonlayer}{nodelayer}
		\node [style=node] (0) at (0, -0) {};
		\node [style=node] (1) at (1, -0) {};
		\node [style=node] (2) at (3, -0) {};
		\node [style=node] (3) at (4, -0) {};
		\node [style=node] (4) at (0, -1) {};
		\node [style=node] (5) at (1, -1) {};
		\node [style=node] (6) at (2, -1) {};
		\node [style=node] (7) at (3, -1) {};
		\node [style=node] (8) at (4, -1) {};
		\node [style=node] (9) at (0, -2) {};
		\node [style=node] (10) at (1, -2) {};
		\node [style=node] (11) at (2, -2) {};
		\node [style=node] (12) at (3, -2) {};
		\node [style=node] (13) at (4, -2) {};
		\node [style=node] (14) at (0, -3) {};
		\node [style=node] (15) at (1, -3) {};
		\node [style=node] (16) at (2, -3) {};
		\node [style=node] (17) at (3, -3) {};
		\node [style=node] (18) at (4, -3) {};
		\node [style=node] (19) at (0, -4) {};
		\node [style=node] (20) at (1, -4) {};
		\node [style=node] (21) at (2, -4) {};
		\node [style=node] (22) at (3, -4) {};
		\node [style=node] (23) at (4, -4) {};
		\node [style=node] (24) at (2, -0) {};
	\end{pgfonlayer}
	\begin{pgfonlayer}{edgelayer}
		\draw [style=dot] (10) to (11);
		\draw [style=dot] (12) to (13);
		\draw [style=dot] (16) to (17);
		\draw [style=dot] (9) to (14);
		\draw [style=dot] (19) to (20);
		\draw [style=dot] (21) to (22);
		\draw [style=dash] (6) to (11);
		\draw [style=dash] (11) to (12);
		\draw [style=dash] (9) to (10);
		\draw [style=dash] (14) to (19);
		\draw [style=dash] (20) to (21);
		\draw [style=dash] (15) to (16);
		\draw [style=dash] (17) to (18);
		\draw [style=dash] (22) to (23);
		\draw [style=dot] (24) to (6);
		\draw [style=dash] (1) to (24);
		\draw [style=dot] (0) to (1);
		\draw [style=bpm] (4) to (5);
	\end{pgfonlayer}
\end{tikzpicture}
\caption{A game $G$. Player A to move.}
\end{figure}

\begin{figure}[H]
\centering
\begin{tikzpicture}
	\begin{pgfonlayer}{nodelayer}
		\node [style=node] (0) at (0, -0) {};
		\node [style=node] (1) at (1, -0) {};
		\node [style=node] (2) at (2, -0) {};
		\node [style=node] (3) at (0, -1) {};
		\node [style=node] (4) at (1, -1) {};
		\node [style=node] (5) at (2, -1) {};
		\node [style=node] (6) at (0, -2) {};
		\node [style=node] (7) at (1, -2) {};
		\node [style=node] (8) at (2, -2) {};
		\node [style=node] (9) at (4, -0) {};
		\node [style=node] (10) at (5, -0) {};
		\node [style=node] (11) at (6, -0) {};
		\node [style=node] (12) at (4, -1) {};
		\node [style=node] (13) at (5, -1) {};
		\node [style=node] (14) at (6, -1) {};
		\node [style=node] (15) at (4, -2) {};
		\node [style=node] (16) at (5, -2) {};
		\node [style=node] (17) at (6, -2) {};
		\node [style=node] (18) at (8, -0) {};
		\node [style=node] (19) at (8, -1) {};
		\node [style=node] (20) at (8, -2) {};
		\node [style=node] (21) at (9, -2) {};
		\node [style=node] (22) at (10, -2) {};
		\node [style=node] (23) at (11, -2) {};
		\node [style=node] (24) at (12, -2) {};
		\node [style=node] (25) at (9, -1) {};
		\node [style=node] (26) at (9, -0) {};
		\node [style=node] (27) at (10, -1) {};
		\node [style=node] (28) at (10, -0) {};
		\node [style=node] (29) at (11, -1) {};
		\node [style=node] (30) at (11, -0) {};
		\node [style=node] (31) at (12, -0) {};
		\node [style=node] (32) at (12, -1) {};
		\node (33) at (1, -2.75) {$G_1$};
		\node (34) at (5, -2.75) {$G_2$};
		\node (35) at (10, -2.75) {$G_3$};
	\end{pgfonlayer}
	\begin{pgfonlayer}{edgelayer}
		\draw [style=dot] (0) to (1);
		\draw [style=dot] (2) to (5);
		\draw [style=dot] (7) to (8);
		\draw [style=bpm] (3) to (4);
		\draw [style=dash] (1) to (2);
		\draw [style=dash] (5) to (8);
		\draw [style=dash] (6) to (7);
		\draw [style=dash] (12) to (15);
		\draw [style=dash] (15) to (16);
		\draw [style=dot] (16) to (17);
		\draw [style=dot] (9) to (12);
		\draw [style=dot] (26) to (28);
		\draw [style=dot] (30) to (31);
		\draw [style=dot] (27) to (29);
		\draw [style=dot] (18) to (19);
		\draw [style=dot] (20) to (21);
		\draw [style=dot] (22) to (23);
		\draw [style=dash] (18) to (26);
		\draw [style=dash] (28) to (30);
		\draw [style=dash] (25) to (27);
		\draw [style=dash] (29) to (32);
		\draw [style=dash] (23) to (24);
		\draw [style=dash] (21) to (22);
		\draw [style=dash] (19) to (20);
	\end{pgfonlayer}
\end{tikzpicture}
\caption{Subgames of $G$.}
\end{figure}

Figure 4.2 shows a game in which there are three subgames, of which only $G_2$ is not a long-chain.  Thus, the player who wins the Nimdots game in the subgame $G_2$ will force their opponent to move in either $G_1$ or $G_3$ -- remember that if you win a Nimdots game then your opponent captured the last box.  But both $G_1$ and $G_3$ are long-chains.  Therefore, winning the Nimdots game in $G_2$ implies that you have gained control.

Let us look at another example.  This time, we consider a game $H$ comprised of three subgames where two of the subgames are not long-chains or cycles.

\begin{figure}[H]
\centering
\begin{tikzpicture}
	\begin{pgfonlayer}{nodelayer}
		\node [style=node] (0) at (2, -0) {};
		\node [style=node] (1) at (2, -1) {};
		\node [style=node] (2) at (0, -2) {};
		\node [style=node] (3) at (1, -2) {};
		\node [style=node] (4) at (2, -2) {};
		\node [style=node] (5) at (2, -0) {};
		\node [style=node] (6) at (3, -0) {};
		\node [style=node] (7) at (4, -0) {};
		\node [style=node] (8) at (2, -1) {};
		\node [style=node] (9) at (3, -1) {};
		\node [style=node] (10) at (4, -1) {};
		\node [style=node] (11) at (2, -2) {};
		\node [style=node] (12) at (3, -2) {};
		\node [style=node] (13) at (4, -2) {};
		\node [style=node] (14) at (0, -2) {};
		\node [style=node] (15) at (0, -3) {};
		\node [style=node] (16) at (0, -4) {};
		\node [style=node] (17) at (1, -4) {};
		\node [style=node] (18) at (2, -4) {};
		\node [style=node] (19) at (3, -4) {};
		\node [style=node] (20) at (4, -4) {};
		\node [style=node] (21) at (1, -3) {};
		\node [style=node] (22) at (1, -2) {};
		\node [style=node] (23) at (2, -3) {};
		\node [style=node] (24) at (2, -2) {};
		\node [style=node] (25) at (3, -3) {};
		\node [style=node] (26) at (3, -2) {};
		\node [style=node] (27) at (4, -2) {};
		\node [style=node] (28) at (4, -3) {};
		\node [style=node] (29) at (0, -1) {};
		\node [style=node] (30) at (1, -1) {};
		\node [style=node] (31) at (1, -0) {};
		\node [style=node] (32) at (0, -0) {};
	\end{pgfonlayer}
	\begin{pgfonlayer}{edgelayer}
		\draw [style=dot] (0) to (1);
		\draw [style=dot] (3) to (4);
		\draw [style=dash] (1) to (4);
		\draw [style=dash] (2) to (3);
		\draw [style=dash] (8) to (11);
		\draw [style=dash] (11) to (12);
		\draw [style=dot] (12) to (13);
		\draw [style=dot] (5) to (8);
		\draw [style=dot] (22) to (24);
		\draw [style=dot] (26) to (27);
		\draw [style=dot] (23) to (25);
		\draw [style=dot] (14) to (15);
		\draw [style=dot] (16) to (17);
		\draw [style=dot] (18) to (19);
		\draw [style=dash] (14) to (22);
		\draw [style=dash] (24) to (26);
		\draw [style=dash] (21) to (23);
		\draw [style=dash] (25) to (28);
		\draw [style=dash] (19) to (20);
		\draw [style=dash] (17) to (18);
		\draw [style=apm] (15) to (16);
	\end{pgfonlayer}
\end{tikzpicture}
\caption{A game $H$. Player B to move.}
\end{figure}

\begin{figure}[H]
\centering
\begin{tikzpicture}
	\begin{pgfonlayer}{nodelayer}
		\node [style=node] (0) at (2, -0) {};
		\node [style=node] (1) at (2, -1) {};
		\node [style=node] (2) at (0, -2) {};
		\node [style=node] (3) at (1, -2) {};
		\node [style=node] (4) at (2, -2) {};
		\node [style=node] (5) at (4, -0) {};
		\node [style=node] (6) at (5, -0) {};
		\node [style=node] (7) at (6, -0) {};
		\node [style=node] (8) at (4, -1) {};
		\node [style=node] (9) at (5, -1) {};
		\node [style=node] (10) at (6, -1) {};
		\node [style=node] (11) at (4, -2) {};
		\node [style=node] (12) at (5, -2) {};
		\node [style=node] (13) at (6, -2) {};
		\node [style=node] (14) at (8, -0) {};
		\node [style=node] (15) at (8, -1) {};
		\node [style=node] (16) at (8, -2) {};
		\node [style=node] (17) at (9, -2) {};
		\node [style=node] (18) at (10, -2) {};
		\node [style=node] (19) at (11, -2) {};
		\node [style=node] (20) at (12, -2) {};
		\node [style=node] (21) at (9, -1) {};
		\node [style=node] (22) at (9, -0) {};
		\node [style=node] (23) at (10, -1) {};
		\node [style=node] (24) at (10, -0) {};
		\node [style=node] (25) at (11, -1) {};
		\node [style=node] (26) at (11, -0) {};
		\node [style=node] (27) at (12, -0) {};
		\node [style=node] (28) at (12, -1) {};
		\node (29) at (1, -2.75) {$H_1$};
		\node (30) at (5, -2.75) {$H_2$};
		\node (31) at (10, -2.75) {$H_3$};
		\node [style=node] (32) at (0, -1) {};
		\node [style=node] (33) at (1, -1) {};
		\node [style=node] (34) at (1, -0) {};
		\node [style=node] (35) at (0, -0) {};
	\end{pgfonlayer}
	\begin{pgfonlayer}{edgelayer}
		\draw [style=dot] (0) to (1);
		\draw [style=dot] (3) to (4);
		\draw [style=dash] (1) to (4);
		\draw [style=dash] (2) to (3);
		\draw [style=dash] (8) to (11);
		\draw [style=dash] (11) to (12);
		\draw [style=dot] (12) to (13);
		\draw [style=dot] (5) to (8);
		\draw [style=dot] (22) to (24);
		\draw [style=dot] (26) to (27);
		\draw [style=dot] (23) to (25);
		\draw [style=dot] (14) to (15);
		\draw [style=dot] (16) to (17);
		\draw [style=dot] (18) to (19);
		\draw [style=dash] (14) to (22);
		\draw [style=dash] (24) to (26);
		\draw [style=dash] (21) to (23);
		\draw [style=dash] (25) to (28);
		\draw [style=dash] (19) to (20);
		\draw [style=dash] (17) to (18);
		\draw [style=apm] (15) to (16);
	\end{pgfonlayer}
\end{tikzpicture}
\caption{Subgames of $H$.}
\end{figure}

It is clear that in this example, a player must win the Nimdots game in the top half of the board in order to gain control.  That is, a player must win the Nimdots game resulting from summing the two subgames that are not long-chains, $H_1$ and $H_2$ -- call it $H'$.  By winning the Nimdots game $H'$ a player has guaranteed that his opponent will be the first to move in $H_3$, and thus has secured control in the final phase of game $H$.

In general, this gives the following theorem which I will call the Nimdots Method Theorem:

\begin{theorem}[The Nimdots Method Theorem]
Assume a Dots and Boxes game $G$ is comprised of subgames $G_1$ through $G_n$, where subgames $G_1$ through $G_k$ are long-chains or cycles and $G_{k+1}$ through $G_n$ are not, for $1 \leq k < n$.  Then, if a player wins the Nimdots game $G'=G_{k+1}+\cdots+G_n$, he will acquire control in game $G$.
\end{theorem}

\begin{proof}
Let $G$ be a Dots and Boxes game constructed in the above fashion.  Without loss of generality assume that Player A wins the Nimdots game $G'$.  Then, winning the Nimdots game $G'$ implies that Player B has captured the last box in $G'$ and must make a final move in the game $G''=G_1+\cdots+G_k$.  But, $G''$ is comprised of subgames that are all either long-chains or cycles.  Therefore, Player B will be the first to open a long-chain or cycle on a board with only long-chains or cycles left giving Player A control.
\end{proof}

You may have noticed that so far all we have done is describe a method for determining who will have control once a long-chain or cycle has been created, and nothing about how to actually secure control has a player.  However, these two ideas are really one in the same once we consider that every Nimdots game is solvable.  That is, in order to guarantee you will have control in a game, all you need to do is make sure that before the first long-chain or cycle is created the remaining Nimdots subgame(s) are winnable.  


\newpage
\section{Limitations of the Nimdots Method}




\begin{figure}[H]
\centering
\begin{tikzpicture}
	\begin{pgfonlayer}{nodelayer}
		\node [style=node] (0) at (0.5, -0) {};
		\node [style=node] (1) at (1.5, -0) {};
		\node [style=node] (2) at (1.5, -1) {};
		\node [style=node] (3) at (0.5, -1) {};
		\node [style=node] (4) at (0, -2.5) {};
		\node [style=node] (5) at (1, -2.5) {};
		\node [style=node] (6) at (1, -3.5) {};
		\node [style=node] (7) at (0, -3.5) {};
		\node [style=node] (8) at (0, -4.5) {};
		\node [style=node] (9) at (1, -4.5) {};
		\node [style=node] (10) at (2, -4.5) {};
		\node [style=node] (11) at (2, -3.5) {};
		\node [style=node] (12) at (0, -6) {};
		\node [style=node] (13) at (1, -6) {};
		\node [style=node] (14) at (0, -7) {};
		\node [style=node] (15) at (0, -8) {};
		\node [style=node] (16) at (1, -8) {};
		\node [style=node] (17) at (1, -7) {};
		\node [style=node] (18) at (2, -7) {};
		\node [style=node] (19) at (2, -8) {};
		\node [style=node] (20) at (4, -0) {};
		\node [style=node] (21) at (4, -1) {};
		\node [style=node] (22) at (5, -0) {};
		\node [style=node] (23) at (5, -1) {};
		\node [style=node] (24) at (6, -0) {};
		\node [style=node] (25) at (6, -1) {};
		\node [style=node] (26) at (8, -0) {};
		\node [style=node] (27) at (8, -1) {};
		\node [style=node] (28) at (9, -0) {};
		\node [style=node] (29) at (9, -1) {};
		\node [style=node] (30) at (10, -0) {};
		\node [style=node] (31) at (10, -1) {};
		\node [style=node] (32) at (4, -2.5) {};
		\node [style=node] (33) at (4, -3.5) {};
		\node [style=node] (34) at (4, -4.5) {};
		\node [style=node] (35) at (5, -4.5) {};
		\node [style=node] (36) at (5, -3.5) {};
		\node [style=node] (37) at (5, -2.5) {};
		\node [style=node] (38) at (6, -3.5) {};
		\node [style=node] (39) at (6, -4.5) {};
		\node [style=node] (40) at (12, -0) {};
		\node [style=node] (41) at (13, -0) {};
		\node [style=node] (42) at (14, -0) {};
		\node [style=node] (43) at (14, -1) {};
		\node [style=node] (44) at (13, -1) {};
		\node [style=node] (45) at (12, -1) {};
		\node [style=node] (46) at (12, -2.5) {};
		\node [style=node] (47) at (13, -2.5) {};
		\node [style=node] (48) at (13, -3.5) {};
		\node [style=node] (49) at (12, -3.5) {};
		\node [style=node] (50) at (12, -4.5) {};
		\node [style=node] (51) at (13, -4.5) {};
		\node [style=node] (52) at (14, -3.5) {};
		\node [style=node] (53) at (14, -4.5) {};
		\node [style=node] (54) at (8, -2.5) {};
		\node [style=node] (55) at (8, -3.5) {};
		\node [style=node] (56) at (8, -4.5) {};
		\node [style=node] (57) at (9, -4.5) {};
		\node [style=node] (58) at (10, -4.5) {};
		\node [style=node] (59) at (10, -3.5) {};
		\node [style=node] (60) at (9, -3.5) {};
		\node [style=node] (61) at (9, -2.5) {};
		\node [style=node] (62) at (4, -6) {};
		\node [style=node] (63) at (4, -7) {};
		\node [style=node] (64) at (4, -8) {};
		\node [style=node] (65) at (5, -7) {};
		\node [style=node] (66) at (5, -6) {};
		\node [style=node] (67) at (6, -7) {};
		\node [style=node] (68) at (6, -8) {};
		\node [style=node] (69) at (5, -8) {};
		\node [style=node] (70) at (8, -6) {};
		\node [style=node] (71) at (8, -7) {};
		\node [style=node] (72) at (8, -8) {};
		\node [style=node] (73) at (9, -7) {};
		\node [style=node] (74) at (9, -6) {};
		\node [style=node] (75) at (9, -8) {};
		\node [style=node] (76) at (10, -7) {};
		\node [style=node] (77) at (10, -8) {};
		\node [style=node] (78) at (12, -6) {};
		\node [style=node] (79) at (12, -7) {};
		\node [style=node] (80) at (12, -8) {};
		\node [style=node] (81) at (13, -8) {};
		\node [style=node] (82) at (14, -8) {};
		\node [style=node] (83) at (14, -7) {};
		\node [style=node] (84) at (13, -7) {};
		\node [style=node] (85) at (13, -6) {};
		\node (86) at (4, -0.25) {};
		\node (87) at (4.75, -1) {};
		\node (88) at (5.25, -1) {};
		\node (89) at (6, -0.25) {};
		\node (90) at (9.25, -1) {};
		\node (91) at (10, -0.25) {};
		\node (92) at (0, -3.25) {};
		\node (93) at (0.75, -2.5) {};
		\node (94) at (1.25, -4.5) {};
		\node (95) at (2, -3.75) {};
		\node (96) at (4, -3.25) {};
		\node (97) at (4.75, -2.5) {};
		\node (98) at (8, -3.75) {};
		\node (99) at (8.75, -4.5) {};
		\node (100) at (8, -3.25) {};
		\node (101) at (8.75, -2.5) {};
		\node (102) at (9.25, -4.5) {};
		\node (103) at (10, -3.75) {};
		\node (104) at (0, -6.75) {};
		\node (105) at (0.75, -6) {};
		\node (106) at (0, -7.25) {};
		\node (107) at (0.75, -8) {};
		\node (108) at (5.25, -8) {};
		\node (109) at (6, -7.25) {};
		\node (110) at (4, -6.75) {};
		\node (111) at (4.75, -6) {};
		\node (112) at (8, -6.75) {};
		\node (113) at (8.75, -6) {};
		\node (114) at (12, -7.25) {};
		\node (115) at (12.75, -8) {};
		\node (116) at (4, -0.25) {};
		\node (117) at (0.5, -0.5) {0};
		\node (118) at (1, -1) {0};
		\node (119) at (1.75, -1.25) {\textbf{$\ast$1}};
		\node (120) at (4.25, -0.75) {$\rightmoon$};
		\node (121) at (5, -0.5) {0};
		\node (122) at (5.75, -0.75) {$\rightmoon$};
		\node (123) at (6.25, -1.25) {\textbf{$\ast$1}};
		\node (124) at (8, -0.5) {1};
		\node (125) at (8.5, -1) {1};
		\node (126) at (9, -0.5) {1};
		\node (127) at (9.75, -0.75) {1};
		\node (128) at (10.25, -1.25) {\textbf{0}};
		\node (129) at (12, -0.5) {0};
		\node (130) at (12.5, -1) {0};
		\node (131) at (13, -0.5) {0};
		\node (132) at (13.5, -1) {0};
		\node (133) at (14, -0.5) {0};
		\node (134) at (14.25, -1.25) {\textbf{$\ast$1}};
		\node (135) at (0.25, -2.75) {$\rightmoon$};
		\node (136) at (0.5, -3.5) {$\rightmoon$};
		\node (137) at (1, -4) {$\rightmoon$};
		\node (138) at (1.75, -4.25) {$\rightmoon$};
		\node (139) at (2.25, -4.75) {\textbf{0}};
		\node (140) at (4.25, -2.75) {$\rightmoon$};
		\node (141) at (4.5, -3.5) {1};
		\node (142) at (5, -4) {$\rightmoon$};
		\node (143) at (5.5, -4.5) {0};
		\node (144) at (6, -4) {0};
		\node (145) at (6.25, -4.75) {\textbf{$\ast$2}};
		\node (146) at (8.25, -2.75) {1};
		\node (147) at (8.5, -3.5) {1};
		\node (148) at (8.25, -4.25) {0};
		\node (149) at (9, -4) {1};
		\node (150) at (9.75, -4.25) {1};
		\node (151) at (10.25, -4.75) {\textbf{$\ast$2}};
		\node (152) at (12, -3) {2};
		\node (153) at (12.5, -2.5) {2};
		\node (154) at (12.5, -3.5) {0};
		\node (155) at (13, -4) {0};
		\node (156) at (13.5, -4.5) {2};
		\node (157) at (14, -4) {2};
		\node (158) at (14.25, -4.75) {\textbf{$\ast$1}};
		\node (159) at (0.25, -6.25) {0};
		\node (160) at (0.5, -7) {0};
		\node (161) at (0.25, -7.75) {2};
		\node (162) at (1, -7.5) {0};
		\node (163) at (1.5, -8) {2};
		\node (164) at (2, -7.5) {2};
		\node (165) at (2.25, -8.25) {\textbf{$\ast$1}};
		\node (166) at (4.25, -6.25) {0};
		\node (167) at (4.5, -7) {0};
		\node (168) at (4, -7.5) {2};
		\node (169) at (4.5, -8) {2};
		\node (170) at (5, -7.5) {0};
		\node (171) at (5.75, -7.75) {0};
		\node (172) at (6.25, -8.25) {\textbf{$\ast$1}};
		\node (173) at (8.25, -6.25) {1};
		\node (174) at (8.5, -7) {1};
		\node (175) at (8, -7.5) {1};
		\node (176) at (8.5, -8) {1};
		\node (177) at (9, -7.5) {1};
		\node (178) at (9.5, -8) {1};
		\node (179) at (10, -7.5) {1};
		\node (180) at (10.25, -8.25) {\textbf{0}};
		\node (181) at (12, -6.5) {1};
		\node (182) at (12.5, -6) {1};
		\node (183) at (12.5, -7) {1};
		\node (184) at (12.25, -7.75) {1};
		\node (185) at (13, -7.5) {1};
		\node (186) at (13.5, -8) {1};
		\node (187) at (14, -7.5) {1};
		\node (188) at (14.25, -8.25) {\textbf{0}};
		\node (189) at (1, -1.5) {(a)};
		\node (190) at (5, -1.5) {(b)};
		\node (191) at (9, -1.5) {(c)};
		\node (192) at (13, -1.5) {(d)};
		\node (193) at (1, -5) {(e)};
		\node (194) at (5, -5) {(f)};
		\node (195) at (9, -5) {(g)};
		\node (196) at (13, -5) {(h)};
		\node (197) at (1, -8.5) {(i)};
		\node (198) at (5, -8.5) {(j)};
		\node (199) at (9, -8.5) {(k)};
		\node (200) at (13, -8.5) {(l)};
	\end{pgfonlayer}
	\begin{pgfonlayer}{edgelayer}
		\draw [style=simple] (0) to (1);
		\draw [style=simple] (1) to (2);
		\draw [style=simple] (20) to (24);
		\draw [style=simple] (26) to (30);
		\draw [style=simple] (40) to (42);
		\draw [style=simple] (7) to (8);
		\draw [style=simple] (8) to (9);
		\draw [style=simple] (5) to (6);
		\draw [style=simple] (6) to (11);
		\draw [style=simple] (33) to (34);
		\draw [style=simple] (34) to (35);
		\draw [style=simple] (38) to (36);
		\draw [style=simple] (37) to (36);
		\draw [style=simple] (61) to (60);
		\draw [style=simple] (60) to (59);
		\draw [style=simple] (49) to (50);
		\draw [style=simple] (50) to (51);
		\draw [style=simple] (47) to (48);
		\draw [style=simple] (48) to (52);
		\draw [style=simple] (17) to (18);
		\draw [style=simple] (13) to (17);
		\draw [style=simple] (66) to (65);
		\draw [style=simple] (65) to (67);
		\draw [style=simple] (74) to (73);
		\draw [style=simple] (73) to (76);
		\draw [style=simple] (85) to (84);
		\draw [style=simple] (84) to (83);
		\draw [style=yo] (21) to (86);
		\draw [style=yo] (21) to (87);
		\draw [style=yo] (25) to (88);
		\draw [style=yo] (25) to (89);
		\draw [style=yo] (31) to (90);
		\draw [style=yo] (31) to (91);
		\draw [style=yo] (4) to (92);
		\draw [style=yo] (4) to (93);
		\draw [style=yo] (10) to (94);
		\draw [style=yo] (10) to (95);
		\draw [style=yo] (32) to (96);
		\draw [style=yo] (32) to (97);
		\draw [style=yo] (54) to (100);
		\draw [style=yo] (54) to (101);
		\draw [style=yo] (56) to (98);
		\draw [style=yo] (56) to (99);
		\draw [style=yo] (58) to (102);
		\draw [style=yo] (58) to (103);
		\draw [style=yo] (12) to (104);
		\draw [style=yo] (12) to (105);
		\draw [style=yo] (15) to (106);
		\draw [style=yo] (15) to (107);
		\draw [style=yo] (62) to (110);
		\draw [style=yo] (62) to (111);
		\draw [style=yo] (68) to (108);
		\draw [style=yo] (68) to (109);
		\draw [style=yo] (70) to (112);
		\draw [style=yo] (70) to (113);
		\draw [style=yo] (80) to (114);
		\draw [style=yo] (80) to (115);
	\end{pgfonlayer}
\end{tikzpicture}
\caption{Nim-values for subgames with one, two, or three boxes, and their corresponding Nimdots value.}
\end{figure}



\end{document}






















